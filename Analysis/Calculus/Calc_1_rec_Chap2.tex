\documentclass[Calculus 1 Recitation.tex]{subfiles}

\begin{document}
\section{Derivatives}
\subsection{Derivatives and Rates of Change}
\begin{myleftlinebox}
	tangent line to the curve $f(x)$ at the point $P(a,f(a))$
	\tcblower
	The line through $P$ with slope:
	\[m=\lim_{x\to a} \frac{f(x)-f(a)}{x-a}\]
	provided that this limit exists. 
	
	Notice that we can define $g(x,a)=\frac{f(x)-f(a)}{x-a}$ and it will be the slope of the secant line passing points $(a,f(a))$ and $(x,f(x))$.

	Another form can be obtained by a change of variable. Letting $x=a+h$, we have
	\[m=\lim_{h\to 0}\frac{f(a+h)-f(a)}{h}\]
\end{myleftlinebox}

\begin{myleftlinebox}
	average velocity and instantaneous velocity
	\tcblower
	With a displacement function, or a position function $x(t)$, we define the following:
	\begin{itemize}
		\item Average velocity:
		\(\bar v(t)=\frac{x(t+\Delta t)-x(t)}{\Delta t}\)
		\item Instantaneous velocity (or just call it velocity):
		\(\displaystyle v(t)=\lim_{\Delta t\to 0}\bar v(t)\)
		\item Speed: the absolute value of velocity $\abs{v(t)}$
	\end{itemize}	
\end{myleftlinebox}

\begin{myleftlinebox}
	the derivative of a function $f$ at a point $a$, denoted by $f'(a)$
	\tcblower
	\[f'(a)=\lim_{x\to a} \frac{f(x)-f(a)}{x-a}=\lim_{h\to 0}\frac{f(a+h)-f(a)}{h}\]
	if the limit exists.
\end{myleftlinebox}

\begin{myleftlinebox}
	rate of change of $y$ with respect to $x$
	\tcblower
	\begin{itemize}
		\item Average rate of change of $y$ with respect to $x$:
		\(\frac{\Delta y}{\Delta x}=\frac{f(x_1+\Delta x)-f(x_1)}{\Delta x}=\frac{f(x_2)-f(x_1)}{x_2-x_1}\)
		\item Instantaneous rate of change of $y$ with respect to $x$: 
		\(\displaystyle \lim_{\Delta x\to 0}\frac{\Delta y}{\Delta x}=\lim_{x_2\to x_1}\frac{f(x_2)-f(x_1)}{x_2-x_1}\)
	\end{itemize}	
\end{myleftlinebox}

\subsection{The Derivative as a Function}

\begin{myleftlinebox}
	the derivative of a function $f$
	\tcblower
	\[f'(x)=\lim_{h\to 0}\frac{f(x+h)-f(x)}{h}\]
	and it is defined where $f'(x)$ exists. Other notations
	\[f'(x)=y'=\frac{\dif y}{\dif x}=\frac{\dif f(x)}{\dif x}=\frac{\dif }{\dif x} f(x)=Df(x)=D_x f(x)\]
	\[f'(a)=\given{\frac{\dif y}{\dif x}}_{x=a}=\givenAlt{\frac{\dif y}{\dif x}}_{x=a}\]
\end{myleftlinebox}

\begin{myleftlinebox}
	differentiable function
	\tcblower
	If $f'(a)$ exists, we say $f$ is differentiable at point $x=a$.
\end{myleftlinebox}

\begin{myleftlinebox}
	differentiability implies continuity
	\tcblower
	\begin{theorem}\label{thm:diffToContin}
		If $f$ is differentiable at $a$, then $f$ is continuous at $a$.
	\end{theorem}
	\begin{proof}
		\begin{equation*}
			\begin{split}
				\lim_{h\to 0}f(x+h)-f(x)&=\lim_{h\to 0} (f(x+h)-f(x))\Pare{\frac{h}{h}}\\
				&= \lim_{h\to 0}h \frac{f(x+h)-f(x)}{h}\\
				&= 0f'(x)=0
			\end{split}
		\end{equation*}
	\end{proof}
	And the converse is not true.
\end{myleftlinebox}

\begin{myleftlinebox}
	cases of function that is not differentiable at point $x=a$.
	\tcblower
	\begin{itemize}
		\item $f(a)$ doesn't exist
		\item $\lim_{x\to a}f(x)$ doesn't exist
		\item $\lim_{x\to a}f(x)\neq f(a)$ 
		\item $\lim_{h\to 0} \frac{f(x+h)-f(x)}{h}$ doesn't exist\begin{itemize}
			\item limit is infinity: notice in this case we say the function has a vertical tangent line and it's not the vertical asymptote.
			\item left limit does not equal to right limit
		\end{itemize}
	\end{itemize}
	The first three cases are for discontinuity.
\end{myleftlinebox}

\begin{myleftlinebox}
	higher derivatives
	\tcblower
	The $n$th derivative of $f(x)$ is denoted by $f^{(n)}(x)$ and is obtained from $f$ by differentiating $n$ times. Other notations:

	\[f^{(n)}(x)=y^{(n)}=\frac{\dif^n y}{\dif x^n}\]

	\[f''(x)=(f'(x))'\]

	In the case of position function $x(t)$, we call
	\begin{itemize}
		\item position $x(t)$
		\item velocity $x'(t)=v(t)=\dot x(t)$
		\item acceleration $x''(t)=a(t)=\dot v(t)=\ddot x(t)$
		\item jerk $x'''(t)=j(t)=\dot a(t)$
	\end{itemize}
\end{myleftlinebox}

\subsection{Differentiation Formulas}

\begin{myleftlinebox}
	derivatives of constant function and power function
	\tcblower
	\begin{itemize}
		\item Constant function $f(x)=c$, $f'=0$
		\item Power function $f(x)=x^n$, $f'=nx^{n-1}$, where $n$ is a positive integer. And this is also true for any real number $n$.
	\end{itemize}
\end{myleftlinebox}

\begin{myleftlinebox}
	rules
	\tcblower
	\begin{itemize}
		\item $(f\pm g)'=f'\pm g'$
		\item $(fg)'=f'g+fg'$
		\item $(f/g)'=\frac{f'g-fg'}{g^2}$
	\end{itemize}
\end{myleftlinebox}

\subsection{Derivatives of Trigonometric Functions}

\begin{myleftlinebox}
	derivatives of trigonometric functions
	\tcblower
	\begin{itemize}
		\item $\cos'(x)=-\sin(x)$
		\item $\sin'(x)=\cos(x)$
	\end{itemize}
\end{myleftlinebox}

\begin{myleftlinebox}
	two special limits
	\tcblower
	\begin{itemize}
		\item $\sin(x)/x\to 1$ as $x\to 0$
		\item $(\cos(x)-1)/x\to 1$ as $x\to 0$
	\end{itemize}
\end{myleftlinebox}

\subsection{Exercises}

\begin{myleftlinebox}
	use the definition of limit to prove the sum, difference, product and quotient rule for derivatives
	\tcblower
	\vspace{3em}
\end{myleftlinebox}

\begin{myleftlinebox}
	find the derivatives of the functions of general types: linear, polynomial, power, rational, trigonometric, exponential and logarithmic
	\tcblower
	\vspace{3em}
\end{myleftlinebox}

\begin{myleftlinebox}
	find the tangent line at point $x=1$ for the following functions: $y=\frac{3}{x^2}$, $y=\sqrt{5x-3}$, $y=\frac{x+1}{x+2}$, $y=6$
	\tcblower
	\vspace{3em}
\end{myleftlinebox}

\begin{myleftlinebox}
	If $f$ is odd, what about $f'$? And what if $f$ is even?
	\tcblower
	\vspace{2em}
\end{myleftlinebox}

\end{document}