\documentclass[Calculus 1 Recitation.tex]{subfiles}

\begin{document}
\section{Applications of Integration}
\subsection{Areas Between Curves}

\begin{myleftlinebox}
	area between curves
	\tcblower
	The area between curves $f(x)$ and $g(x)$ and between $x=a$ and $x=b$ is 
	\[A=\int_a^b \abs{f(x)-g(x)}\dif x\]
	Similarly we have the area between curves $h(y)$ and $l(y)$ and between $y=c$ and $y=d$ is
	\[A = \int_c^d \abs{h(y)-l(y)}\dif y\]
\end{myleftlinebox}

\subsection{Volumes}
\begin{myleftlinebox}
	definition of Volume
	\tcblower
	Let $S$ be a solid that lies between $x=a$ and $x=b$. If the cross-sectional area of $S$ in the plane $P_x$ through $x$ and perpendicular to the $x$-axis, is $A(x)$, where $A$ is a continuous function, then the volume of $S$ can be defined by
	\[V = \lim_{n\to\infty} \sum_{i=1}^n A(x_i^*)\Delta x=\int_a^b A(x)\dif x\]
	Alternatively, if the solid lies between $y=c$ and $y=d$ and the cross-sectional area perpendicular to the $y$-axis is $B(y)$ and $B$ is continuous, then we have \[V = \int_c^d B(y)\dif y.\]
\end{myleftlinebox}

\begin{myleftlinebox}
	formula for the volumes of solids of revolution
	\tcblower
	For an area bounded by $f(x)<g(x)$ and $x=a$, $x=b$, the volume of the solid obtained by revolving this area about the $x$-axis is
	\[\int_a^b \pi \Pare{g^2(x)-f^2(x)}\dif x.\]
	Similarly, for an area bounded by $f(y)<g(y)$ and $y=c$, $y=d$, the volume of the solid obtained by revolving this area about the $y$-axis is
	\[\int_c^d \pi \Pare{g^2(y)-f^2(y)}\dif y.\]
\end{myleftlinebox}\

\subsection{Volumes by Cylindrical Shells}
\begin{myleftlinebox}
	the method of cylindrical shells
	\tcblower
	The volume of a cylinder shell is $V=\pi(r_2^2-r_1^2)h=2\pi rh\Delta r$ where $r=\frac{r_1+r_2}{2}$ is the average radius of the shell, $\Delta r=r_2-r_1$ is the thickness of the shell.

	And with this we can derive that the volume of the solid obtained by rotating the area bounded by $y=0$, $y=f(x)$, $x=a$ and $x=b$, around $y$-axis, is
	\[V=\int_a^b 2\pi xf(x) \dif x\]

	However if the region is more easily expressed by left and right boundaries $x=g_1(y)$ and $x=g_2(y)$, the method of disks is preferable. The volume of a disk is 
	\[V = \int_{g(a)}^{g(b)} \pi(g_1(y)^2-g_2(y)^2)\dif y\]
\end{myleftlinebox}

\subsection{Work}
\begin{myleftlinebox}
	Work
	\tcblower
	Work is the energy transferred to or from an object via the application of force along a displacement.
	In classical mechanics, we define force to be the time derivative of the moment, namely,
	\[\bfF = \frac{\dif}{\dif t} (m\bfv)\]
	And work
	\[W = \int_{t_1}^{t_2} \bfF\cdot \bfv \dif t\]
	Here $m$ is the mass of an object, a scalar; $\bfv$ is its velocity, a vector; $t$ is time, a scalar; $\bfF$ is also a vector, $\cdot$ is vector doc product.
\end{myleftlinebox}

\subsection{Average Value of a Function}
\begin{myleftlinebox}
	average value of a function on the interval
	\tcblower
	we define the average value of $f$ on $[a,b]$ as
	\[\frac{1}{b-a}\int_a^b f(x)\dif x\]
\end{myleftlinebox}

\begin{myleftlinebox}
	The Mean Value Theorem for Integrals
	\tcblower
	\begin{theorem}
		If $f$ is continuous on $[a,b]$, then $\exists c\in[a,b]$ such that
		\[f(c) = \frac{1}{b-a}\int_a^b f(x)\dif x\]
	\end{theorem}
\end{myleftlinebox}

\subsection{Exercises}
\begin{myleftlinebox}
	Find the area of the triangle enclosed by $y=x$, $y=2x$ and $y=2-x$ using integration, both on $x$ axis and on $y$ axis.
	\tcblower
	\vspace{2em}
\end{myleftlinebox}

\begin{myleftlinebox}
	Consider the area under the function $y=4-x^2$ from $x=0$ to $x=2$. Find the volumes of the solids obtained by revolving it around $x$-axis and $y$-axis.
	\tcblower
	\vspace{2em}
\end{myleftlinebox}

\begin{myleftlinebox}
	Consider the area bounded by $y=1/x$, $y=2x$ and $y=3$. Find the volumes of the solids obtained by revolving it around $x$-axis and $y$-axis.
	\tcblower
	\vspace{2em}
\end{myleftlinebox}

\begin{myleftlinebox}
	Now you have the tool for the volumes, and in fact, it can also be used to find the surface areas. Derive the formula of the surface area of a unit ball is 3 dimension.
	\tcblower
	\vspace{2em}
\end{myleftlinebox}

\begin{myleftlinebox}
	Consider the area bounded by $y=x$, $y=\sqrt{x}$, $x=0$ and $x=1$. Find the volume and the surface area of solid obtained by revolving this area around $y=x$.
	\tcblower
	\vspace{2em}
\end{myleftlinebox}

\begin{myleftlinebox}
	Consider the area bounded by $y=\sqrt{x}$, $y=x^2$, $x=0$ and $x=1$. Find the volume of the solid obtained by revolving this area abound $y=-1$ using method of cylinder shells and method of disks.
	\tcblower
	\vspace{2em}
\end{myleftlinebox}

\begin{myleftlinebox}
	Consider the area bounded by $y=x$, $y=x^2$, $x=0.5$ and $x=1$. Find the volumes of the solids obtained by revolving this area abound $y=-1$, $y=2$, $x=-1$ and $x=2$.
	\tcblower
	\vspace{2em}
\end{myleftlinebox}

\begin{myleftlinebox}
	Consider an object of mass $m$ with a constant force $\bfF$ applied to it. It's initial speed is $v_0$ at $t_0$. Find its velocity at $t_1$ and work done by $\bfF$.
	\tcblower
	\vspace{2em}
\end{myleftlinebox}

\begin{myleftlinebox}
	A ball of \SI{1}{kg} is dropped at \SI{100}{m} above the ground. Suppose the gravitational constant is\SI{10}{m/s^2}. Find the average speed during the whole falling time, then find the average speed starting from the time when the ball is at the height of \SI{50}{m} above the ground, to the time when the ball is at the ground.
	\tcblower
	\vspace{2em}
\end{myleftlinebox}

\begin{myleftlinebox}
	Prove the Mean Value Theorem for Integrals using the Mean Value Theorem.
	\tcblower
	\vspace{2em}
\end{myleftlinebox}

\end{document}