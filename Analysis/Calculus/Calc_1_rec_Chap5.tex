\documentclass[Calculus 1 Recitation.tex]{subfiles}

\begin{document}
\section{Applications of Integration}
\subsection{Areas Between Curves}

\begin{myleftlinebox}
	area between curves
	\tcblower
	The area between curves $f(x)$ and $g(x)$ and between $x=a$ and $x=b$ is 
	\[A=\int_a^b \abs{f(x)-g(x)}\dif x\]
	Similarly we have the area between curves $h(y)$ and $l(y)$ and between $y=c$ and $y=d$ is
	\[A = \int_c^d \abs{h(y)-l(y)}\dif y\]
\end{myleftlinebox}

\subsection{Volumes}
\begin{myleftlinebox}
	definition of Volume
	\tcblower
	Let $S$ be a solid that lies between $x=a$ and $x=b$. If the cross-sectional area of $S$ in the plane $P_x$ through $x$ and perpendicular to the $x$-axis, is $A(x)$, where $A$ is a continuous function, then the volume of $S$ can be defined by
	\[V = \lim_{n\to\infty} \sum_{i=1}^n A(x_i^*)\Delta x=\int_a^b A(x)\dif x\]
	Alternatively, if the solid lies between $y=c$ and $y=d$ and the cross-sectional area perpendicular to the $y$-axis is $B(y)$ and $B$ is continuous, then we have \[V = \int_c^d B(y)\dif y.\]
\end{myleftlinebox}

\begin{myleftlinebox}
	formula for the volumes of solids of revolution
	\tcblower
	For an area bounded by $f(x)<g(x)$ and $x=a$, $x=b$, the volume of the solid obtained by revolving this area about the $x$-axis is
	\[\int_a^b \pi \Pare{g^2(x)-f^2(x)}\dif x.\]
	Similarly, for an area bounded by $f(y)<g(y)$ and $y=c$, $y=d$, the volume of the solid obtained by revolving this area about the $y$-axis is
	\[\int_c^d \pi \Pare{g^2(y)-f^2(y)}\dif y.\]
\end{myleftlinebox}

\subsection{Exercises}
\begin{myleftlinebox}
	Find the area of the triangle enclosed by $y=x$, $y=2x$ and $y=2-x$ using integration, both on $x$ axis and on $y$ axis.
	\tcblower
	\vspace{2em}
\end{myleftlinebox}

\begin{myleftlinebox}
	Consider the area under the function $y=4-x^2$ from $x=0$ to $x=2$. Find the volumes of the solids obtained by revolving it around $x$-axis and $y$-axis.
	\tcblower
	\vspace{2em}
\end{myleftlinebox}

\begin{myleftlinebox}
	Consider the area bounded by $y=1/x$, $y=2x$ and $y=3$. Find the volumes of the solids obtained by revolving it around $x$-axis and $y$-axis.
	\tcblower
	\vspace{2em}
\end{myleftlinebox}

\begin{myleftlinebox}
	Now you have the tool for the volumes, and in fact, it can also be used to find the surface areas. Derive the formula of the surface area of a unit ball is 3 dimension.
	\tcblower
	\vspace{2em}
\end{myleftlinebox}

\end{document}