\documentclass[Calculus 1 Recitation.tex]{subfiles}

\begin{document}
\section{Functions and Limits}
\subsection{Four Ways to Represent a Function}
\begin{myleftlinebox}
	function, it's domain and range
	\tcblower
	function $y=f(x)$ and $X, Y$ are two sets. $x\in X$ and $y\in Y$. A function is a rule that assigns each $x\in X$ to exactly one element $y=f(x)\in Y$. Domain of a function $X$ is the set of inputs accepted by the function so that the function makes sense.
	\begin{itemize}
		\item age, height are positive by default
		\item $1/0$ doesn't make sense
	\end{itemize}	
	range can be obtained after you have the domain $Y=\{y=f(x) \mid x\in X\}$.
\end{myleftlinebox}

\begin{myleftlinebox}
	piecewise defined function
	\tcblower
	Divide domain of $f$ into subsets $X_1,X_2,\dots$ without intersection, so $X_i\cap X_j=\emptyset$. Then, 
	\[f(x) = \begin{cases}
		f_1(x), & x\in X_1\\
		f_2(x), & x\in X_2\\
		\vdots
	\end{cases}\]
\end{myleftlinebox}

\begin{myleftlinebox}
	even function and odd function
	\tcblower
	function $f(x)$ is even if $f(x)=f(-x)$ and is odd if $f(x)=-f(x)$, for any $x$ in its domain.
\end{myleftlinebox}

\begin{myleftlinebox}
	increasing and decreasing
	\tcblower
	function $f(x)$ is increasing on an interval $I$ if $f(x_1)\geq f(x_2)$, for any $x_1, x_2\in I$, $x_1>x_2$. 

	function $f(x)$ is decreasing on an interval $I$ if $f(x_1)\leq f(x_2)$, for any $x_1, x_2\in I$, $x_1>x_2$. 
\end{myleftlinebox}


\subsection{Mathematical Models}
\begin{myleftlinebox}
	mathematical model
	\tcblower
	math description of a phenomenon, usually by equations. For example someone's age vs year.
	\begin{center}
		\begin{tabular}{c|c}
			\hline
			year (at Jan. 1st) & age \\\hline
			2020 & 1\\
			2021 & 2\\
			2022 & 3\\
			\hline
		\end{tabular}
	\end{center}
\end{myleftlinebox}

\begin{myleftlinebox}
	linear function: $y$ is a linear function of $x$.
	\tcblower
	when changes of $y$ is proportion to the changes of $x$. For an arbitrary point belong to this relationship, $(x,y)$ and a fixed point $(x_1,y_1)$ also belong to this relationship,
	\begin{align*}
		y-y_1 &= m(x-x_1),k\neq 0\\
		y &= mx-mx_1+y_1 = mx+b
	\end{align*}
	we call $m$ the slope and b the $y$-intercept.
\end{myleftlinebox}

\begin{myleftlinebox}
	polynomial function: $P(x)$
	\tcblower
	$P(x) = a_n x^{n}+a_{n-1} x^{n-1}+\cdots+a_1 x + a_0$, $n$ is nonnegative integer, $a_n\neq 0$ and its degree is $n$. A linear function is a polynomial of degree 1, and a quadratic function is a polynomial of degree 2, cubic for 3.
\end{myleftlinebox}

\begin{myleftlinebox}
	power function
	\tcblower
	$f(x) = x^a$. If $a$ is an integer $n$,
	\begin{itemize}
		\item if $n$ is even, then $f(x)$ is even for $f(x)=f(-x)$
		\item if $n$ is odd, then $f(x)$ is odd for $f(x)=-f(-x)$
	\end{itemize}	
	If $a=1/n$ with $n$ a positive integer, $f(x)=x^a$ is a root function. If $a=-1$, $f(x)=x^{-1}$ is a reciprocal function.
\end{myleftlinebox}


\begin{myleftlinebox}
	rational function
	\tcblower
	$f(x) = P(x)/Q(x)$, where $P$ and $Q$ are two polynomial functions.
\end{myleftlinebox}

\begin{myleftlinebox}
	Trigonometric function
	\tcblower
	Given a right triangle, an angle $x$ in radian measure. For this angle $x$, define the opposite side to be the side opposite to $x$, with length $a$, define the adjacent side to be the side between $x$ and the right angle, with length $b$, and the hypotenuse side to be the side opposite to the right angle, with length $c$, then
	
	$\sin(x) = a/c$, $\cos(x)=b/c$, and $\tan(x) = a/b$; then $\csc(x)=1/\sin(x)$, $\sec(x)=1/\cos(x)$ and $\cot(x)=1/	\tan(x)$.
	\begin{center}
		\usetikzlibrary{angles,quotes}
		\begin{tikzpicture}[
			my angle/.style = {
				draw, angle radius=7mm, 
				angle eccentricity=1.1, 
				right, inner sep=1pt,
				font=\footnotesize
				}
			]
			\coordinate (a) at (0,0);
			\coordinate (c) at (4,0);
			\coordinate (b) at (4,2);
			\draw (a) -- (c)node[midway, below]{$b$, adjacent};
			\draw (b) -- (c)node[midway,right]{$a$, opposite};
			\draw (b) -- (a)node[midway,left, above,rotate=26.5]{$c$, hypotenuse}; % Triangle.
			\draw (a) node[anchor=east,align=center] {$A$};
			\draw (b) node[anchor=west,align=center] {$B$};
			\draw (c) node[anchor=west]{$C$};
			\pic[my angle,"$x$"] {angle = c--a--b};
		\end{tikzpicture}
	\end{center}
\end{myleftlinebox}

\begin{myleftlinebox}
	period function
	\tcblower
	if $f(x)=f(x+T)$, where $T$ is a constant.
\end{myleftlinebox}

\begin{myleftlinebox}
	algebraic function, and transcendental if not
	\tcblower
	functions constructed using addition, subtraction, multiplication, division, raising to a whole number power, and taking roots. And some transcendental functions: $\sin(x)$, $\log(x)$, $e^x$.
\end{myleftlinebox}

\begin{myleftlinebox}
	Exponential and Logarithmic function
	\tcblower
	Exponential function has the form $y=f(x)=b^x$, and logarithmic function has the form $y=f(x)=\log_b x = \log x/\log b$
\end{myleftlinebox}

\subsection{New Functions from Old Functions}

\begin{myleftlinebox}
	shifting a function $f(x)$
	\tcblower
	new function $g(x) = f(x-h)+v$, has the plot same as shift the plot of $f(x)$ $v$ units to the right vertically and $h$ units upward horizontally. Notice when $h<0$, shifting $h$ units to the right equals to shifting $|h|$ units to the left.
\end{myleftlinebox}

\begin{myleftlinebox}
	stretching and reflecting a function $f(x)$
	\tcblower
	new function $g(x) = v\times f(x/h)$, has the plot same as stretch the plot of $f(x)$ by a factor of $v$ vertically and $h$ units horizontally. 
	
	If $v<0$, we do a reflection about the line $y=0$ (x-axis) first then stretch by a factor of $|v|$ vertically, and if $h<0$, we do a reflection about the line $x=0$ (y-axis) first then stretch by a factor of $|h|$ horizontally.
	
	if $|v|<1$ or $|h|<1$, it's a shrinking operation, not stretching.
\end{myleftlinebox}


\begin{myleftlinebox}
	combination of functions $f(x)$ and $g(x)$
	\tcblower
	$f+g$ sum, $f-g$ difference, $fg$ product, and $f/g$ quotient. The domain of the new function is the intersection of domains of $f$ and $g$.
\end{myleftlinebox}

\begin{myleftlinebox}
	composition of functions $f(x)$ and $g(x)$
	\tcblower
	$(f\circ g)(x) = f(g(x))$. To make sure $x$ is accepted, first $g(x)$ need to make sense, then exclude those $x$ so that if letting $z=g(x)$, $f(z)$ make sense.
	\begin{enumerate}
		\item Let $z=g(x)$ so that $f\circ g(x) = f(z)$
		\item find domain of $f(z)$, say set/interval $Z$
		\item simplify $z=g(x)\in Z$ and obtain $x\in X_1$
		\item find domain of $g(x)$, $X_2$
		\item the domain of $f\circ g(x)$ is then $X_1\cap X_2$
	\end{enumerate}
\end{myleftlinebox}

\subsection{The Tangent and Velocity Problems}

\begin{myleftlinebox}
	secant line and tangent line
	\tcblower
	With a given curve $C$, a secant line is a line passing through two points of a curve. In most cases, as one point is brought towards the other, the secant line tends to be the tangent line at the other point.

	The slope of the tangent line is the limit of the slopes of the secant
	lines.
\end{myleftlinebox}

\begin{myleftlinebox}
	difference quotient
	\tcblower
	\[\frac{f(x+h)-f(x)}{h}, h\neq0\]
\end{myleftlinebox}


\subsection{The Limit of a Function}\label{sec:funcLimit}

\begin{myleftlinebox}
	the limit of $f(x)$ as $x$ approaches $a$
	\tcblower
	Suppose $f(x)$ is defined when $x$ is near the number $a$, Then we write
	\[\lim_{x\to a}f(x)=L\]

	or \(f(x)\to L\) as $x\to a$, if we can make the values of $f(x)$ arbitrarily close to $L$ 	by restricting $x$ to be sufficiently close to $a$ (on either side of $a$) but not equal to $a$. 

	In $\epsilon-\delta$ language, the condition is: $\forall \epsilon > 0, \exists \delta > 0$ s.t. $\forall x, 0 < |x - a| < \delta$ we have $|f(x) - L| < \epsilon$.

	If $f(x)$ is only defined on one side of $a$, check the next definition.
\end{myleftlinebox}

\begin{myleftlinebox}
	left-side limits
	\tcblower
	Suppose $f(x)$ is defined when $x$ is near the number $a$, and also $x<a$. Then we write
	\[\lim_{x\to a^-}f(x)=L\]

	or \(f(x)\to L\) as $x\to a^-$, if we can make the values of $f(x)$ arbitrarily close to $L$ by restricting $x$ to be sufficiently close to $a$ (on left side of $a$) but not equal to $a$. 

	In $\epsilon-\delta$ language, the condition is: $\forall \epsilon > 0, \exists \delta > 0$ s.t. $\forall x, 0 < a-x < \delta$ we have $|f(x) - L| < \epsilon$.

	Remarks:
	\begin{itemize}
		\item Right-side limit is defined in a similar way. 
		\item $\lim_{x\to a}f(x)=L \iff \lim_{x\to a^-}f(x)=\lim_{x\to a^+}f(x)=L$
		\item If the domain of $f$ stays in one side of $a$, for example $f(x)=\frac{x}{\sqrt{x}}$, the limit at $a$ can still be defined properly to be the side limit. Thus $\lim_{x\to 0}\frac{x}{\sqrt{x}}=0$.
	\end{itemize}
\end{myleftlinebox}

\begin{myleftlinebox}
	infinite limits
	\tcblower
	In this case the limit doesn't exist.

	Suppose $f(x)$ is defined when $x$ is near the number $a$, Then we write
	\[\lim_{x\to a}f(x)=\infty\]

	or \(f(x)\to \infty\) as $x\to a$, if we can make the values of $f(x)$ arbitrarily large by restricting $x$ to be sufficiently close to $a$ (on either side of $a$) but not equal to $a$. 

	In $\epsilon-\delta$ language, the condition is: $\forall M > 0, \exists \delta > 0$ s.t. $\forall x, 0 < |x - a| < \delta$ we have $f(x)>M$.

	Negative infinite limits, one-sided infinite limits can be defined similarly.
	
	And then we call $x=a$ the vertical asymptote of the curve $f(x)$.
\end{myleftlinebox}


\subsection{Calculating Limits using the Limit Laws}


\begin{myleftlinebox}
	limit laws
	\tcblower
	Following operation is interchangeable with finding the limits.
	\begin{itemize}
		\item summation
		\item difference
		\item scalar multiplication
		\item product
		\item quotient (excluding the case where the denominator has limit $0$)
		\item power to $n$ or $1/n$, $n$ is any positive integer
	\end{itemize}
\end{myleftlinebox}

\begin{myleftlinebox}
	direct substitution property
	\tcblower
	if the function limit at $a$ is equal to the function value at $a$. And we call the function is continuous at $a$ if this property hold.
\end{myleftlinebox}

\begin{myleftlinebox}
	the Squeeze theorem
	\tcblower
	\begin{lemma}
		if $f(x)\leq g(x)$ when $x$ is near $a$, and the limits of $f$ and $g$ exist at $a$, then $\lim_{x\to a} f(x)\leq \lim_{x\to a} g(x)$.
	\end{lemma}
	\begin{theorem}
		With this, we can show that if $f\leq g\leq h$ when $x$ is near $a$, and $\lim_{x\to a} f(x) = \lim_{x\to a} h(x)=L$, then $\lim_{x\to a} g(x)=L$.
	\end{theorem}
\end{myleftlinebox}


\subsection{The Precise Definition of a Limit}

see the $\epsilon-\delta$ language part in \autoref{sec:funcLimit}. For $\lim_{x\to a}f(x)=L$, we mean that $f(x)$ will approach $L$, as long as we move $x$ to a point close enough to $a$, but not $a$ itself.

\begin{itemize}
	\item $a$, the target point, where we want to know about the function limit
	\item $\epsilon$ measures how close the function value $f(x)$ at the moving point $x$ to the limit $L$, with $\abs{f(x)-L}<\epsilon$
	\item  we want $\epsilon$ arbitrarily small, so $\epsilon$ can be any positive number
	\item $\delta$ measures how close the moving point $x$ to the target point $a$, with $0<\abs{x-a}<\delta$
	\item In many simple cases, $\delta$ is a function of $\epsilon$. So if we can prove that $0<\abs{x-a}<\delta(\epsilon)$ leads to $\abs{f(x)-L}<\epsilon$, then the limit $L$ is found.
\end{itemize}

The easiest case is when $f(x)=x, x\neq a$, we can consider $\delta(\epsilon)=\epsilon$ so that we have $\lim_{x\to a} f(x)=a$.

\subsection{Continuity}

\begin{myleftlinebox}
	point continuity
	\tcblower
	A function $f$ is continuous at $x=a$ if $\lim_{x\to a} f(x) = f(a)$
\end{myleftlinebox}

\begin{myleftlinebox}
	type of discontinuity at a point $x=a$
	\tcblower
	\begin{itemize}
		\item removable: when we can define a new value for $f(a)$ so that $f(x)$ can regain continuity at $x=a$. (a pothole)
		\item jump: when the left limit and the right limit exist but they are not equal (stairs)
		\item infinite: when the left limit or the right limit doesn't exist. ($1/x$ and $\sin(1/x)$)
	\end{itemize}
\end{myleftlinebox}

\begin{myleftlinebox}
	left continuity and interval continuity
	\tcblower
	A function $f$ is continuous from the right at a point $x=a$ if $\lim_{x\to a^+} f(x) = f(a)$.

	A function $f$ is continuous on an interval if it is continuous at every number in the interval.
\end{myleftlinebox}

\begin{myleftlinebox}
	Properties of continuous function
	\tcblower
	\begin{theorem}\label{thm:continFunc}
		If $f$ and $g$ are both continuous at $a$, then their combination $f\pm g$, $fg$, $f/g$ where $g(a)\neq 0$, are continuous at $a$.

		If $g$ is continuous at $a$ and $f$ is continuous at $g(a)$, then the composition $f(g(x))$ is continuous at $a$.
	\end{theorem}
\end{myleftlinebox}

\begin{myleftlinebox}
	Intermediate Value Theorem
	\tcblower
	\begin{theorem}\label{thm:intermediateValue}
		Suppose that $f$ is continuous on the closed interval $[a, b]$ and let $N$ be any number between $f(a)$ and $f(b)$, where $f(a) \neq f(b)$. Then there exists a number c in $(a, b)$ such that $f(c) = N$.
	\end{theorem}
	meaning a continuous function takes on every intermediate value between the function values at two ends of the interval.
\end{myleftlinebox}


\subsection{Exercises}

\begin{myleftlinebox}
	Define $y(x) = \frac{x^2-x}{2(x-2)}$ and $g(x) = \frac{x}{2}$. State the difference between them and plot them.
	\tcblower
	\vspace{2em}
\end{myleftlinebox}

\begin{myleftlinebox}
	Given that $y=x^2$, is $y$ a function of $x$? is $x$ a function of $y$? If not, add a restriction to make it/them function(s) and plot them.
	\tcblower
	\vspace{2em}
\end{myleftlinebox}


\begin{myleftlinebox}
	are the following functions odd, even, or neither. \(x^2\) where \(x>0\), \(\tan(x+\pi/4)\), \(x^{20}\), \(x^{-20}\), \(e^{x^3}\), $x^x$ where $x$ is a integer.
	\tcblower
	\vspace{5em}
\end{myleftlinebox}

\begin{myleftlinebox}
	transform the function $f(x)=\frac{1}{2x}+2$ to $g(x)=-\frac{1}{x-2}$, and plot what you gain at each step.
	\tcblower
	\vspace{2em}
\end{myleftlinebox}

\begin{myleftlinebox}
	find difference quotient of \(x^2+2x+3\) and \(\frac{2}{x}\).
	\tcblower
	\vspace{2em}
\end{myleftlinebox}

\begin{myleftlinebox}
	plot function $y=\sgn(x) := \begin{cases}
		1, & x>0\\
		0, & x=0\\
		-1.& x<0
	\end{cases}$ and function $z = g(x)=\begin{cases}
		\abs{x},& x\neq 0\\
		1, &x=0
	\end{cases}$. And then find the limit, left limit and right limit of $f(x)$ and $g(x)$ at $x=0$.
	\tcblower
	\vspace{3em}
\end{myleftlinebox}

\begin{myleftlinebox}
	An object is moving straightforward. The location $x$ and time $t$ has an approximated relation $x = 2t^3+t^2+5t$. What's the average speed from $t=1$ to $t=3$?
	\tcblower
	\vspace{2em}
\end{myleftlinebox}

\begin{myleftlinebox}
	For the previous question, first simplify the difference quotient of the distance function $\frac{x(t+\Delta t)-x(t)}{\Delta t}$ at $t=1$. What's the limit when $\Delta t$ goes to $0$? Also what's the limit of $x(t)$ at $t=1$? Use words to explain their difference.
	\tcblower
	\vspace{2em}
\end{myleftlinebox}


\begin{myleftlinebox}
	Use $\epsilon-\delta$ language to show that $\lim_{x\to 1} 1/x = 1$ and $\lim_{x\to 2} x^3 = 8$.
	\tcblower
	\vspace{2em}
\end{myleftlinebox}

\begin{myleftlinebox}
	Use $\epsilon-\delta$ language to prove $x\sin(1/x)$ has limit $0$ at $x=0$ and $\sin(1/x)$ doesn't have limit at $x=0$. Then use Squeeze theorem to prove the first part.
	\tcblower
	\vspace{2em}
\end{myleftlinebox}

\begin{myleftlinebox}
	Use $\epsilon-\delta$ language to prove that the limit at point $x=a$ doesn't exist when the left limit doesn't equal to the right limit.
	\tcblower
	\vspace{2em}
\end{myleftlinebox}

\begin{myleftlinebox}
	Use $\epsilon-\delta$ language to prove that function of the form $k x^n$ is continuous everywhere, where $k\neq 0$ and $n$ is an integer.
	\tcblower
	\vspace{2em}
\end{myleftlinebox}

\begin{myleftlinebox}
	Use $\epsilon-\delta$ language to prove the properties of the continuous function \autoref {thm:continFunc}.
	\tcblower
	\vspace{2em}
\end{myleftlinebox}

\begin{myleftlinebox}
	Show that there's a solution of the equation $x^4-5=0$ between $0$ and $2$.
	\tcblower
	\vspace{2em}
\end{myleftlinebox}

\begin{myleftlinebox}
	Show function $f(x)=x$ and function $g(x)=\sin(x)$ intersect with each other.
	\tcblower
	\vspace{2em}
\end{myleftlinebox}

\begin{myleftlinebox}
	In the film \emph{The Curious Case of Benjamin Button}, Benjamin was born at the age of 85. His age decreases as time passes until the end of his life at age 0. Now consider somewhere on earth a normal child Xavier was born at the same time when Benjamin was born. Assume Xavier is gonna live for at least 85 years, prove there exists a time when Benjamin and Xavier will be at the same age. Make a story of this kind on your own. Maybe transferring water from cup A to cup B.
	\tcblower
	\vspace{2em}
\end{myleftlinebox}


\end{document}