\documentclass{article}

%%%%%%%%%%%%%%%%% this version: 03-14-2017 %%%%%%%%%%%%%%%%%%%%%%%%%%%%%%%

%%% standard packages
\usepackage{amsthm, amsmath, amssymb, amsfonts, graphicx, epsfig}
%\graphicspath{ {D:/Notes/others/assets/} }

\usepackage{algorithm, algpseudocode}

\allowdisplaybreaks
\usepackage{setspace}
%\usepackage{accents}
%%%
%\usepackage{kbordermatrix}

%%%%%%%%%%%%%%%%%%%%%%% ADDITIONAL FONTS %%%%%%%%%%%%%%%%%%%%%%%%%%%%%%
%%% Package to make \mathbbm, in particular to have 1 as for indicator function
\usepackage{bbm}
%%% Package to make special curl fonts, by using \mathscr{F}
\usepackage{mathrsfs}
%%% dsfont sometimes used instead of mathbb or mathbbm
%\usepackage{dsfont}
%%%%%%%%%%%%%%%%%%%%%%%%%%%%%%%%%%%%%%%%%%%%%%%%%%%%%%%%%%%%%%%%%%%%%%%


%%%%%%%%%%%%%%%%%%%%%%%%%%%%%%%%%%%%%%%%%%%%%%%%%%%%%%%%%%%%%%%%%%%%%%
%%% The ulem package provides various types of underlining that can
%%% stretch between words and be broken across lines. Convenient for editing. \sout{xxxx}
%%% http://ctan.unixbrain.com/macros/latex/contrib/ulem/ulem.pdf
%%%% Remark: if used with natbib, then the bibliography comes underlined. Comment this package at last compilation
%\usepackage{ulem}
%%%%%%%%%%%%%%%%%%%%%%%%%%%%%%%%%%%%%%%%%%%%%%%%%%%%%%%%%%%%%%%%%%%%%%

%%%%%%%%%%%%%%%%%%%%%%%%%%%%%%%%%%%%%%%%%%%%%%%%%%%%%%%%%%%%%%%%%%%%%%
%% Cancel is used to cross out in math mode. \xcancel, \cancel Convenient for editting.
%% http://ctan.math.utah.edu/ctan/tex-archive/macros/latex/contrib/cancel/cancel.pdf
%\usepackage{cancel}
%%%%%%%%%%%%%%%%%%%%%%%%%%%%%%%%%%%%%%%%%%%%%%%%%%%%%%%%%%%%%%%%%%%%%%


%%%%%%%%%%%%%%%%%%%%%%%%%%%%%%%%%%%%%%%%%%%%%%%%%%%%%%%%%%%%%%%%%%%%%%%
%This packages adds support of handling eps images to package graphics
%or graphicx with option pdftex. If an eps image is detected, epstopdf is
%automatically called to convert it to pdf format.
\usepackage{epstopdf}
%%%%%%%%%%%%%%%%%%%%%%%%%%%%%%%%%%%%%%%%%%%%%%%%%%%%%%%%%%%%%%%%%%%%%%%


%%%%%%%%%%%%%%%%%%%%%%%%%%%%%%%%%%%%%%%%%%%%%%%%%%%%%%%%%%%%%%%%%%%%%%%
% various features for using graphics, including subfigure, captions, subcaptions etc.
\usepackage{graphicx}
%\usepackage{caption}
\usepackage[font=sl,labelfont=bf]{caption}
\usepackage{subcaption}
%%%%%%%%%%%%%%%%%%%%%%%%%%%%%%%%%%%%%%%%%%%%%%%%%%%%%%%%%%%%%%%%%%%%%%%


%%%%%%%%%%%%%%%%%%%%%%%%%%%%%%%%%%%%%%%%%%%%%%%%%%%%%%%%%%%%%%%%%%%%%%%
% This package improves the interface for defining floating objects such
% as figures and tables in LaTeX.
% http://www.ctan.org/pkg/float
\usepackage{float}
\restylefloat{table}
%%%%%%%%%%%%%%%%%%%%%%%%%%%%%%%%%%%%%%%%%%%%%%%%%%%%%%%%%%%%%%%%%%%%%%%


%%% use for diagonals in the table's cells
%\usepackage{slashbox}  %Removed by Ares

%%%%%%%%%%%%%%%%%%%%%%%%%%%%%%%%%%%%%%%%%%%%%%%%%%%%%%%%%%%%%%%%%%%%%%%


%%%%%%%%%%%%%%%%%%%%%%%%%%%%%%%%%%%%%%%%%%%%%%%%%%%%%%%%%%%%%%%%%%%%%%%
% This package gives the enumerate environment an optional argument
% which determines the style in which the counter is printed.
% http://www.ctex.org/documents/packages/table/enumerate.pdf
\usepackage{enumerate}
%%%%%%%%%%%%%%%%%%%%%%%%%%%%%%%%%%%%%%%%%%%%%%%%%%%%%%%%%%%%%%%%%%%%%%%


%%%%%%%%%%%%%%%%%%%%%%%%%%%%%%%%%%%%%%%%%%%%%%%%%%%%%%%%%%%%%%%%%%%%%%%
%selectively in/exclude pieces of text: the user can determine new comment versions,
%and each is controlled separately. Special comments can be determined where the
%user specifies the action that is to be taken with each comment line.
% http://get-software.net/macros/latex/contrib/comment/comment.pdf
%\usepackage{comment}
%%%%%%%%%%%%%%%%%%%%%%%%%%%%%%%%%%%%%%%%%%%%%%%%%%%%%%%%%%%%%%%%%%%%%%%


%%%%%%%%%%%%%%%%%%%%%%%%%%%%%%%%%%%%%%%%%%%%%%%%%%%%%%%%%%%%%%%%%%%%%%
%% To display the labels used in a tex file in the dvi file (for example, if a theorem is labelled with the \label command) use the package
%% http://www.ctan.org/pkg/showkeys

%%%%%% Use Option 1
%%%%%% This will display all labels (for example, in the case of a labelled theorem, the label of the theorem will occur in the margin of the dvi file.
%%%%%% The command above by itself not only displays labels when they are first named but also when they are cited (or referenced).
%\usepackage{showkeys}

%%%%%% Use Option 2
%%%%%% It will NOT display citations and references. Only Labels
%\usepackage[notref, notcite]{showkeys}

%%%%%% Use Options 3
%%%%%% formats to smaller fonts the labels etc.
%\usepackage[usenames,dvipsnames]{color}
%\providecommand*\showkeyslabelformat[1]{{\normalfont \tiny#1}}
%\usepackage[notref,notcite,color]{showkeys}
%\definecolor{labelkey}{rgb}{0,0,1}
%%%%%%%%%%%%%%%%%%%%%%%%%%%%%%%%%%%%%%%%%%%%%%%%%%%%%%%%%%%%%%%%%%%%%%


%%%%%%%%%%%%%%%%%%%%%%%%%%%%%%%%%%%%%%%%%%%%%%%%%%%%%%%%%%%%%%%%%%%%%%%
% It extends the functionality of all the LATEX cross-referencing commands (including the table of contents, bibliographies etc) to produce \special commands which a driver can turn into hypertext links; it also provides new commands to allow the user to write ad hoc hypertext links, including those to external documents and URLs.
% http://www.tug.org/applications/hyperref/manual.html
\usepackage[colorlinks=true, pdfstartview=FitV, linkcolor=blue,
            citecolor=blue, urlcolor=blue]{hyperref}
\usepackage[usenames]{color}
%%%%%%%%%%%%%%%%%%%%%%%%%%%%%%%%%%%%%%%%%%%%%%%%%%%%%%%%%%%%%%%%%%%%%%%%
% custom textbox
\usepackage{tcolorbox}
\tcbuselibrary{breakable}
\tcbuselibrary{skins}
\tcbset{my left line/.style={
          enhanced, frame hidden, borderline west = {0.5pt}{0pt}{black}, % specify border
          opacityframe=0, opacityback=0,opacityfill=0, % color
          arc = 0mm, left skip=1em, % border
          left = 1mm,top=0mm,bottom=0mm,boxsep=1mm,middle=1mm, % text and border
}}

\newtcbox{\mybox}[1][]{my left line, #1}
\newtcolorbox{myleftlinebox}[1][breakable]{my left line, #1}
% usage
% use \mybox[on line]{your TEXT} for inline box otherwise forcing line breaks   
% for whole break, use \begin{myleftlinebox
%%%%%%%%%%%%%%%%%%%%%%%%%%%%%%%%%%%%%%%%%%%%%%%%%%%%%%%%%%%%%%%%%%%%%%%
\usepackage[utf8]{inputenc}


%%%%%%%%%%%%%%%%%%%%%%%%%%%%%%%%%%%%%%%%%%%%%%%%%%%%%%%%%%%%%%%%%%%%%%%
% a good looking way to format urls
% http://mirror.its.uidaho.edu/pub/tex-archive/help/Catalogue/entries/url.html
\usepackage{url}
% Define a new 'leo' style for the package that will use a smaller font.
\makeatletter\def\url@leostyle{%
 \@ifundefined{selectfont}{\def\UrlFont{\sf}}{\def\UrlFont{\scriptsize\ttfamily}}} \makeatother\urlstyle{leo}
%%%%%%%%%%%%%%%%%%%%%%%%%%%%%%%%%%%%%%%%%%%%%%%%%%%%%%%%%%%%%%%%%%%%%%%


%%%%%%%%%%%%%%%%%%%%%%%%%%%%%%%%%%%%%%%%%%%%%%%%%%%%%%%%%%%%%%%%%%%%%%%%%%%%
%%%%%% The present package defines the environment mdframed which automatically deals with page breaks in framed text.
%%%%%% http://mirrors.ibiblio.org/CTAN/macros/latex/contrib/mdframed/mdframed.pdf
\usepackage[framemethod=default]{mdframed}
%%%%%%%%%%%%%%%%%%%%%%%%%%%%%%%%%%%%%%%%%%%%%%%%%%%%%%%%%%%%%%%%%%%%%%%%%%%%

%%%%%%%%%%%%%%%%%%%%%%%%%%%%%%%%%%%%%%%%%%%%%%%%%%%%%%%%%%%%%%%%%%%%%%%%%%%%%%%%
%%%%%% An extended implementation of the array and tabular environments
%%%%%% which extends the options for column formats, and provides "programmable" format specifications.
%\usepackage{array}
%%%%%%%%%%%%%%%%%%%%%%%%%%%%%%%%%%%%%%%%%%%%%%%%%%%%%%%%%%%%%%%%%%%%%%%%%%%

%%%%%%%%%%%%%%%%%%%%%%%%%%%%%%%%%%%%%%%%%%%%%%%%%%%%%%%%%%%%%%%%%%%%%%%%%%%%
%%% The package enhances the quality of tables in LATEX, providing extra
%%% commands as well as behind-the-scenes optimization. Guidelines
%%% are given as to what constitutes a good table in this context.
%%% From version 1.61, the package offers long table compatibility.
%\usepackage{booktabs}
%%%%%%%%%%%%%%%%%%%%%%%%%%%%%%%%%%%%%%%%%%%%%%%%%%%%%%%%%%%%%%%%%%%%%%%%%%%%%%


\usepackage{tikz}




%% or use GEOMETRY package
\usepackage[margin=1.0in, letterpaper]{geometry}
%\def\baselinestretch{1.1}


%%%%%%%%%%%%% OR USE EXACT DIMENSIONS %
%\setlength{\textwidth}{6.5in}     %%
%\setlength{\oddsidemargin}{0in}   %%
%\setlength{\evensidemargin}{0in}  %%
%\setlength{\textheight}{8.5in}    %%
%\setlength{\topmargin}{0in}       %%
%\setlength{\headheight}{0in}      %%
%\setlength{\headsep}{.3in}         %%
%\setlength{\footskip}{.5in}       %%
%%%%%%%%%%%%%%%%%%%%%%%%%%%%%%%%%%%%%%%%%%%%%%%%%%%%%%%%%%%%%%%%%%%%%%%


%%%%%%%%%%%%%%%%%%%%%%%%%%%%%%%%%%%%%%%%%%%%%%%%%%%%%%%%%%%%%%%%%%%%%%%
%%%%%%%%%%%%%%%%%%%%%%%% NUMBERING %%%%%%%%%%%%%
\newtheorem{theorem}{Theorem}
\newtheorem{conjecture}{Conjecture}
\newtheorem{conclusion}{Conclusion}
\newtheorem{proposition}[theorem]{Proposition}
\newtheorem{lemma}[theorem]{Lemma}
\newtheorem{corollary}[theorem]{Corollary}
\newtheorem{assumption}{Assumption}
\newtheorem{condition}{C}
\theoremstyle{definition}
\newtheorem{definition}[theorem]{Definition}
\newtheorem{example}[theorem]{Example}
\theoremstyle{remark}
\newtheorem{remark}[theorem]{Remark}
\newtheorem{question}[theorem]{Question}
\newtheorem{problem}[theorem]{Problem}
\newtheorem{NB}[theorem]{Nota Bene}

\numberwithin{equation}{section}
\numberwithin{theorem}{section}
%\renewcommand{\labelitemi}{ {\small $\rhd$}}
%%%%%%%%%%%%%%%%%%%%%%%%%%%%%%%%%%%%%%%%%%%%%%%%%%%%%%%%%%%%%%%%%%%%%%%

%%%%%%%%%%%%%%%%%%%%%%%%%%%%%%%%%%%%%%%%%%%%%%%%%%%%%%%%%%%%%%%%%%%%%%%
\definecolor{Red}{rgb}{0.9,0,0.0}
\definecolor{Blue}{rgb}{0,0.0,1.0}
%%%%%%%%%%%%%%%%%%%%%%%%%%%%%%%%%%%%%
%%% used for editing and making comments in color
% Example \ig{Remarks and Commets}
\newcommand{\ig}[1]{\textcolor[rgb]{0.00, 0.0, 1.0}{{\tiny \textsuperscript{[\textrm{IC:Rem}]}} \ #1}}
\newcommand{\igAdd}[1]{\textcolor[rgb]{0.7, 0.0, 0.0}{{\tiny \textsuperscript{[\textrm{IC:Add}]}} \ #1}}
\newcommand{\igEdit}[1]{\textcolor[rgb]{0.7, 0.0, 0.0}{{\tiny \textsuperscript{[\textrm{IC:Edit}]}} \  #1}}

\newcommand{\trb}[1]{\begin{color}[rgb]{0.98, 0.0, 0.98}{TRB: #1} \end{color}}
\newcommand{\ti}[1]{\begin{color}[rgb]{1.00,0.00,0.00}{TI: #1} \end{color}}
\newcommand{\Red}[1]{\textcolor{Red}{#1}}
%%%%%%%%%%%%%%%%%%%%%%%%%%%%%%%%%%%%%

%%%%%%%%%%%%%%%%%%%%%%%%%%%%%%%%%%%%%
%%%     Igor's macros
%% \mathcal Letters
\def\cA{\mathcal{A}}
\def\cB{\mathcal{B}}
\def\cC{\mathcal{C}}
\def\cD{\mathcal{D}}
\def\cE{\mathcal{E}}
\def\cF{\mathcal{F}}
\def\cG{\mathcal{G}}
\def\cH{\mathcal{H}}
\def\cI{\mathcal{I}}
\def\cJ{\mathcal{J}}
\def\cK{\mathcal{K}}
\def\cL{\mathcal{L}}
\def\cM{\mathcal{M}}
\def\cN{\mathcal{N}}
\def\cO{\mathcal{O}}
\def\cP{\mathcal{P}}
\def\cQ{\mathcal{Q}}
\def\cR{\mathcal{R}}
\def\cS{\mathcal{S}}
\def\cT{\mathcal{T}}
\def\cU{\mathcal{U}}
\def\cV{\mathcal{V}}
\def\cW{\mathcal{W}}
\def\cX{\mathcal{X}}
\def\cY{\mathcal{Y}}
\def\cZ{\mathcal{Z}}

%% \mathbb Letters
\def\bA{\mathbb{A}}
\def\bB{\mathbb{B}}
\def\bC{\mathbb{C}}
\def\bD{\mathbb{D}}
\def\bE{\mathbb{E}}
\def\bF{\mathbb{F}}
\def\bG{\mathbb{G}}
\def\bH{\mathbb{H}}
\def\bI{\mathbb{I}}
\def\bJ{\mathbb{J}}
\def\bK{\mathbb{K}}
\def\bL{\mathbb{L}}
\def\bM{\mathbb{M}}
\def\bN{\mathbb{N}}
\def\bO{\mathbb{O}}
\def\bP{\mathbb{P}}
\def\bQ{\mathbb{Q}}
\def\bR{\mathbb{R}}
\def\bS{\mathbb{S}}
\def\bT{\mathbb{T}}
\def\bU{\mathbb{U}}
\def\bV{\mathbb{V}}
\def\bW{\mathbb{W}}
\def\bX{\mathbb{X}}
\def\bY{\mathbb{Y}}
\def\bZ{\mathbb{Z}}

%% \mathscr Letters, for filtration, sigma algebras etc
\def\sA{\mathscr{A}}
\def\sB{\mathscr{B}}
\def\sC{\mathscr{C}}
\def\sD{\mathscr{D}}
\def\sE{\mathscr{E}}
\def\sF{\mathscr{F}}
\def\sG{\mathscr{G}}
\def\sH{\mathscr{H}}
\def\sI{\mathscr{I}}
\def\sJ{\mathscr{J}}
\def\sK{\mathscr{K}}
\def\sL{\mathscr{L}}
\def\sM{\mathscr{M}}
\def\sN{\mathscr{N}}
\def\sO{\mathscr{O}}
\def\sP{\mathscr{P}}
\def\sQ{\mathscr{Q}}
\def\sR{\mathscr{R}}
\def\sS{\mathscr{S}}
\def\sT{\mathscr{T}}
\def\sU{\mathscr{U}}
\def\sV{\mathscr{V}}
\def\sW{\mathscr{W}}
\def\sX{\mathscr{X}}
\def\sY{\mathscr{Y}}
\def\sZ{\mathscr{Z}}


%%%% \mathsf for Matrices
\def\mA{\mathsf{A}}
\def\mB{\mathsf{B}}
\def\mC{\mathsf{C}}
\def\mD{\mathsf{D}}
\def\mE{\mathsf{E}}
\def\mF{\mathsf{F}}
\def\mG{\mathsf{G}}
\def\mH{\mathsf{H}}
\def\mI{\mathsf{I}}
\def\mJ{\mathsf{J}}
\def\mK{\mathsf{K}}
\def\mL{\mathsf{L}}
\def\mM{\mathsf{M}}
\def\mN{\mathsf{N}}
\def\mO{\mathsf{O}}
\def\mP{\mathsf{P}}
\def\mQ{\mathsf{Q}}
\def\mR{\mathsf{R}}
\def\mS{\mathsf{S}}
\def\mT{\mathsf{T}}
\def\mU{\mathsf{U}}
\def\mV{\mathsf{V}}
\def\mW{\mathsf{W}}
\def\mX{\mathsf{X}}
\def\mY{\mathsf{Y}}
\def\mZ{\mathsf{Z}}


%%%% \mathbf for Matrices or vectors
\def\bfB{\boldsymbol{B}}
\def\bfC{\boldsymbol{C}}
\def\bfD{\boldsymbol{D}}
\def\bfA{\boldsymbol{A}}
\def\bfE{\boldsymbol{E}}
\def\bfF{\boldsymbol{F}}
\def\bfG{\boldsymbol{G}}
\def\bfH{\boldsymbol{H}}
\def\bfI{\boldsymbol{I}}
\def\bfJ{\boldsymbol{J}}
\def\bfK{\boldsymbol{K}}
\def\bfL{\boldsymbol{L}}
\def\bfM{\boldsymbol{M}}
\def\bfN{\boldsymbol{N}}
\def\bfO{\boldsymbol{O}}
\def\bfP{\boldsymbol{P}}
\def\bfQ{\boldsymbol{Q}}
\def\bfR{\boldsymbol{R}}
\def\bfS{\boldsymbol{S}}
\def\bfT{\boldsymbol{T}}
\def\bfU{\boldsymbol{U}}
\def\bfV{\boldsymbol{V}}
\def\bfW{\boldsymbol{W}}
\def\bfX{\boldsymbol{X}}
\def\bfY{\boldsymbol{Y}}
\def\bfZ{\boldsymbol{Z}}

%%%% \mathbf for Matrices
\def\bfa{\boldsymbol{a}}
\def\bfb{\boldsymbol{b}}
\def\bfc{\boldsymbol{c}}
\def\bfd{\boldsymbol{d}}
\def\bfe{\boldsymbol{e}}
\def\bff{\boldsymbol{f}}
\def\bfg{\boldsymbol{g}}
\def\bfh{\boldsymbol{h}}
\def\bfi{\boldsymbol{i}}
\def\bfj{\boldsymbol{j}}
\def\bfk{\boldsymbol{k}}
\def\bfl{\boldsymbol{l}}
\def\bfm{\boldsymbol{m}}
\def\bfn{\boldsymbol{n}}
\def\bfo{\boldsymbol{o}}
\def\bfp{\boldsymbol{p}}
\def\bfq{\boldsymbol{q}}
\def\bfr{\boldsymbol{r}}
\def\bfs{\boldsymbol{s}}
\def\bft{\boldsymbol{t}}
\def\bfu{\boldsymbol{u}}
\def\bfv{\boldsymbol{v}}
\def\bfw{\boldsymbol{w}}
\def\bfx{\boldsymbol{x}}
\def\bfy{\boldsymbol{y}}
\def\bfz{\boldsymbol{z}}

%%%% \boldsymbol for lower case greek letter in math mode
\def\bfalpha{\boldsymbol{\alpha}}
\def\bfbeta{\boldsymbol{\beta}}
\def\bfgamma{\boldsymbol{\gamma}}
\def\bfdelta{\boldsymbol{\delta}}
\def\bfepsilon{\boldsymbol{\epsilon}}
\def\bfzeta{\boldsymbol{\zeta}}
\def\bfeta{\boldsymbol{\eta}}
\def\bftheta{\boldsymbol{\theta}}
\def\bfiota{\boldsymbol{\iota}}
\def\bfkappa{\boldsymbol{\kappa}}
\def\bflambda{\boldsymbol{\lambda}}
\def\bfmu{\boldsymbol{\mu}}
\def\bfnu{\boldsymbol{\nu}}
\def\bfomicron{\boldsymbol{\omicron}}
\def\bfpi{\boldsymbol{\pi}}
\def\bfrho{\boldsymbol{\rho}}
\def\bfsigma{\boldsymbol{\sigma}}
\def\bftau{\boldsymbol{\tau}}
\def\bfupsilon{\boldsymbol{\upsilon}}
\def\bfphi{\boldsymbol{\phi}}
\def\bfchi{\boldsymbol{\chi}}
\def\bfpsi{\boldsymbol{\psi}}
\def\bfomega{\boldsymbol{\omega}}

%%%% \boldsymbol for upper case greek letter in math mode
\def\bfPhi{\boldsymbol{\Phi}}
\def\bfTheta{\boldsymbol{\Theta}}
\def\bfPsi{\boldsymbol{\Psi}}
\def\bfOmega{\boldsymbol{\Omega}}
\def\bfSigma{\boldsymbol{\Sigma}}


%%%%%%%%%%%%%%%%%% Shortcuts %%%%%%%%%%%%%%%%%%%%%%%%%%%%%%%%%%%%%%%%%%%
\newcommand{\wt}{\widetilde}
\newcommand{\wh}{\widehat}

%%%%%%%%%%%%%%%%%%%%%%%%%%%%%%%%%%%%%%%%%%%%%%%%%%%%%%%%%%%%%%%%%%%%%%%%%%%%%%%%%
%%%%%%%%%%%%%%%%%%   Nonstandard notations  %%%%%%%%%%%%%%%%%%%%%%%%%%%%%%%%%%%%%
\newcommand{\pd}[1]{\partial_{#1}}      % partial derivative
\newcommand{\1}{\mathbbm{1}}            % preferable way of writing indicator function
\newcommand{\set}[1]{\{#1\}}            % set: {xyz} to be used for inline formulas
\newcommand{\Set}[1]{\left\{#1\right\}} % set: {xyz} to be used for separate (not inline) formulas
\renewcommand{\mid}{\,|\,}              % mid bar with small spaces before and after: x | y
\newcommand{\Mid}{\,\Big | \,}          % big bar with small spaces before and after:
\newcommand{\norm}[1]{ \| #1 \| }       % mid bar with small spaces before and after: x | y
\newcommand{\abs}[1]{\left\vert#1\right\vert}   % absolute value
\newcommand{\CB}[1]{\left\{ #1 \right\}}
\newcommand{\SB}[1]{\left[ #1 \right]}
\newcommand{\Pare}[1]{\left( #1 \right)}
\newcommand{\AB}[1]{\left \langle #1 \right \rangle}
\newcommand{\given}[1]{\left.#1\right|}
\newcommand{\givenAlt}[1]{\left.#1\right]}
\newcommand{\Tran}[1]{{#1}^\top}
\newcommand{\bsde}{BS$\Delta$E}         % BS\DeltaE
\newcommand{\bsdes}{BS$\Delta$Es}       % BS\DeltaE
\newcommand{\ow}{\text{otherwise}}
\newcommand{\imblies}{\Longleftarrow}
\newcommand{\Exp}[1]{\mathrm{E}\left[ #1 \right]}
\newcommand{\Var}[1]{\mathrm{Var}\left[ #1 \right]}
\newcommand{\Cov}[1]{\mathrm{Cov}\left[ #1 \right]}
\newcommand{\Bspace}{\;\;\;\;}
\newcommand{\dif}{\,\mathrm{d}}        % used for differential, same as in commath.sty

\newcommand{\LHS}{\text{LHS}} % left hand side
\newcommand{\RHS}{\text{RHS}} % right hand side
\newcommand{\Adj}{\text{Adj}} %adjoint matrix

\newcommand{\RNum}[1]{\uppercase\expandafter{\romannumeral #1\relax}} % romanian numerals

\DeclareMathOperator{\logit}{logit}
\DeclareMathOperator*{\esssup}{ess\,sup} % ess sup
\DeclareMathOperator*{\essinf}{ess\,inf} % ess inf
\DeclareMathOperator*{\esslimsup}{ess\,\limsup}
\DeclareMathOperator*{\essliminf}{ess\,\liminf}
\DeclareMathOperator*{\argmin}{arg\,min} % argmin
\DeclareMathOperator*{\diag}{diag} % diag
\DeclareMathOperator*{\argmax}{arg\,max} % argmax
\DeclareMathOperator*{\Arg}{Arg} % arguments
\DeclareMathOperator*{\rank}{rank\,} % argmax
\DeclareMathOperator*{\KL}{KL} % KL divergence
\DeclareMathOperator*{\Proj}{Proj} % Projection
\DeclareMathOperator{\Std}{\mathrm{Std}} % \std for Standard deviation
\DeclareMathOperator{\sgn}{\mathrm{sgn}} % sign of a variable
\DeclareMathOperator{\tr}{\mathrm{trace}} % matrix trace
% trigonometric and hyperbolic functions
\DeclareMathOperator{\sech}{sech}
\DeclareMathOperator{\csch}{csch}
\DeclareMathOperator{\arcsec}{arcsec}
\DeclareMathOperator{\arccot}{arcCot}
\DeclareMathOperator{\arccsc}{arcCsc}
\DeclareMathOperator{\arccosh}{arcCosh}
\DeclareMathOperator{\arcsinh}{arcsinh}
\DeclareMathOperator{\arctanh}{arctanh}
\DeclareMathOperator{\arcsech}{arcsech}
\DeclareMathOperator{\arccsch}{arcCsch}
\DeclareMathOperator{\arccoth}{arcCoth} 

%%%%%%%%%%%%%%%%%%%%%%%%%%%%%%% FINANCE %%%%%%%%%%%%%%%%%%%%%%%%%%%%%%%%%%%%%%%%%%%%%%%%%%
\DeclareMathOperator{\var}{\mathrm{V}@\mathrm{R}}           % \V@R Value-at-risk
\DeclareMathOperator{\tce}{\mathrm{TCE}}                    % Tail Conditional Expectation
\DeclareMathOperator{\tvar}{\mathrm{TV}@\mathrm{R}}         % \TV@R tail Value-at-risk
\DeclareMathOperator{\avar}{\mathrm{AV}@\mathrm{R}}         % \AV@R average Value-at-risk
\DeclareMathOperator{\ent}{\mathrm{Ent}}                    % \ent = Entropic Risk Measure
\DeclareMathOperator{\glr}{\mathrm{GLR}}                    % \glr = gain to loss ratio

\DeclareMathOperator{\ES}{\mathrm{ES}}

\newcommand{\ask}{\text{ask}}           % ask price
\newcommand{\bid}{\text{bid}}           % bid price

%%%%%%%%%%%%%%%%%%%%%%%%%%%%%%%%%%%%%%%%%%%%%%%%%%%%%%%
% \bibliographystyle{amsplain}
% \bibliographystyle{alpha} % standard LaTeX bibliography format. Preferable to be used
% \bibliography{MathFinanceMaster-12-28-2014}
%\bibliography{D:/_research/latex/lib_igor/igor_bib_mathfinance}
% help on how to use several bit files \bibliography{videogames,comics,interface,theory}
%%%%%%%%%%%%%%%%%%%%%%%%%%%%%%%%%%%%%%%%%%%%%%%%%%%%%%%




\title{HW0}
\author{Yuanxing Cheng, A20453410, CS577-f22}

\begin{document}

\maketitle

\section*{A}

let \(a = \SB{1,2,3}^\top\), \(b = \SB{4,5,6}^\top\), \(c = \SB{-1,1,3}^\top\).
\subsection*{1}

\begin{myleftlinebox}
    Find \(2a-b\)
    \tcblower
    \[
        2a-b = \SB{2\cdot 1-4,2\cdot 2-5,2\cdot 3-3 }^\top = \SB{-2,-1,0}^\top
    \]
\end{myleftlinebox}


\subsection*{2}
\begin{myleftlinebox}
    Find \(\hat a\) the unit vector along \(a\)
    \tcblower
    \[
        \hat a = \frac{a}{\abs{a}} = \frac{\SB{1,2,3}^\top}{\sqrt{1^2+2^2+3^2}}=\SB{\frac{1}{\sqrt{14}},\frac{2}{\sqrt{14}},\frac{3}{\sqrt{14}}}
    \]
\end{myleftlinebox}


\subsection*{3}
\begin{myleftlinebox}
    Find \(\norm a\) and teh angle of \(a\) relative to positive \(x\) axis
    \tcblower
    \[
        \norm{a} = \norm{a}_2 = \sqrt{1^2+2^2+3^2} = \sqrt{14}
    \]
    \[
        \theta = \arccos\Pare{\frac{a\cdot i}{\abs{a}\abs{i}}} = \arccos\Pare{\frac{1\cdot 1+2\cdot 0+3\cdot 0}{\sqrt{14}\cdot 1}} = \arccos\frac{1}{\sqrt{14}}
    \]
\end{myleftlinebox}



\subsection*{4}
\begin{myleftlinebox}
    Find direction cosines of \(a\)
    \tcblower
    \begin{align*}
        \theta_1 &= \theta = \arccos\Pare{1/\sqrt{14}}\\
        \theta_2 &= \arccos\Pare{\frac{a\cdot j}{\abs{a}\abs{i}}} = \arccos\Pare{\frac{1\cdot 0+2\cdot 1+3\cdot 0}{\sqrt{14}\cdot 1}} = \arccos\Pare{\frac{2}{\sqrt{14}}}\\
        \theta_3 &= \arccos\Pare{\frac{a\cdot k}{\abs{a}\abs{i}}} = \arccos\Pare{\frac{1\cdot 0+2\cdot 0+3\cdot 1}{\sqrt{14}\cdot 1}} = \arccos\Pare{\frac{3}{\sqrt{14}}}
    \end{align*}
\end{myleftlinebox}


\subsection*{5}
\begin{myleftlinebox}
    Find angle between \(a\) and \(b\)
    \tcblower
    \[
        \theta_{a,b} = \arccos\Pare{\frac{a\cdot b}{\abs{a}\cdot\abs{b}}}=\arccos\Pare{\frac{1\cdot 4+2\cdot 5+3\cdot 6}{\sqrt{14}\cdot \sqrt{4^2+5^2+6^2}}} = \Pare{\frac{32}{7\sqrt{22}}}
    \]
\end{myleftlinebox}


\subsection*{6}
\begin{myleftlinebox}
    Find \(a\cdot b\) and \(b\cdot a\)
    \tcblower
    \[
        a\cdot b = b\cdot a = 1\cdot 4+2\cdot 5+3\cdot 6 = 32
    \]
\end{myleftlinebox}


\subsection*{7}
\begin{myleftlinebox}
    Find \(a\cdot b\) by using the angle between \(a\) and \(b\)
    \tcblower
    \[
        a \cdot b = \abs{a}\abs{b}\cos \theta_{a,b} = 7\sqrt{22} \cdot \frac{32}{7\sqrt{22}} = 32
    \]
\end{myleftlinebox}


\subsection*{8}
\begin{myleftlinebox}
    The scalar projection of \(b\) onto \(\hat a\)
    \tcblower
    \[
        \abs{\Proj\Pare{b;\hat a}} = b\cdot \hat a = \frac{32}{\sqrt{14}}
    \]
    checking,
    \[
        \Proj\Pare{b;\hat a} = \frac{a a^\top}{a^\top a}b = \begin{bmatrix}
            1 & 2 & 3\\
            2 & 4 & 6\\
            3 & 6 & 9
        \end{bmatrix}b/\abs{a}^2 = \SB{32,64,96}^\top/14 = \SB{16/7,32/7,48/7}^\top
    \]
\end{myleftlinebox}



\subsection*{9}
\begin{myleftlinebox}
    Find a vector which is perpendicular to both \(a\)
    \tcblower
    consider \(x=[0,-3,x_3]^\top\) and then since \(x\) is perpendicular to \(a\), we have

    \[
        0 = x\cdot a = -3\cdot 2 + 3x_3\implies x_3 = 2
    \]

    so \(x =[0,-3,2]^\top\perp a\)
\end{myleftlinebox}


\subsection*{10}
\begin{myleftlinebox}
    Find \(a\times b\) and \(b\times a\)
    \tcblower
    using matrix notation

    \[
        a\times b = \begin{bmatrix}
            i & j & k\\
            1 & 2 & 3\\
            4 & 5 & 6
        \end{bmatrix} = -3i-(-6)j + (-3)k =[-3,6,-3]^\top
    \]  

    \[
    b\times a = - a\times b = [3,-6,3]^\top
    \]  
\end{myleftlinebox}



\subsection*{11}
\begin{myleftlinebox}
    Find a vector which is perpendicular to both \(a \) and \(b\)
    \tcblower
    \(a\times b\) is perpendicular to both vectors
\end{myleftlinebox}



\subsection*{12}
\begin{myleftlinebox}
    Find the linear dependency between these vecs
    \tcblower
    notice \(b-3a+c = 0\), thus they are linearly dependent.

\end{myleftlinebox}


\subsection*{13}
\begin{myleftlinebox}
    Find \(a^\top b\) and \(ab^\top\)
    \tcblower
    \[
        a^\top b = a\cdot b = 32, ab^\top = \begin{bmatrix}
            4 & 5 & 6\\
            8 & 10 & 12\\
            12 & 15 & 18
        \end{bmatrix}
    \]

\end{myleftlinebox}



\section*{B}

\(
    A = \begin{bmatrix}
        1 & 2 & 3\\
        4 & -2 & 3\\
        0 & -5 & 1
    \end{bmatrix}
\), \(B=\begin{bmatrix}
    1 & 2 & 1\\
    2 & 1 & -4\\
    3 & -2 & 1
\end{bmatrix}\), \(C = \begin{bmatrix}
    1 & 2 & 3\\
    4 & 5 & 6\\
    -1 & 1 & 3
\end{bmatrix}\), \(d = [1,2,3]^\top\)

\subsection*{1}
\begin{myleftlinebox}
    Find \(2A-B\)
    \tcblower
    using elementwise operation
    \[
        2A-B = \begin{bmatrix}
            1 & 2 & 5\\
            6 & -5 & 10\\
            -3 & 12 & -3
        \end{bmatrix}
    \]
\end{myleftlinebox}


\subsection*{2}
\begin{myleftlinebox}
    Find \(AB\) and \(BA\)
    \tcblower
    \[
        AB = \begin{bmatrix}
            14 & -2 & -4\\
            9 & 0 & 15\\
            7 & 7 & -21
        \end{bmatrix}, BA =  \begin{bmatrix}
            9 & 3 & 8\\
            6 & -18 & 13\\
            -5 & 15 & 2
        \end{bmatrix}
    \]
\end{myleftlinebox}


\subsection*{3}
\begin{myleftlinebox}
    Find \(\Pare{AB}^\top\) and \(B^\top A^\top\)
    \tcblower
    \[
        \Pare{AB}^\top = \begin{bmatrix}
            14 & 9 & 7\\
            -2 & 0 & 7\\
            -4 & 15 & -21
        \end{bmatrix}, B^\top A^\top = \Pare{AB}^\top
    \]
\end{myleftlinebox}


\subsection*{4}
\begin{myleftlinebox}
    Find \(\abs{A}\) and \(\abs{C}\)
    \tcblower
    \[
        a\cdot b = b\cdot a = 1\cdot 4+2\cdot 5+3\cdot 6 = 32
    \]
\end{myleftlinebox}
Notice \(C\) is from last problem whose row vectors are linearly dependent, thus has \(0\) determinant.
\[
    \abs{A}=2+0+60-0-(-8)-15 = 55, \abs{C}=0
\]

\subsection*{5}
\begin{myleftlinebox}
    Find whose matrix row vecs form an orthogonal set
    \tcblower
    \[
        a\cdot b = b\cdot a = 1\cdot 4+2\cdot 5+3\cdot 6 = 32
    \]
\end{myleftlinebox}

for \(A\), \(a^{row}_2\cdot a^{row}_3 = -13\) thus \(A\) is not; \(C\) not full rank, thus not. Now check \(B\), 
\begin{align*}
    b^{row}_2\cdot b^{row}_3 & = 6-2-4=0\\
    b^{row}_1\cdot b^{row}_3 & =3-4+1=0\\
    b^{row}_2\cdot b^{row}_1 & =2+2-4=0
\end{align*}

\subsection*{6}
\begin{myleftlinebox}
    Find \(A^{-1}\) and \(B^{-1}\)
    \tcblower
    Using adjoint  matrices.

    \[
        A^{-1} =\frac{1}{55}\begin{bmatrix}
            2-15 & -(-2)+15 & 6-(-6)\\
            -(-4)+0 & -1-0 & -3+12\\
            20-0 & -5+0 & -2-8
        \end{bmatrix} = \begin{bmatrix}
            -13/55 & 17/55 & 12/55\\
            4/55 & -1/55 & 9/55\\
            4/11 & -1/11 & -2/11
        \end{bmatrix},
    \]

    \[
        B^{-1}	 = \frac{1}{-42}\begin{bmatrix}
            1-8 & -2+(-2) & -8-1\\
            -2+(-12) & 1-3 & -(-4)+2\\
            -4-3 & -(-2)+6 & 1-4
        \end{bmatrix} = \begin{bmatrix}
            1/6 & 2/21 & 3/14\\
            1/3 & 1/21 & -1/7\\
            1/6 & -4/21 & 1/14
        \end{bmatrix}
    \]
\end{myleftlinebox}

\subsection*{7}
\begin{myleftlinebox}
    Find \(C^{-1}\)
    \tcblower
    As row's are linearly dependent, inverse didn't exist. 

\end{myleftlinebox}

\subsection*{8}
\begin{myleftlinebox}
    Find \(Ad\)
    \tcblower
    \[
        Ad = [14,9,7]^\top
    \]
\end{myleftlinebox}


\subsection*{9}
\begin{myleftlinebox}
    Find the scalar projection of the rows of \(A\) onto the vector \(d\) with noralizing \(d\)
    \tcblower
    \[
        A\hat d= [\sqrt{14},9/\sqrt{14},7/\sqrt{14}]^\top
    \]
\end{myleftlinebox}


\subsection*{10}
\begin{myleftlinebox}
    Find the vector projection of the rows of \(A\) onto the vector \(d\) with noralizing \(d\)
    \tcblower
    \[
        \Proj\Pare{A;\hat d}^\top = \Pare{\frac{d d^\top}{d^\top d}A^\top}^\top =\frac{1}{d^\top d}A d d^\top= A \begin{bmatrix}
            1 & 2 & 3\\
            2 & 4 & 6\\
            3 & 6 & 9
        \end{bmatrix}/\abs{d}^2 =  \begin{bmatrix}
            \sqrt{14} & 2\sqrt{14} & 3\sqrt{14}\\
            9/\sqrt{14} & 18/\sqrt{14} & 27/\sqrt{14} \\
            7/\sqrt{14} & \sqrt{14} & 21/\sqrt{14}
        \end{bmatrix}
    \]
\end{myleftlinebox}



\subsection*{11}
\begin{myleftlinebox}
    Find linear combination of the columns of \(A\) using the elements of \(d\)
    \tcblower
    \[
        Ad = [14,9,7]^\top
    \]
\end{myleftlinebox}



\subsection*{12}
\begin{myleftlinebox}
    Find \(x\) for \(Bx=d\)
    \tcblower
    Notice \(B=[d,e,f]\) so if \(x = [x_1,x_2,x_3]^\top\),

    \[
        x_1 d+x_2 e+x_3 f = d \implies x=[1,0,0]^\top
    \]

    Since \(B\) nonsingular, so that's the only solution.

\end{myleftlinebox}


\subsection*{13}
\begin{myleftlinebox}
    Find \(x\) for \(Cx=d\) and reason
    \tcblower
    Using Gaussian elimination we end up with

    \[
        \begin{bmatrix}
            1 & 2 & 3\\
            0 & -3 & -6\\
            0 & 3 & 6
        \end{bmatrix} [x_1,x_2,x_3]^\top = [1, -2,4]^\top
    \]
    Seeing last rows we know there's no such \(x\) satisfies the equation.

\end{myleftlinebox}

\section*{C}
\(
    D = \begin{bmatrix}
        1 & 2\\
        3 & 2
    \end{bmatrix}
\), \(E =\begin{bmatrix}
    2 & -2 \\
    -2 & 5
\end{bmatrix}\), \(F = \begin{bmatrix}
    1 & 2\\
    2 & 4
\end{bmatrix}\)
\subsection*{1}
\begin{myleftlinebox}
    Find eigenvalues and eigenvectors of \(D\)
    \tcblower
    \[
        \det\Pare{\lambda I-D } = (\lambda-1)(\lambda -2)-6 = (\lambda-4)(\lambda+1)
    \]
    so the eigenvalues are \(-1\) and \(4\), then solve eigenvectors,
    \[
        \Pare{- I-D }x = 0 \implies x = [1,-1]^\top
    \]
    \[
        \Pare{4 I-D }x = 0 \implies x = [2,3]^\top
    \]
\end{myleftlinebox}


\subsection*{2}
\begin{myleftlinebox}
    Find the dot product between the eigenvalues of \(D\)
    \tcblower
    \[
        1\cdot 2-1\cdot 3 = -1
    \]
\end{myleftlinebox}

\subsection*{3}
\begin{myleftlinebox}
    Find the dot product between the eigenvalues of \(E\)
    \tcblower
    \[
        \det\Pare{\lambda I-E } = (\lambda-2)(\lambda -5)-4 = (\lambda-6)(\lambda-1)=0\implies \lambda_1 = 1,\lambda_2=6
    \]
    \[
        \Pare{I-E }x = 0 \implies x = [2,1]^\top
    \]
    \[
        \Pare{6I-E }x = 0 \implies x = [-1,2]^\top
    \]
    \[
        -2+2=0
    \]
\end{myleftlinebox}


\subsection*{4}
\begin{myleftlinebox}
    Find the property of the eigenvalues of \(E\) and reason
    \tcblower
    the eigenvectors of \(E\) are orthogonal due to the fact that \(E\) is symmetric. Reason

    Let \(Ex = \lambda_1 x\) and \(Ey = \lambda_2 y\) where \(\lambda_1\neq \lambda_2\) we have

    \begin{equation}\label{eqn:symM_eig_Ortho}
        \begin{split}
            \lambda_1 x\cdot y & = Ex \cdot y\\
            &= (Ex)^\top y = x^\top Ey \\
            & = x\cdot (Ey) = x\cdot \lambda_2 y\\
            \implies (\lambda_1 - \lambda_2)x\cdot y &= 0
        \end{split}
    \end{equation}
    Since we've assumed \(\lambda_1\neq \lambda_2\), \(x\cdot y=0\) thus orthogonal.

\end{myleftlinebox}

\subsection*{5}
\begin{myleftlinebox}
    Find trivial \(x\) for \(Fx=0\)
    \tcblower
    Trivial solution for \(Fx=0\) is just zero vector \([0,0]^\top\).
\end{myleftlinebox}


\subsection*{6}
\begin{myleftlinebox}
    Find non-trivial \(x\) for \(Fx=0\)
    \tcblower
    non trivial solution, we can have \(x=[2,-1]^\top\) or  \(x=[4,-2]^\top\)
\end{myleftlinebox}

\section*{7}
\begin{myleftlinebox}
    the only solution \(x\) to \(Dx=0\) and reason why single solution
    \tcblower
    it only has trivial solution \([0,0]^\top\). Reason is by gaussian elimination or null space dimention equals to \(n-rank(D)=2-2=0\) thus null space contains only zero vector.
\end{myleftlinebox}

\section*{D}
\(f(x) = x^2+3\), \(g(x)=x^2\), \(q(x,y) = x^2+y^2\)
\subsection*{1}

\begin{myleftlinebox}
    first and second derivative of \(f\) wrt \(x\)
    \tcblower
    \[
        f'(x) = 2x, f''(x) = 2  
    \]
\end{myleftlinebox}

\subsection*{2}
\begin{myleftlinebox}
    partial derivatives of \(q\)
    \tcblower
    \[
        \frac{\partial q}{\partial x} =2x, \frac{\partial q}{\partial y} =2y
    \]
\end{myleftlinebox}

\subsection*{3}
\begin{myleftlinebox}
    gradient of \(q\)
    \tcblower
    \[
        \nabla q = [\frac{\partial q}{\partial x},\frac{\partial q}{\partial y}]^\top=[2x,2y]^\top
    \]
\end{myleftlinebox}

\subsection*{4}
\begin{myleftlinebox}
    the derivative of \(f\circ g\) wrt \(x\), with and without using chain rule
    \tcblower
    \[
        f\circ g(x) = x^4+3\implies \Pare{f\circ g}'(x) = 4x^3\\
        \Pare{f\circ g}'(x) = \given{\frac{\partial f(y)}{\partial y}}_{y=g(x)} \frac{\partial g(x)}{\partial x} = 2(x^2)*2x=4x^3
    \]
\end{myleftlinebox}

\subsection*{E}\
Doing A, B, C with python, see following notebook pdf.





\newpage

%\bibliographystyle{acm}
%\bibliography{D:/Papers.bib}

\newpage
\appendix
\section{Appendix}

\end{document}