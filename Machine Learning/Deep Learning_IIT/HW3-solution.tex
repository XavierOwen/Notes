\documentclass{article}

%%%%%%%%%%%%%%%%% this version: 03-14-2017 %%%%%%%%%%%%%%%%%%%%%%%%%%%%%%%

%%% standard packages
\usepackage{amsthm, amsmath, amssymb, amsfonts, graphicx, epsfig}
%\graphicspath{ {D:/Notes/others/assets/} }

\usepackage{algorithm, algpseudocode}

\allowdisplaybreaks
\usepackage{setspace}
%\usepackage{accents}
%%%
%\usepackage{kbordermatrix}

%%%%%%%%%%%%%%%%%%%%%%% ADDITIONAL FONTS %%%%%%%%%%%%%%%%%%%%%%%%%%%%%%
%%% Package to make \mathbbm, in particular to have 1 as for indicator function
\usepackage{bbm}
%%% Package to make special curl fonts, by using \mathscr{F}
\usepackage{mathrsfs}
%%% dsfont sometimes used instead of mathbb or mathbbm
%\usepackage{dsfont}
%%%%%%%%%%%%%%%%%%%%%%%%%%%%%%%%%%%%%%%%%%%%%%%%%%%%%%%%%%%%%%%%%%%%%%%


%%%%%%%%%%%%%%%%%%%%%%%%%%%%%%%%%%%%%%%%%%%%%%%%%%%%%%%%%%%%%%%%%%%%%%
%%% The ulem package provides various types of underlining that can
%%% stretch between words and be broken across lines. Convenient for editing. \sout{xxxx}
%%% http://ctan.unixbrain.com/macros/latex/contrib/ulem/ulem.pdf
%%%% Remark: if used with natbib, then the bibliography comes underlined. Comment this package at last compilation
%\usepackage{ulem}
%%%%%%%%%%%%%%%%%%%%%%%%%%%%%%%%%%%%%%%%%%%%%%%%%%%%%%%%%%%%%%%%%%%%%%

%%%%%%%%%%%%%%%%%%%%%%%%%%%%%%%%%%%%%%%%%%%%%%%%%%%%%%%%%%%%%%%%%%%%%%
%% Cancel is used to cross out in math mode. \xcancel, \cancel Convenient for editting.
%% http://ctan.math.utah.edu/ctan/tex-archive/macros/latex/contrib/cancel/cancel.pdf
%\usepackage{cancel}
%%%%%%%%%%%%%%%%%%%%%%%%%%%%%%%%%%%%%%%%%%%%%%%%%%%%%%%%%%%%%%%%%%%%%%


%%%%%%%%%%%%%%%%%%%%%%%%%%%%%%%%%%%%%%%%%%%%%%%%%%%%%%%%%%%%%%%%%%%%%%%
%This packages adds support of handling eps images to package graphics
%or graphicx with option pdftex. If an eps image is detected, epstopdf is
%automatically called to convert it to pdf format.
\usepackage{epstopdf}
%%%%%%%%%%%%%%%%%%%%%%%%%%%%%%%%%%%%%%%%%%%%%%%%%%%%%%%%%%%%%%%%%%%%%%%


%%%%%%%%%%%%%%%%%%%%%%%%%%%%%%%%%%%%%%%%%%%%%%%%%%%%%%%%%%%%%%%%%%%%%%%
% various features for using graphics, including subfigure, captions, subcaptions etc.
\usepackage{graphicx}
%\usepackage{caption}
\usepackage[font=sl,labelfont=bf]{caption}
\usepackage{subcaption}
%%%%%%%%%%%%%%%%%%%%%%%%%%%%%%%%%%%%%%%%%%%%%%%%%%%%%%%%%%%%%%%%%%%%%%%


%%%%%%%%%%%%%%%%%%%%%%%%%%%%%%%%%%%%%%%%%%%%%%%%%%%%%%%%%%%%%%%%%%%%%%%
% This package improves the interface for defining floating objects such
% as figures and tables in LaTeX.
% http://www.ctan.org/pkg/float
\usepackage{float}
\restylefloat{table}
%%%%%%%%%%%%%%%%%%%%%%%%%%%%%%%%%%%%%%%%%%%%%%%%%%%%%%%%%%%%%%%%%%%%%%%


%%% use for diagonals in the table's cells
%\usepackage{slashbox}  %Removed by Ares

%%%%%%%%%%%%%%%%%%%%%%%%%%%%%%%%%%%%%%%%%%%%%%%%%%%%%%%%%%%%%%%%%%%%%%%


%%%%%%%%%%%%%%%%%%%%%%%%%%%%%%%%%%%%%%%%%%%%%%%%%%%%%%%%%%%%%%%%%%%%%%%
% This package gives the enumerate environment an optional argument
% which determines the style in which the counter is printed.
% http://www.ctex.org/documents/packages/table/enumerate.pdf
\usepackage{enumerate}
%%%%%%%%%%%%%%%%%%%%%%%%%%%%%%%%%%%%%%%%%%%%%%%%%%%%%%%%%%%%%%%%%%%%%%%


%%%%%%%%%%%%%%%%%%%%%%%%%%%%%%%%%%%%%%%%%%%%%%%%%%%%%%%%%%%%%%%%%%%%%%%
%selectively in/exclude pieces of text: the user can determine new comment versions,
%and each is controlled separately. Special comments can be determined where the
%user specifies the action that is to be taken with each comment line.
% http://get-software.net/macros/latex/contrib/comment/comment.pdf
%\usepackage{comment}
%%%%%%%%%%%%%%%%%%%%%%%%%%%%%%%%%%%%%%%%%%%%%%%%%%%%%%%%%%%%%%%%%%%%%%%


%%%%%%%%%%%%%%%%%%%%%%%%%%%%%%%%%%%%%%%%%%%%%%%%%%%%%%%%%%%%%%%%%%%%%%
%% To display the labels used in a tex file in the dvi file (for example, if a theorem is labelled with the \label command) use the package
%% http://www.ctan.org/pkg/showkeys

%%%%%% Use Option 1
%%%%%% This will display all labels (for example, in the case of a labelled theorem, the label of the theorem will occur in the margin of the dvi file.
%%%%%% The command above by itself not only displays labels when they are first named but also when they are cited (or referenced).
%\usepackage{showkeys}

%%%%%% Use Option 2
%%%%%% It will NOT display citations and references. Only Labels
%\usepackage[notref, notcite]{showkeys}

%%%%%% Use Options 3
%%%%%% formats to smaller fonts the labels etc.
%\usepackage[usenames,dvipsnames]{color}
%\providecommand*\showkeyslabelformat[1]{{\normalfont \tiny#1}}
%\usepackage[notref,notcite,color]{showkeys}
%\definecolor{labelkey}{rgb}{0,0,1}
%%%%%%%%%%%%%%%%%%%%%%%%%%%%%%%%%%%%%%%%%%%%%%%%%%%%%%%%%%%%%%%%%%%%%%


%%%%%%%%%%%%%%%%%%%%%%%%%%%%%%%%%%%%%%%%%%%%%%%%%%%%%%%%%%%%%%%%%%%%%%%
% It extends the functionality of all the LATEX cross-referencing commands (including the table of contents, bibliographies etc) to produce \special commands which a driver can turn into hypertext links; it also provides new commands to allow the user to write ad hoc hypertext links, including those to external documents and URLs.
% http://www.tug.org/applications/hyperref/manual.html
\usepackage[colorlinks=true, pdfstartview=FitV, linkcolor=blue,
            citecolor=blue, urlcolor=blue]{hyperref}
\usepackage[usenames]{color}
%%%%%%%%%%%%%%%%%%%%%%%%%%%%%%%%%%%%%%%%%%%%%%%%%%%%%%%%%%%%%%%%%%%%%%%%
% custom textbox
\usepackage{tcolorbox}
\tcbuselibrary{breakable}
\tcbuselibrary{skins}
\tcbset{my left line/.style={
          enhanced, frame hidden, borderline west = {0.5pt}{0pt}{black}, % specify border
          opacityframe=0, opacityback=0,opacityfill=0, % color
          arc = 0mm, left skip=1em, % border
          left = 1mm,top=0mm,bottom=0mm,boxsep=1mm,middle=1mm, % text and border
}}

\newtcbox{\mybox}[1][]{my left line, #1}
\newtcolorbox{myleftlinebox}[1][breakable]{my left line, #1}
% usage
% use \mybox[on line]{your TEXT} for inline box otherwise forcing line breaks   
% for whole break, use \begin{myleftlinebox
%%%%%%%%%%%%%%%%%%%%%%%%%%%%%%%%%%%%%%%%%%%%%%%%%%%%%%%%%%%%%%%%%%%%%%%
\usepackage[utf8]{inputenc}


%%%%%%%%%%%%%%%%%%%%%%%%%%%%%%%%%%%%%%%%%%%%%%%%%%%%%%%%%%%%%%%%%%%%%%%
% a good looking way to format urls
% http://mirror.its.uidaho.edu/pub/tex-archive/help/Catalogue/entries/url.html
\usepackage{url}
% Define a new 'leo' style for the package that will use a smaller font.
\makeatletter\def\url@leostyle{%
 \@ifundefined{selectfont}{\def\UrlFont{\sf}}{\def\UrlFont{\scriptsize\ttfamily}}} \makeatother\urlstyle{leo}
%%%%%%%%%%%%%%%%%%%%%%%%%%%%%%%%%%%%%%%%%%%%%%%%%%%%%%%%%%%%%%%%%%%%%%%


%%%%%%%%%%%%%%%%%%%%%%%%%%%%%%%%%%%%%%%%%%%%%%%%%%%%%%%%%%%%%%%%%%%%%%%%%%%%
%%%%%% The present package defines the environment mdframed which automatically deals with page breaks in framed text.
%%%%%% http://mirrors.ibiblio.org/CTAN/macros/latex/contrib/mdframed/mdframed.pdf
\usepackage[framemethod=default]{mdframed}
%%%%%%%%%%%%%%%%%%%%%%%%%%%%%%%%%%%%%%%%%%%%%%%%%%%%%%%%%%%%%%%%%%%%%%%%%%%%

%%%%%%%%%%%%%%%%%%%%%%%%%%%%%%%%%%%%%%%%%%%%%%%%%%%%%%%%%%%%%%%%%%%%%%%%%%%%%%%%
%%%%%% An extended implementation of the array and tabular environments
%%%%%% which extends the options for column formats, and provides "programmable" format specifications.
%\usepackage{array}
%%%%%%%%%%%%%%%%%%%%%%%%%%%%%%%%%%%%%%%%%%%%%%%%%%%%%%%%%%%%%%%%%%%%%%%%%%%

%%%%%%%%%%%%%%%%%%%%%%%%%%%%%%%%%%%%%%%%%%%%%%%%%%%%%%%%%%%%%%%%%%%%%%%%%%%%
%%% The package enhances the quality of tables in LATEX, providing extra
%%% commands as well as behind-the-scenes optimization. Guidelines
%%% are given as to what constitutes a good table in this context.
%%% From version 1.61, the package offers long table compatibility.
%\usepackage{booktabs}
%%%%%%%%%%%%%%%%%%%%%%%%%%%%%%%%%%%%%%%%%%%%%%%%%%%%%%%%%%%%%%%%%%%%%%%%%%%%%%


\usepackage{tikz}




%% or use GEOMETRY package
\usepackage[margin=1.0in, letterpaper]{geometry}
%\def\baselinestretch{1.1}


%%%%%%%%%%%%% OR USE EXACT DIMENSIONS %
%\setlength{\textwidth}{6.5in}     %%
%\setlength{\oddsidemargin}{0in}   %%
%\setlength{\evensidemargin}{0in}  %%
%\setlength{\textheight}{8.5in}    %%
%\setlength{\topmargin}{0in}       %%
%\setlength{\headheight}{0in}      %%
%\setlength{\headsep}{.3in}         %%
%\setlength{\footskip}{.5in}       %%
%%%%%%%%%%%%%%%%%%%%%%%%%%%%%%%%%%%%%%%%%%%%%%%%%%%%%%%%%%%%%%%%%%%%%%%


%%%%%%%%%%%%%%%%%%%%%%%%%%%%%%%%%%%%%%%%%%%%%%%%%%%%%%%%%%%%%%%%%%%%%%%
%%%%%%%%%%%%%%%%%%%%%%%% NUMBERING %%%%%%%%%%%%%
\newtheorem{theorem}{Theorem}
\newtheorem{conjecture}{Conjecture}
\newtheorem{conclusion}{Conclusion}
\newtheorem{proposition}[theorem]{Proposition}
\newtheorem{lemma}[theorem]{Lemma}
\newtheorem{corollary}[theorem]{Corollary}
\newtheorem{assumption}{Assumption}
\newtheorem{condition}{C}
\theoremstyle{definition}
\newtheorem{definition}[theorem]{Definition}
\newtheorem{example}[theorem]{Example}
\theoremstyle{remark}
\newtheorem{remark}[theorem]{Remark}
\newtheorem{question}[theorem]{Question}
\newtheorem{problem}[theorem]{Problem}
\newtheorem{NB}[theorem]{Nota Bene}

\numberwithin{equation}{section}
\numberwithin{theorem}{section}
%\renewcommand{\labelitemi}{ {\small $\rhd$}}
%%%%%%%%%%%%%%%%%%%%%%%%%%%%%%%%%%%%%%%%%%%%%%%%%%%%%%%%%%%%%%%%%%%%%%%

%%%%%%%%%%%%%%%%%%%%%%%%%%%%%%%%%%%%%%%%%%%%%%%%%%%%%%%%%%%%%%%%%%%%%%%
\definecolor{Red}{rgb}{0.9,0,0.0}
\definecolor{Blue}{rgb}{0,0.0,1.0}
%%%%%%%%%%%%%%%%%%%%%%%%%%%%%%%%%%%%%
%%% used for editing and making comments in color
% Example \ig{Remarks and Commets}
\newcommand{\ig}[1]{\textcolor[rgb]{0.00, 0.0, 1.0}{{\tiny \textsuperscript{[\textrm{IC:Rem}]}} \ #1}}
\newcommand{\igAdd}[1]{\textcolor[rgb]{0.7, 0.0, 0.0}{{\tiny \textsuperscript{[\textrm{IC:Add}]}} \ #1}}
\newcommand{\igEdit}[1]{\textcolor[rgb]{0.7, 0.0, 0.0}{{\tiny \textsuperscript{[\textrm{IC:Edit}]}} \  #1}}

\newcommand{\trb}[1]{\begin{color}[rgb]{0.98, 0.0, 0.98}{TRB: #1} \end{color}}
\newcommand{\ti}[1]{\begin{color}[rgb]{1.00,0.00,0.00}{TI: #1} \end{color}}
\newcommand{\Red}[1]{\textcolor{Red}{#1}}
%%%%%%%%%%%%%%%%%%%%%%%%%%%%%%%%%%%%%

%%%%%%%%%%%%%%%%%%%%%%%%%%%%%%%%%%%%%
%%%     Igor's macros
%% \mathcal Letters
\def\cA{\mathcal{A}}
\def\cB{\mathcal{B}}
\def\cC{\mathcal{C}}
\def\cD{\mathcal{D}}
\def\cE{\mathcal{E}}
\def\cF{\mathcal{F}}
\def\cG{\mathcal{G}}
\def\cH{\mathcal{H}}
\def\cI{\mathcal{I}}
\def\cJ{\mathcal{J}}
\def\cK{\mathcal{K}}
\def\cL{\mathcal{L}}
\def\cM{\mathcal{M}}
\def\cN{\mathcal{N}}
\def\cO{\mathcal{O}}
\def\cP{\mathcal{P}}
\def\cQ{\mathcal{Q}}
\def\cR{\mathcal{R}}
\def\cS{\mathcal{S}}
\def\cT{\mathcal{T}}
\def\cU{\mathcal{U}}
\def\cV{\mathcal{V}}
\def\cW{\mathcal{W}}
\def\cX{\mathcal{X}}
\def\cY{\mathcal{Y}}
\def\cZ{\mathcal{Z}}

%% \mathbb Letters
\def\bA{\mathbb{A}}
\def\bB{\mathbb{B}}
\def\bC{\mathbb{C}}
\def\bD{\mathbb{D}}
\def\bE{\mathbb{E}}
\def\bF{\mathbb{F}}
\def\bG{\mathbb{G}}
\def\bH{\mathbb{H}}
\def\bI{\mathbb{I}}
\def\bJ{\mathbb{J}}
\def\bK{\mathbb{K}}
\def\bL{\mathbb{L}}
\def\bM{\mathbb{M}}
\def\bN{\mathbb{N}}
\def\bO{\mathbb{O}}
\def\bP{\mathbb{P}}
\def\bQ{\mathbb{Q}}
\def\bR{\mathbb{R}}
\def\bS{\mathbb{S}}
\def\bT{\mathbb{T}}
\def\bU{\mathbb{U}}
\def\bV{\mathbb{V}}
\def\bW{\mathbb{W}}
\def\bX{\mathbb{X}}
\def\bY{\mathbb{Y}}
\def\bZ{\mathbb{Z}}

%% \mathscr Letters, for filtration, sigma algebras etc
\def\sA{\mathscr{A}}
\def\sB{\mathscr{B}}
\def\sC{\mathscr{C}}
\def\sD{\mathscr{D}}
\def\sE{\mathscr{E}}
\def\sF{\mathscr{F}}
\def\sG{\mathscr{G}}
\def\sH{\mathscr{H}}
\def\sI{\mathscr{I}}
\def\sJ{\mathscr{J}}
\def\sK{\mathscr{K}}
\def\sL{\mathscr{L}}
\def\sM{\mathscr{M}}
\def\sN{\mathscr{N}}
\def\sO{\mathscr{O}}
\def\sP{\mathscr{P}}
\def\sQ{\mathscr{Q}}
\def\sR{\mathscr{R}}
\def\sS{\mathscr{S}}
\def\sT{\mathscr{T}}
\def\sU{\mathscr{U}}
\def\sV{\mathscr{V}}
\def\sW{\mathscr{W}}
\def\sX{\mathscr{X}}
\def\sY{\mathscr{Y}}
\def\sZ{\mathscr{Z}}


%%%% \mathsf for Matrices
\def\mA{\mathsf{A}}
\def\mB{\mathsf{B}}
\def\mC{\mathsf{C}}
\def\mD{\mathsf{D}}
\def\mE{\mathsf{E}}
\def\mF{\mathsf{F}}
\def\mG{\mathsf{G}}
\def\mH{\mathsf{H}}
\def\mI{\mathsf{I}}
\def\mJ{\mathsf{J}}
\def\mK{\mathsf{K}}
\def\mL{\mathsf{L}}
\def\mM{\mathsf{M}}
\def\mN{\mathsf{N}}
\def\mO{\mathsf{O}}
\def\mP{\mathsf{P}}
\def\mQ{\mathsf{Q}}
\def\mR{\mathsf{R}}
\def\mS{\mathsf{S}}
\def\mT{\mathsf{T}}
\def\mU{\mathsf{U}}
\def\mV{\mathsf{V}}
\def\mW{\mathsf{W}}
\def\mX{\mathsf{X}}
\def\mY{\mathsf{Y}}
\def\mZ{\mathsf{Z}}


%%%% \mathbf for Matrices or vectors
\def\bfB{\boldsymbol{B}}
\def\bfC{\boldsymbol{C}}
\def\bfD{\boldsymbol{D}}
\def\bfA{\boldsymbol{A}}
\def\bfE{\boldsymbol{E}}
\def\bfF{\boldsymbol{F}}
\def\bfG{\boldsymbol{G}}
\def\bfH{\boldsymbol{H}}
\def\bfI{\boldsymbol{I}}
\def\bfJ{\boldsymbol{J}}
\def\bfK{\boldsymbol{K}}
\def\bfL{\boldsymbol{L}}
\def\bfM{\boldsymbol{M}}
\def\bfN{\boldsymbol{N}}
\def\bfO{\boldsymbol{O}}
\def\bfP{\boldsymbol{P}}
\def\bfQ{\boldsymbol{Q}}
\def\bfR{\boldsymbol{R}}
\def\bfS{\boldsymbol{S}}
\def\bfT{\boldsymbol{T}}
\def\bfU{\boldsymbol{U}}
\def\bfV{\boldsymbol{V}}
\def\bfW{\boldsymbol{W}}
\def\bfX{\boldsymbol{X}}
\def\bfY{\boldsymbol{Y}}
\def\bfZ{\boldsymbol{Z}}

%%%% \mathbf for Matrices
\def\bfa{\boldsymbol{a}}
\def\bfb{\boldsymbol{b}}
\def\bfc{\boldsymbol{c}}
\def\bfd{\boldsymbol{d}}
\def\bfe{\boldsymbol{e}}
\def\bff{\boldsymbol{f}}
\def\bfg{\boldsymbol{g}}
\def\bfh{\boldsymbol{h}}
\def\bfi{\boldsymbol{i}}
\def\bfj{\boldsymbol{j}}
\def\bfk{\boldsymbol{k}}
\def\bfl{\boldsymbol{l}}
\def\bfm{\boldsymbol{m}}
\def\bfn{\boldsymbol{n}}
\def\bfo{\boldsymbol{o}}
\def\bfp{\boldsymbol{p}}
\def\bfq{\boldsymbol{q}}
\def\bfr{\boldsymbol{r}}
\def\bfs{\boldsymbol{s}}
\def\bft{\boldsymbol{t}}
\def\bfu{\boldsymbol{u}}
\def\bfv{\boldsymbol{v}}
\def\bfw{\boldsymbol{w}}
\def\bfx{\boldsymbol{x}}
\def\bfy{\boldsymbol{y}}
\def\bfz{\boldsymbol{z}}

%%%% \boldsymbol for lower case greek letter in math mode
\def\bfalpha{\boldsymbol{\alpha}}
\def\bfbeta{\boldsymbol{\beta}}
\def\bfgamma{\boldsymbol{\gamma}}
\def\bfdelta{\boldsymbol{\delta}}
\def\bfepsilon{\boldsymbol{\epsilon}}
\def\bfzeta{\boldsymbol{\zeta}}
\def\bfeta{\boldsymbol{\eta}}
\def\bftheta{\boldsymbol{\theta}}
\def\bfiota{\boldsymbol{\iota}}
\def\bfkappa{\boldsymbol{\kappa}}
\def\bflambda{\boldsymbol{\lambda}}
\def\bfmu{\boldsymbol{\mu}}
\def\bfnu{\boldsymbol{\nu}}
\def\bfomicron{\boldsymbol{\omicron}}
\def\bfpi{\boldsymbol{\pi}}
\def\bfrho{\boldsymbol{\rho}}
\def\bfsigma{\boldsymbol{\sigma}}
\def\bftau{\boldsymbol{\tau}}
\def\bfupsilon{\boldsymbol{\upsilon}}
\def\bfphi{\boldsymbol{\phi}}
\def\bfchi{\boldsymbol{\chi}}
\def\bfpsi{\boldsymbol{\psi}}
\def\bfomega{\boldsymbol{\omega}}

%%%% \boldsymbol for upper case greek letter in math mode
\def\bfPhi{\boldsymbol{\Phi}}
\def\bfTheta{\boldsymbol{\Theta}}
\def\bfPsi{\boldsymbol{\Psi}}
\def\bfOmega{\boldsymbol{\Omega}}
\def\bfSigma{\boldsymbol{\Sigma}}


%%%%%%%%%%%%%%%%%% Shortcuts %%%%%%%%%%%%%%%%%%%%%%%%%%%%%%%%%%%%%%%%%%%
\newcommand{\wt}{\widetilde}
\newcommand{\wh}{\widehat}

%%%%%%%%%%%%%%%%%%%%%%%%%%%%%%%%%%%%%%%%%%%%%%%%%%%%%%%%%%%%%%%%%%%%%%%%%%%%%%%%%
%%%%%%%%%%%%%%%%%%   Nonstandard notations  %%%%%%%%%%%%%%%%%%%%%%%%%%%%%%%%%%%%%
\newcommand{\pd}[1]{\partial_{#1}}      % partial derivative
\newcommand{\1}{\mathbbm{1}}            % preferable way of writing indicator function
\newcommand{\set}[1]{\{#1\}}            % set: {xyz} to be used for inline formulas
\newcommand{\Set}[1]{\left\{#1\right\}} % set: {xyz} to be used for separate (not inline) formulas
\renewcommand{\mid}{\,|\,}              % mid bar with small spaces before and after: x | y
\newcommand{\Mid}{\,\Big | \,}          % big bar with small spaces before and after:
\newcommand{\norm}[1]{ \| #1 \| }       % mid bar with small spaces before and after: x | y
\newcommand{\abs}[1]{\left\vert#1\right\vert}   % absolute value
\newcommand{\CB}[1]{\left\{ #1 \right\}}
\newcommand{\SB}[1]{\left[ #1 \right]}
\newcommand{\Pare}[1]{\left( #1 \right)}
\newcommand{\AB}[1]{\left \langle #1 \right \rangle}
\newcommand{\given}[1]{\left.#1\right|}
\newcommand{\givenAlt}[1]{\left.#1\right]}
\newcommand{\Tran}[1]{{#1}^\top}
\newcommand{\bsde}{BS$\Delta$E}         % BS\DeltaE
\newcommand{\bsdes}{BS$\Delta$Es}       % BS\DeltaE
\newcommand{\ow}{\text{otherwise}}
\newcommand{\imblies}{\Longleftarrow}
\newcommand{\Exp}[1]{\mathrm{E}\left[ #1 \right]}
\newcommand{\Var}[1]{\mathrm{Var}\left[ #1 \right]}
\newcommand{\Cov}[1]{\mathrm{Cov}\left[ #1 \right]}
\newcommand{\Bspace}{\;\;\;\;}
\newcommand{\dif}{\,\mathrm{d}}        % used for differential, same as in commath.sty

\newcommand{\LHS}{\text{LHS}} % left hand side
\newcommand{\RHS}{\text{RHS}} % right hand side
\newcommand{\Adj}{\text{Adj}} %adjoint matrix

\newcommand{\RNum}[1]{\uppercase\expandafter{\romannumeral #1\relax}} % romanian numerals

\DeclareMathOperator{\logit}{logit}
\DeclareMathOperator*{\esssup}{ess\,sup} % ess sup
\DeclareMathOperator*{\essinf}{ess\,inf} % ess inf
\DeclareMathOperator*{\esslimsup}{ess\,\limsup}
\DeclareMathOperator*{\essliminf}{ess\,\liminf}
\DeclareMathOperator*{\argmin}{arg\,min} % argmin
\DeclareMathOperator*{\diag}{diag} % diag
\DeclareMathOperator*{\argmax}{arg\,max} % argmax
\DeclareMathOperator*{\Arg}{Arg} % arguments
\DeclareMathOperator*{\rank}{rank\,} % argmax
\DeclareMathOperator*{\KL}{KL} % KL divergence
\DeclareMathOperator*{\Proj}{Proj} % Projection
\DeclareMathOperator{\Std}{\mathrm{Std}} % \std for Standard deviation
\DeclareMathOperator{\sgn}{\mathrm{sgn}} % sign of a variable
\DeclareMathOperator{\tr}{\mathrm{trace}} % matrix trace
% trigonometric and hyperbolic functions
\DeclareMathOperator{\sech}{sech}
\DeclareMathOperator{\csch}{csch}
\DeclareMathOperator{\arcsec}{arcsec}
\DeclareMathOperator{\arccot}{arcCot}
\DeclareMathOperator{\arccsc}{arcCsc}
\DeclareMathOperator{\arccosh}{arcCosh}
\DeclareMathOperator{\arcsinh}{arcsinh}
\DeclareMathOperator{\arctanh}{arctanh}
\DeclareMathOperator{\arcsech}{arcsech}
\DeclareMathOperator{\arccsch}{arcCsch}
\DeclareMathOperator{\arccoth}{arcCoth} 

%%%%%%%%%%%%%%%%%%%%%%%%%%%%%%% FINANCE %%%%%%%%%%%%%%%%%%%%%%%%%%%%%%%%%%%%%%%%%%%%%%%%%%
\DeclareMathOperator{\var}{\mathrm{V}@\mathrm{R}}           % \V@R Value-at-risk
\DeclareMathOperator{\tce}{\mathrm{TCE}}                    % Tail Conditional Expectation
\DeclareMathOperator{\tvar}{\mathrm{TV}@\mathrm{R}}         % \TV@R tail Value-at-risk
\DeclareMathOperator{\avar}{\mathrm{AV}@\mathrm{R}}         % \AV@R average Value-at-risk
\DeclareMathOperator{\ent}{\mathrm{Ent}}                    % \ent = Entropic Risk Measure
\DeclareMathOperator{\glr}{\mathrm{GLR}}                    % \glr = gain to loss ratio

\DeclareMathOperator{\ES}{\mathrm{ES}}

\newcommand{\ask}{\text{ask}}           % ask price
\newcommand{\bid}{\text{bid}}           % bid price

%%%%%%%%%%%%%%%%%%%%%%%%%%%%%%%%%%%%%%%%%%%%%%%%%%%%%%%
% \bibliographystyle{amsplain}
% \bibliographystyle{alpha} % standard LaTeX bibliography format. Preferable to be used
% \bibliography{MathFinanceMaster-12-28-2014}
%\bibliography{D:/_research/latex/lib_igor/igor_bib_mathfinance}
% help on how to use several bit files \bibliography{videogames,comics,interface,theory}
%%%%%%%%%%%%%%%%%%%%%%%%%%%%%%%%%%%%%%%%%%%%%%%%%%%%%%%




\title{cs577 Assignment 2: Solution}
\author{Yuanxing Cheng, A20453410, CS577-f22\\ Department of Mathematics \\Illinois Institute of Technology}

\begin{document}

\maketitle

\section*{Theoretical questions}

\subsection*{Loss}
\subsubsection*{1}
\begin{myleftlinebox}
    Write the equation for L1, L2, Huber and Log-cosh loss function and compare them. Explain their advantages/purposes.
    \tcblower
    Let \(d_j=\hat y_j^{(i)}-y_j^{(i)}\) for \(i\)-th sample at \(j\)-th dimension, we have
    \begin{itemize}
        \item L1 loss: \(\sum_j \abs{d_j}\)
        \item L2 loss: \(\sum_j d_j^2\)
        \item Huber loss: \(\sum_j \rho_\sigma(d_j)\) where \(\rho_\sigma(d)=\begin{cases}
            \frac{1}{2}d^2, &\abs{d}\leq\sigma\\
            \sigma(d-\frac{1}{2}\sigma), &\ow
        \end{cases}\)
        \item Log-cosh loss: \(\sum_j \log(\cosh(d_j))\)
    \end{itemize}
    Their advantages/purposes are:
    \begin{itemize}
        \item L1 loss: use absolute value to avoid cancelling of positive and negative \(d_j\)
        \item L2 loss: use square to avoid this cancelling while make loss differentiable
        \item Huber loss: to cap the loss for outliers
        \item log-cosh: cap the loss for outliers and differentiable
    \end{itemize}
\end{myleftlinebox}
\subsubsection*{2}
\begin{myleftlinebox}
    Write the equation for cross-entropy loss and explain how it is derived using maximum log-likelihood. Explain the worst cross-entropy value you expect for random assignment.
    \tcbline
    \[\hat y_j^{(i)}=P(y=j\mid X^{(i)})\]
    then the negative log likelihood is
    \[l(\theta) = -\log\Pare{\prod_{i=1}^m\prod_{j=1}^k P(y=j\mid X^{(i)})^{y_j^{(i)}} }=-\sum_{i=1}^m\sum_{j=1}^k y_j^{(i)}\log\Pare{\hat y_j^{(i)}}  \]
    The random assignment will lead to \(-\log k\).
\end{myleftlinebox}
\subsubsection*{3}
\begin{myleftlinebox}
    Write the equation for softmax loss and describe when to use it
    \tcbline
    \[l(\theta) = -\sum_{i=1}^m\sum_{j=1}^k y_j^{(i)}\log\Pare{\hat y_j^{(i)}}=-\sum_{i=1}^m\sum_{j=1}^k y_j^{(i)}\Pare{z_j-\log\sum_{j=1}^k\exp(z_j)}\]
    When to use: multi-class classification problem and when using softmax as activation before output layer
\end{myleftlinebox}
\subsubsection*{4}
\begin{myleftlinebox}
    Write the equation for Kullback-Liebler loss and explain its meaning. Explain the circumstances under which there is no difference between using cross-entropy or Kullback-Liebler to train the network.
    \tcblower
    \[l(\theta) = \sum_{i=1}^m\sum_{j=1}^k y_j^{(i)} \log\Pare{\frac{y_j^{i}}{\hat y_j^{(i)}}}  \]
    meaning: the similarity between two distributions. This equals to cross-entropy loss when \(\sum_{i=1}^m\sum_{j=1}^k y_j^{(i)} \log\Pare{y_j^{i}}=0\)
\end{myleftlinebox}

\subsubsection*{5}
\begin{myleftlinebox}
    Explain the Hinge loss and squared Hinge loss. Describe the fundamental idea behind it and the worst value you expect for it before learning.
    \tcbline
    Hinge loss: used in binary classification while the labels are \(-1,1\). It's positive if prediction and input disagree in sign or if agree in sign and \(\abs{\hat y^{(i)}}<\sigma\). It's negative if predictions agree in sign and \(\abs{\hat y^{(i)}}>\sigma\). It has the form:
    \[l(\theta)=\sum_j\max\Pare{0,\sigma-y_j^{(i)}\hat y_j^{(i)}}\]
    The squared hinge loss also measure the label distance with margin \(\sigma\). It has the form:
    \[l(\theta)=\frac{1}{2}\sum_{j=1}^k \max\Pare{0,\hat y_j^{(i)}-\hat y^{(i)}_{true}+\sigma}^2\]
    The worst value can go infinity large.
\end{myleftlinebox}

\subsubsection*{6}
\begin{myleftlinebox}
    Compute the hinge loss for a 3-class classification problem with three examples with label \(1,2,3\), with prediction scores \(\hat y^{(1)}=(0.5,0.4,0.3)\), \(\hat y^{(2)}=(1.3,0.8,-0.6)\) and \(\hat y^{(1)}=(1.4,-0.4,2.7)\) (one-against-all-others).
    \tcblower
    \begin{align*}
        L_1 &= \max(0,0.4-0.5+1)+\max(0,0.3-0.5+1)=1.7\\
        L_2 &= \max(0,1.3-0.8+1)+\max(0,-0.6-0.8+1)=1.5\\
        L_3 &= \max(0,1.4-2.7+1)+\max(0,-0.4-2.7+1)=0
    \end{align*}
\end{myleftlinebox}

\subsubsection*{7}
\begin{myleftlinebox}
    Explain the purpose of adding a regularization term to the loss function. Explain the difference between L1 and L2 regularization and how they affect the weight distribution in the network. Explain the way to choose the regularization term coefficient.
    \tcblower
    Purpose:
    \begin{itemize}
        \item simple explanation is better (for better generalization)
        \item when parameters are small enough, we can remove it and make the model simpler
        \item smaller parameters are more stable that will generalize better
    \end{itemize}
    Difference between L1 and L2: L2 is more sensitive to outliers due to the square. Method to choose regularization coefficient is to plot the path of the coefficient \(\lambda\) vs the model parameters.
\end{myleftlinebox}

\subsubsection*{8}
\begin{myleftlinebox}
    Explain how L1 and L2 loss terms affect gradients in the network
    \tcblower
    As regularization term is added into the loss, the gradients also changes. It slightly alters the coefficients by a linear function on \(\lambda\) and thus prevent overfitting.
\end{myleftlinebox}
\subsubsection*{9}
\begin{myleftlinebox}
    Explain the difference between kernel, bias, and activity regularization
    \tcblower
    \begin{itemize}
        \item kernel regularizer: on model parameters $\theta$
        \item bias regularizer: on $\theta_0$
        \item activity regularizer: on \(\hat y\)
    \end{itemize}
\end{myleftlinebox}

\subsection*{Regularization}
\subsubsection*{1}
\begin{myleftlinebox}
    Explain how weight decay is related to adding a regularization term to the loss function.
    \tcblower
    Add the regularization term results in extra term in the gradient and thus in gradient descent method, extra term is substructed and that's weight decay.
\end{myleftlinebox}

\subsubsection*{2}
\begin{myleftlinebox}
    Explain how early stopping to prevent overfitting is performed. Explain the strategies to reuse the validation data.
    \tcblower
    Early stopp when validation error starts to increase or when training error stops decreasing and this prevent overfitting. 
    To reuse validation data, we have 2 strategies.
    \begin{itemize}
        \item use only train data and then train for number of iterations determine from validation data
        \item after early stop, continue trainnnning from previous weights with train data while validation loss is bigger than training loss
    \end{itemize}
\end{myleftlinebox}

\subsubsection*{3}
\begin{myleftlinebox}
    Explain how data augmentation is performed and how it assists in prevent overfitting.
    \tcblower
    Add synthetic data to increase variability in training so that we have better generalization
    \begin{itemize}
        \item augment in feature or data domain
        \item augment by interpolating between examples on by adding noise
        \item augment by transforming img data: crop, rotate, rescale, intensity
        \item other popular method in image classification, that are illumination/rotation/scale invariante
    \end{itemize}
\end{myleftlinebox}

\subsubsection*{4}
\begin{myleftlinebox}
    Explain how dropout is performed. What are advantages/disadvantages of dropout?
    \tcblower
    Steps for drop out.
    \begin{enumerate}
        \item at each training stage, drop out units in fully connected layers with probability \(1-\rho\) where \(\rho\) is a hyper-parameter. 
        \item removed nodes are reinstated with original weights in the subsequent state
    \end{enumerate}
    Advantages:
    \begin{itemize}
        \item reduce node interations
        \item reduce overfitting
        \item increase training speed
        \item reduce dependency on a single node
        \item disttribute features across multiple nodes    
    \end{itemize}
    Disadvantages: longer training time
\end{myleftlinebox}

\subsubsection*{5}
\begin{myleftlinebox}
    Explain how the expected value of all combinations of dropped out networks can be approximated efficiently during testing.
    \tcblower
    At testing, we can multipy the output of each node by dropout probability so that it equals to the expected value.
\end{myleftlinebox}

\subsubsection*{6}
\begin{myleftlinebox}
    Explain how batch normalization is performed during training and during testing. In what way does batch normalization introduces randomness into training.
    \tcblower
    During training the layer output are normalized using output mean \(\mu_j\) and output  standard deviation \(\sigma_j\): \(\hat z_j^{(i)}=\frac{z_j^{(i)}-\mu_j}{\sigma_j}\). These numbers are stored and used again in testing process.
    During training, because batches are randomly selected, batch normalization adds randomness into the training and thus reduces overfitting.
\end{myleftlinebox}

\subsubsection*{7}
\begin{myleftlinebox}
    Explain the purpose of scale and shift parameters in batch normalization. What are the values of scale and shift parameters that will cause the normalization to be canceled? Explain how the scale and shift parameters can be learned and what is a good initial value for them.
    \tcblower
    To terminate training, we need scale and shift after normalization: \(\tilde z_j=r_j \hat z_j+\beta_j\). If \(r_j=\sigma_j\) and \(\beta_j=\mu_j\), batch normalization is canceled.
    These values can be learned if batch normalization is not needed. A good initial values could be zero for mean and one for standard deviation.
\end{myleftlinebox}

\subsubsection*{8}
\begin{myleftlinebox}
    Explain how ensemble classifiers can assist with overfitting. Describe the possible strategies for producing ensemble classifiers.
    \tcblower
    Ensemble classifiers train multiple independent models with:
    \begin{itemize}
        \item change data
        \item change parameters
        \item record multiple snapshots of the model during training with various learning rate
    \end{itemize}    
\end{myleftlinebox}

\end{document}