\documentclass[11pt,a4paper]{article}
\usepackage[utf8]{inputenc}
\usepackage{graphicx}
\graphicspath{ {D:/Notes/others/assets/} }
\usepackage{amsthm, amsmath, amssymb, amsfonts, graphicx, epsfig}

%%%%%%%%%%%%%%%%% this version: 03-14-2017 %%%%%%%%%%%%%%%%%%%%%%%%%%%%%%%

%%% standard packages
\usepackage{amsthm, amsmath, amssymb, amsfonts, graphicx, epsfig}
%\graphicspath{ {D:/Notes/others/assets/} }

\usepackage{algorithm, algpseudocode}

\allowdisplaybreaks
\usepackage{setspace}
%\usepackage{accents}
%%%
%\usepackage{kbordermatrix}

%%%%%%%%%%%%%%%%%%%%%%% ADDITIONAL FONTS %%%%%%%%%%%%%%%%%%%%%%%%%%%%%%
%%% Package to make \mathbbm, in particular to have 1 as for indicator function
\usepackage{bbm}
%%% Package to make special curl fonts, by using \mathscr{F}
\usepackage{mathrsfs}
%%% dsfont sometimes used instead of mathbb or mathbbm
%\usepackage{dsfont}
%%%%%%%%%%%%%%%%%%%%%%%%%%%%%%%%%%%%%%%%%%%%%%%%%%%%%%%%%%%%%%%%%%%%%%%


%%%%%%%%%%%%%%%%%%%%%%%%%%%%%%%%%%%%%%%%%%%%%%%%%%%%%%%%%%%%%%%%%%%%%%
%%% The ulem package provides various types of underlining that can
%%% stretch between words and be broken across lines. Convenient for editing. \sout{xxxx}
%%% http://ctan.unixbrain.com/macros/latex/contrib/ulem/ulem.pdf
%%%% Remark: if used with natbib, then the bibliography comes underlined. Comment this package at last compilation
%\usepackage{ulem}
%%%%%%%%%%%%%%%%%%%%%%%%%%%%%%%%%%%%%%%%%%%%%%%%%%%%%%%%%%%%%%%%%%%%%%

%%%%%%%%%%%%%%%%%%%%%%%%%%%%%%%%%%%%%%%%%%%%%%%%%%%%%%%%%%%%%%%%%%%%%%
%% Cancel is used to cross out in math mode. \xcancel, \cancel Convenient for editting.
%% http://ctan.math.utah.edu/ctan/tex-archive/macros/latex/contrib/cancel/cancel.pdf
%\usepackage{cancel}
%%%%%%%%%%%%%%%%%%%%%%%%%%%%%%%%%%%%%%%%%%%%%%%%%%%%%%%%%%%%%%%%%%%%%%


%%%%%%%%%%%%%%%%%%%%%%%%%%%%%%%%%%%%%%%%%%%%%%%%%%%%%%%%%%%%%%%%%%%%%%%
%This packages adds support of handling eps images to package graphics
%or graphicx with option pdftex. If an eps image is detected, epstopdf is
%automatically called to convert it to pdf format.
\usepackage{epstopdf}
%%%%%%%%%%%%%%%%%%%%%%%%%%%%%%%%%%%%%%%%%%%%%%%%%%%%%%%%%%%%%%%%%%%%%%%


%%%%%%%%%%%%%%%%%%%%%%%%%%%%%%%%%%%%%%%%%%%%%%%%%%%%%%%%%%%%%%%%%%%%%%%
% various features for using graphics, including subfigure, captions, subcaptions etc.
\usepackage{graphicx}
%\usepackage{caption}
\usepackage[font=sl,labelfont=bf]{caption}
\usepackage{subcaption}
%%%%%%%%%%%%%%%%%%%%%%%%%%%%%%%%%%%%%%%%%%%%%%%%%%%%%%%%%%%%%%%%%%%%%%%


%%%%%%%%%%%%%%%%%%%%%%%%%%%%%%%%%%%%%%%%%%%%%%%%%%%%%%%%%%%%%%%%%%%%%%%
% This package improves the interface for defining floating objects such
% as figures and tables in LaTeX.
% http://www.ctan.org/pkg/float
\usepackage{float}
\restylefloat{table}
%%%%%%%%%%%%%%%%%%%%%%%%%%%%%%%%%%%%%%%%%%%%%%%%%%%%%%%%%%%%%%%%%%%%%%%


%%% use for diagonals in the table's cells
%\usepackage{slashbox}  %Removed by Ares

%%%%%%%%%%%%%%%%%%%%%%%%%%%%%%%%%%%%%%%%%%%%%%%%%%%%%%%%%%%%%%%%%%%%%%%


%%%%%%%%%%%%%%%%%%%%%%%%%%%%%%%%%%%%%%%%%%%%%%%%%%%%%%%%%%%%%%%%%%%%%%%
% This package gives the enumerate environment an optional argument
% which determines the style in which the counter is printed.
% http://www.ctex.org/documents/packages/table/enumerate.pdf
\usepackage{enumerate}
%%%%%%%%%%%%%%%%%%%%%%%%%%%%%%%%%%%%%%%%%%%%%%%%%%%%%%%%%%%%%%%%%%%%%%%


%%%%%%%%%%%%%%%%%%%%%%%%%%%%%%%%%%%%%%%%%%%%%%%%%%%%%%%%%%%%%%%%%%%%%%%
%selectively in/exclude pieces of text: the user can determine new comment versions,
%and each is controlled separately. Special comments can be determined where the
%user specifies the action that is to be taken with each comment line.
% http://get-software.net/macros/latex/contrib/comment/comment.pdf
%\usepackage{comment}
%%%%%%%%%%%%%%%%%%%%%%%%%%%%%%%%%%%%%%%%%%%%%%%%%%%%%%%%%%%%%%%%%%%%%%%


%%%%%%%%%%%%%%%%%%%%%%%%%%%%%%%%%%%%%%%%%%%%%%%%%%%%%%%%%%%%%%%%%%%%%%
%% To display the labels used in a tex file in the dvi file (for example, if a theorem is labelled with the \label command) use the package
%% http://www.ctan.org/pkg/showkeys

%%%%%% Use Option 1
%%%%%% This will display all labels (for example, in the case of a labelled theorem, the label of the theorem will occur in the margin of the dvi file.
%%%%%% The command above by itself not only displays labels when they are first named but also when they are cited (or referenced).
%\usepackage{showkeys}

%%%%%% Use Option 2
%%%%%% It will NOT display citations and references. Only Labels
%\usepackage[notref, notcite]{showkeys}

%%%%%% Use Options 3
%%%%%% formats to smaller fonts the labels etc.
%\usepackage[usenames,dvipsnames]{color}
%\providecommand*\showkeyslabelformat[1]{{\normalfont \tiny#1}}
%\usepackage[notref,notcite,color]{showkeys}
%\definecolor{labelkey}{rgb}{0,0,1}
%%%%%%%%%%%%%%%%%%%%%%%%%%%%%%%%%%%%%%%%%%%%%%%%%%%%%%%%%%%%%%%%%%%%%%


%%%%%%%%%%%%%%%%%%%%%%%%%%%%%%%%%%%%%%%%%%%%%%%%%%%%%%%%%%%%%%%%%%%%%%%
% It extends the functionality of all the LATEX cross-referencing commands (including the table of contents, bibliographies etc) to produce \special commands which a driver can turn into hypertext links; it also provides new commands to allow the user to write ad hoc hypertext links, including those to external documents and URLs.
% http://www.tug.org/applications/hyperref/manual.html
\usepackage[colorlinks=true, pdfstartview=FitV, linkcolor=blue,
            citecolor=blue, urlcolor=blue]{hyperref}
\usepackage[usenames]{color}
%%%%%%%%%%%%%%%%%%%%%%%%%%%%%%%%%%%%%%%%%%%%%%%%%%%%%%%%%%%%%%%%%%%%%%%%
% custom textbox
\usepackage{tcolorbox}
\tcbuselibrary{breakable}
\tcbuselibrary{skins}
\tcbset{my left line/.style={
          enhanced, frame hidden, borderline west = {0.5pt}{0pt}{black}, % specify border
          opacityframe=0, opacityback=0,opacityfill=0, % color
          arc = 0mm, left skip=1em, % border
          left = 1mm,top=0mm,bottom=0mm,boxsep=1mm,middle=1mm, % text and border
}}

\newtcbox{\mybox}[1][]{my left line, #1}
\newtcolorbox{myleftlinebox}[1][breakable]{my left line, #1}
% usage
% use \mybox[on line]{your TEXT} for inline box otherwise forcing line breaks   
% for whole break, use \begin{myleftlinebox
%%%%%%%%%%%%%%%%%%%%%%%%%%%%%%%%%%%%%%%%%%%%%%%%%%%%%%%%%%%%%%%%%%%%%%%
\usepackage[utf8]{inputenc}


%%%%%%%%%%%%%%%%%%%%%%%%%%%%%%%%%%%%%%%%%%%%%%%%%%%%%%%%%%%%%%%%%%%%%%%
% a good looking way to format urls
% http://mirror.its.uidaho.edu/pub/tex-archive/help/Catalogue/entries/url.html
\usepackage{url}
% Define a new 'leo' style for the package that will use a smaller font.
\makeatletter\def\url@leostyle{%
 \@ifundefined{selectfont}{\def\UrlFont{\sf}}{\def\UrlFont{\scriptsize\ttfamily}}} \makeatother\urlstyle{leo}
%%%%%%%%%%%%%%%%%%%%%%%%%%%%%%%%%%%%%%%%%%%%%%%%%%%%%%%%%%%%%%%%%%%%%%%


%%%%%%%%%%%%%%%%%%%%%%%%%%%%%%%%%%%%%%%%%%%%%%%%%%%%%%%%%%%%%%%%%%%%%%%%%%%%
%%%%%% The present package defines the environment mdframed which automatically deals with page breaks in framed text.
%%%%%% http://mirrors.ibiblio.org/CTAN/macros/latex/contrib/mdframed/mdframed.pdf
\usepackage[framemethod=default]{mdframed}
%%%%%%%%%%%%%%%%%%%%%%%%%%%%%%%%%%%%%%%%%%%%%%%%%%%%%%%%%%%%%%%%%%%%%%%%%%%%

%%%%%%%%%%%%%%%%%%%%%%%%%%%%%%%%%%%%%%%%%%%%%%%%%%%%%%%%%%%%%%%%%%%%%%%%%%%%%%%%
%%%%%% An extended implementation of the array and tabular environments
%%%%%% which extends the options for column formats, and provides "programmable" format specifications.
%\usepackage{array}
%%%%%%%%%%%%%%%%%%%%%%%%%%%%%%%%%%%%%%%%%%%%%%%%%%%%%%%%%%%%%%%%%%%%%%%%%%%

%%%%%%%%%%%%%%%%%%%%%%%%%%%%%%%%%%%%%%%%%%%%%%%%%%%%%%%%%%%%%%%%%%%%%%%%%%%%
%%% The package enhances the quality of tables in LATEX, providing extra
%%% commands as well as behind-the-scenes optimization. Guidelines
%%% are given as to what constitutes a good table in this context.
%%% From version 1.61, the package offers long table compatibility.
%\usepackage{booktabs}
%%%%%%%%%%%%%%%%%%%%%%%%%%%%%%%%%%%%%%%%%%%%%%%%%%%%%%%%%%%%%%%%%%%%%%%%%%%%%%


\usepackage{tikz}




%% or use GEOMETRY package
\usepackage[margin=1.0in, letterpaper]{geometry}
%\def\baselinestretch{1.1}


%%%%%%%%%%%%% OR USE EXACT DIMENSIONS %
%\setlength{\textwidth}{6.5in}     %%
%\setlength{\oddsidemargin}{0in}   %%
%\setlength{\evensidemargin}{0in}  %%
%\setlength{\textheight}{8.5in}    %%
%\setlength{\topmargin}{0in}       %%
%\setlength{\headheight}{0in}      %%
%\setlength{\headsep}{.3in}         %%
%\setlength{\footskip}{.5in}       %%
%%%%%%%%%%%%%%%%%%%%%%%%%%%%%%%%%%%%%%%%%%%%%%%%%%%%%%%%%%%%%%%%%%%%%%%


%%%%%%%%%%%%%%%%%%%%%%%%%%%%%%%%%%%%%%%%%%%%%%%%%%%%%%%%%%%%%%%%%%%%%%%
%%%%%%%%%%%%%%%%%%%%%%%% NUMBERING %%%%%%%%%%%%%
\newtheorem{theorem}{Theorem}
\newtheorem{conjecture}{Conjecture}
\newtheorem{conclusion}{Conclusion}
\newtheorem{proposition}[theorem]{Proposition}
\newtheorem{lemma}[theorem]{Lemma}
\newtheorem{corollary}[theorem]{Corollary}
\newtheorem{assumption}{Assumption}
\newtheorem{condition}{C}
\theoremstyle{definition}
\newtheorem{definition}[theorem]{Definition}
\newtheorem{example}[theorem]{Example}
\theoremstyle{remark}
\newtheorem{remark}[theorem]{Remark}
\newtheorem{question}[theorem]{Question}
\newtheorem{problem}[theorem]{Problem}
\newtheorem{NB}[theorem]{Nota Bene}

\numberwithin{equation}{section}
\numberwithin{theorem}{section}
%\renewcommand{\labelitemi}{ {\small $\rhd$}}
%%%%%%%%%%%%%%%%%%%%%%%%%%%%%%%%%%%%%%%%%%%%%%%%%%%%%%%%%%%%%%%%%%%%%%%

%%%%%%%%%%%%%%%%%%%%%%%%%%%%%%%%%%%%%%%%%%%%%%%%%%%%%%%%%%%%%%%%%%%%%%%
\definecolor{Red}{rgb}{0.9,0,0.0}
\definecolor{Blue}{rgb}{0,0.0,1.0}
%%%%%%%%%%%%%%%%%%%%%%%%%%%%%%%%%%%%%
%%% used for editing and making comments in color
% Example \ig{Remarks and Commets}
\newcommand{\ig}[1]{\textcolor[rgb]{0.00, 0.0, 1.0}{{\tiny \textsuperscript{[\textrm{IC:Rem}]}} \ #1}}
\newcommand{\igAdd}[1]{\textcolor[rgb]{0.7, 0.0, 0.0}{{\tiny \textsuperscript{[\textrm{IC:Add}]}} \ #1}}
\newcommand{\igEdit}[1]{\textcolor[rgb]{0.7, 0.0, 0.0}{{\tiny \textsuperscript{[\textrm{IC:Edit}]}} \  #1}}

\newcommand{\trb}[1]{\begin{color}[rgb]{0.98, 0.0, 0.98}{TRB: #1} \end{color}}
\newcommand{\ti}[1]{\begin{color}[rgb]{1.00,0.00,0.00}{TI: #1} \end{color}}
\newcommand{\Red}[1]{\textcolor{Red}{#1}}
%%%%%%%%%%%%%%%%%%%%%%%%%%%%%%%%%%%%%

%%%%%%%%%%%%%%%%%%%%%%%%%%%%%%%%%%%%%
%%%     Igor's macros
%% \mathcal Letters
\def\cA{\mathcal{A}}
\def\cB{\mathcal{B}}
\def\cC{\mathcal{C}}
\def\cD{\mathcal{D}}
\def\cE{\mathcal{E}}
\def\cF{\mathcal{F}}
\def\cG{\mathcal{G}}
\def\cH{\mathcal{H}}
\def\cI{\mathcal{I}}
\def\cJ{\mathcal{J}}
\def\cK{\mathcal{K}}
\def\cL{\mathcal{L}}
\def\cM{\mathcal{M}}
\def\cN{\mathcal{N}}
\def\cO{\mathcal{O}}
\def\cP{\mathcal{P}}
\def\cQ{\mathcal{Q}}
\def\cR{\mathcal{R}}
\def\cS{\mathcal{S}}
\def\cT{\mathcal{T}}
\def\cU{\mathcal{U}}
\def\cV{\mathcal{V}}
\def\cW{\mathcal{W}}
\def\cX{\mathcal{X}}
\def\cY{\mathcal{Y}}
\def\cZ{\mathcal{Z}}

%% \mathbb Letters
\def\bA{\mathbb{A}}
\def\bB{\mathbb{B}}
\def\bC{\mathbb{C}}
\def\bD{\mathbb{D}}
\def\bE{\mathbb{E}}
\def\bF{\mathbb{F}}
\def\bG{\mathbb{G}}
\def\bH{\mathbb{H}}
\def\bI{\mathbb{I}}
\def\bJ{\mathbb{J}}
\def\bK{\mathbb{K}}
\def\bL{\mathbb{L}}
\def\bM{\mathbb{M}}
\def\bN{\mathbb{N}}
\def\bO{\mathbb{O}}
\def\bP{\mathbb{P}}
\def\bQ{\mathbb{Q}}
\def\bR{\mathbb{R}}
\def\bS{\mathbb{S}}
\def\bT{\mathbb{T}}
\def\bU{\mathbb{U}}
\def\bV{\mathbb{V}}
\def\bW{\mathbb{W}}
\def\bX{\mathbb{X}}
\def\bY{\mathbb{Y}}
\def\bZ{\mathbb{Z}}

%% \mathscr Letters, for filtration, sigma algebras etc
\def\sA{\mathscr{A}}
\def\sB{\mathscr{B}}
\def\sC{\mathscr{C}}
\def\sD{\mathscr{D}}
\def\sE{\mathscr{E}}
\def\sF{\mathscr{F}}
\def\sG{\mathscr{G}}
\def\sH{\mathscr{H}}
\def\sI{\mathscr{I}}
\def\sJ{\mathscr{J}}
\def\sK{\mathscr{K}}
\def\sL{\mathscr{L}}
\def\sM{\mathscr{M}}
\def\sN{\mathscr{N}}
\def\sO{\mathscr{O}}
\def\sP{\mathscr{P}}
\def\sQ{\mathscr{Q}}
\def\sR{\mathscr{R}}
\def\sS{\mathscr{S}}
\def\sT{\mathscr{T}}
\def\sU{\mathscr{U}}
\def\sV{\mathscr{V}}
\def\sW{\mathscr{W}}
\def\sX{\mathscr{X}}
\def\sY{\mathscr{Y}}
\def\sZ{\mathscr{Z}}


%%%% \mathsf for Matrices
\def\mA{\mathsf{A}}
\def\mB{\mathsf{B}}
\def\mC{\mathsf{C}}
\def\mD{\mathsf{D}}
\def\mE{\mathsf{E}}
\def\mF{\mathsf{F}}
\def\mG{\mathsf{G}}
\def\mH{\mathsf{H}}
\def\mI{\mathsf{I}}
\def\mJ{\mathsf{J}}
\def\mK{\mathsf{K}}
\def\mL{\mathsf{L}}
\def\mM{\mathsf{M}}
\def\mN{\mathsf{N}}
\def\mO{\mathsf{O}}
\def\mP{\mathsf{P}}
\def\mQ{\mathsf{Q}}
\def\mR{\mathsf{R}}
\def\mS{\mathsf{S}}
\def\mT{\mathsf{T}}
\def\mU{\mathsf{U}}
\def\mV{\mathsf{V}}
\def\mW{\mathsf{W}}
\def\mX{\mathsf{X}}
\def\mY{\mathsf{Y}}
\def\mZ{\mathsf{Z}}


%%%% \mathbf for Matrices or vectors
\def\bfB{\boldsymbol{B}}
\def\bfC{\boldsymbol{C}}
\def\bfD{\boldsymbol{D}}
\def\bfA{\boldsymbol{A}}
\def\bfE{\boldsymbol{E}}
\def\bfF{\boldsymbol{F}}
\def\bfG{\boldsymbol{G}}
\def\bfH{\boldsymbol{H}}
\def\bfI{\boldsymbol{I}}
\def\bfJ{\boldsymbol{J}}
\def\bfK{\boldsymbol{K}}
\def\bfL{\boldsymbol{L}}
\def\bfM{\boldsymbol{M}}
\def\bfN{\boldsymbol{N}}
\def\bfO{\boldsymbol{O}}
\def\bfP{\boldsymbol{P}}
\def\bfQ{\boldsymbol{Q}}
\def\bfR{\boldsymbol{R}}
\def\bfS{\boldsymbol{S}}
\def\bfT{\boldsymbol{T}}
\def\bfU{\boldsymbol{U}}
\def\bfV{\boldsymbol{V}}
\def\bfW{\boldsymbol{W}}
\def\bfX{\boldsymbol{X}}
\def\bfY{\boldsymbol{Y}}
\def\bfZ{\boldsymbol{Z}}

%%%% \mathbf for Matrices
\def\bfa{\boldsymbol{a}}
\def\bfb{\boldsymbol{b}}
\def\bfc{\boldsymbol{c}}
\def\bfd{\boldsymbol{d}}
\def\bfe{\boldsymbol{e}}
\def\bff{\boldsymbol{f}}
\def\bfg{\boldsymbol{g}}
\def\bfh{\boldsymbol{h}}
\def\bfi{\boldsymbol{i}}
\def\bfj{\boldsymbol{j}}
\def\bfk{\boldsymbol{k}}
\def\bfl{\boldsymbol{l}}
\def\bfm{\boldsymbol{m}}
\def\bfn{\boldsymbol{n}}
\def\bfo{\boldsymbol{o}}
\def\bfp{\boldsymbol{p}}
\def\bfq{\boldsymbol{q}}
\def\bfr{\boldsymbol{r}}
\def\bfs{\boldsymbol{s}}
\def\bft{\boldsymbol{t}}
\def\bfu{\boldsymbol{u}}
\def\bfv{\boldsymbol{v}}
\def\bfw{\boldsymbol{w}}
\def\bfx{\boldsymbol{x}}
\def\bfy{\boldsymbol{y}}
\def\bfz{\boldsymbol{z}}

%%%% \boldsymbol for lower case greek letter in math mode
\def\bfalpha{\boldsymbol{\alpha}}
\def\bfbeta{\boldsymbol{\beta}}
\def\bfgamma{\boldsymbol{\gamma}}
\def\bfdelta{\boldsymbol{\delta}}
\def\bfepsilon{\boldsymbol{\epsilon}}
\def\bfzeta{\boldsymbol{\zeta}}
\def\bfeta{\boldsymbol{\eta}}
\def\bftheta{\boldsymbol{\theta}}
\def\bfiota{\boldsymbol{\iota}}
\def\bfkappa{\boldsymbol{\kappa}}
\def\bflambda{\boldsymbol{\lambda}}
\def\bfmu{\boldsymbol{\mu}}
\def\bfnu{\boldsymbol{\nu}}
\def\bfomicron{\boldsymbol{\omicron}}
\def\bfpi{\boldsymbol{\pi}}
\def\bfrho{\boldsymbol{\rho}}
\def\bfsigma{\boldsymbol{\sigma}}
\def\bftau{\boldsymbol{\tau}}
\def\bfupsilon{\boldsymbol{\upsilon}}
\def\bfphi{\boldsymbol{\phi}}
\def\bfchi{\boldsymbol{\chi}}
\def\bfpsi{\boldsymbol{\psi}}
\def\bfomega{\boldsymbol{\omega}}

%%%% \boldsymbol for upper case greek letter in math mode
\def\bfPhi{\boldsymbol{\Phi}}
\def\bfTheta{\boldsymbol{\Theta}}
\def\bfPsi{\boldsymbol{\Psi}}
\def\bfOmega{\boldsymbol{\Omega}}
\def\bfSigma{\boldsymbol{\Sigma}}


%%%%%%%%%%%%%%%%%% Shortcuts %%%%%%%%%%%%%%%%%%%%%%%%%%%%%%%%%%%%%%%%%%%
\newcommand{\wt}{\widetilde}
\newcommand{\wh}{\widehat}

%%%%%%%%%%%%%%%%%%%%%%%%%%%%%%%%%%%%%%%%%%%%%%%%%%%%%%%%%%%%%%%%%%%%%%%%%%%%%%%%%
%%%%%%%%%%%%%%%%%%   Nonstandard notations  %%%%%%%%%%%%%%%%%%%%%%%%%%%%%%%%%%%%%
\newcommand{\pd}[1]{\partial_{#1}}      % partial derivative
\newcommand{\1}{\mathbbm{1}}            % preferable way of writing indicator function
\newcommand{\set}[1]{\{#1\}}            % set: {xyz} to be used for inline formulas
\newcommand{\Set}[1]{\left\{#1\right\}} % set: {xyz} to be used for separate (not inline) formulas
\renewcommand{\mid}{\,|\,}              % mid bar with small spaces before and after: x | y
\newcommand{\Mid}{\,\Big | \,}          % big bar with small spaces before and after:
\newcommand{\norm}[1]{ \| #1 \| }       % mid bar with small spaces before and after: x | y
\newcommand{\abs}[1]{\left\vert#1\right\vert}   % absolute value
\newcommand{\CB}[1]{\left\{ #1 \right\}}
\newcommand{\SB}[1]{\left[ #1 \right]}
\newcommand{\Pare}[1]{\left( #1 \right)}
\newcommand{\AB}[1]{\left \langle #1 \right \rangle}
\newcommand{\given}[1]{\left.#1\right|}
\newcommand{\givenAlt}[1]{\left.#1\right]}
\newcommand{\Tran}[1]{{#1}^\top}
\newcommand{\bsde}{BS$\Delta$E}         % BS\DeltaE
\newcommand{\bsdes}{BS$\Delta$Es}       % BS\DeltaE
\newcommand{\ow}{\text{otherwise}}
\newcommand{\imblies}{\Longleftarrow}
\newcommand{\Exp}[1]{\mathrm{E}\left[ #1 \right]}
\newcommand{\Var}[1]{\mathrm{Var}\left[ #1 \right]}
\newcommand{\Cov}[1]{\mathrm{Cov}\left[ #1 \right]}
\newcommand{\Bspace}{\;\;\;\;}
\newcommand{\dif}{\,\mathrm{d}}        % used for differential, same as in commath.sty

\newcommand{\LHS}{\text{LHS}} % left hand side
\newcommand{\RHS}{\text{RHS}} % right hand side
\newcommand{\Adj}{\text{Adj}} %adjoint matrix

\newcommand{\RNum}[1]{\uppercase\expandafter{\romannumeral #1\relax}} % romanian numerals

\DeclareMathOperator{\logit}{logit}
\DeclareMathOperator*{\esssup}{ess\,sup} % ess sup
\DeclareMathOperator*{\essinf}{ess\,inf} % ess inf
\DeclareMathOperator*{\esslimsup}{ess\,\limsup}
\DeclareMathOperator*{\essliminf}{ess\,\liminf}
\DeclareMathOperator*{\argmin}{arg\,min} % argmin
\DeclareMathOperator*{\diag}{diag} % diag
\DeclareMathOperator*{\argmax}{arg\,max} % argmax
\DeclareMathOperator*{\Arg}{Arg} % arguments
\DeclareMathOperator*{\rank}{rank\,} % argmax
\DeclareMathOperator*{\KL}{KL} % KL divergence
\DeclareMathOperator*{\Proj}{Proj} % Projection
\DeclareMathOperator{\Std}{\mathrm{Std}} % \std for Standard deviation
\DeclareMathOperator{\sgn}{\mathrm{sgn}} % sign of a variable
\DeclareMathOperator{\tr}{\mathrm{trace}} % matrix trace
% trigonometric and hyperbolic functions
\DeclareMathOperator{\sech}{sech}
\DeclareMathOperator{\csch}{csch}
\DeclareMathOperator{\arcsec}{arcsec}
\DeclareMathOperator{\arccot}{arcCot}
\DeclareMathOperator{\arccsc}{arcCsc}
\DeclareMathOperator{\arccosh}{arcCosh}
\DeclareMathOperator{\arcsinh}{arcsinh}
\DeclareMathOperator{\arctanh}{arctanh}
\DeclareMathOperator{\arcsech}{arcsech}
\DeclareMathOperator{\arccsch}{arcCsch}
\DeclareMathOperator{\arccoth}{arcCoth} 

%%%%%%%%%%%%%%%%%%%%%%%%%%%%%%% FINANCE %%%%%%%%%%%%%%%%%%%%%%%%%%%%%%%%%%%%%%%%%%%%%%%%%%
\DeclareMathOperator{\var}{\mathrm{V}@\mathrm{R}}           % \V@R Value-at-risk
\DeclareMathOperator{\tce}{\mathrm{TCE}}                    % Tail Conditional Expectation
\DeclareMathOperator{\tvar}{\mathrm{TV}@\mathrm{R}}         % \TV@R tail Value-at-risk
\DeclareMathOperator{\avar}{\mathrm{AV}@\mathrm{R}}         % \AV@R average Value-at-risk
\DeclareMathOperator{\ent}{\mathrm{Ent}}                    % \ent = Entropic Risk Measure
\DeclareMathOperator{\glr}{\mathrm{GLR}}                    % \glr = gain to loss ratio

\DeclareMathOperator{\ES}{\mathrm{ES}}

\newcommand{\ask}{\text{ask}}           % ask price
\newcommand{\bid}{\text{bid}}           % bid price

%%%%%%%%%%%%%%%%%%%%%%%%%%%%%%%%%%%%%%%%%%%%%%%%%%%%%%%
% \bibliographystyle{amsplain}
% \bibliographystyle{alpha} % standard LaTeX bibliography format. Preferable to be used
% \bibliography{MathFinanceMaster-12-28-2014}
%\bibliography{D:/_research/latex/lib_igor/igor_bib_mathfinance}
% help on how to use several bit files \bibliography{videogames,comics,interface,theory}
%%%%%%%%%%%%%%%%%%%%%%%%%%%%%%%%%%%%%%%%%%%%%%%%%%%%%%%




\title{Notes for Bayesian Data Analysis 3}
\author{Yuanxing Cheng}


\begin{document}

\maketitle
\tableofcontents
\newpage

\section{Probability and inference}
\subsection{The three steps of Bayesian data analysis}

\begin{itemize}
    \item Full probability model: a joint probability distribution of all observable and unobservable, \emph{remember the underlying knowledge and data collection process}
    \item Conditioning on observed data: get posterior distribution, i.e. the conditional probability distri of the unobserved quantities, \emph{given the observed data}
    \item Evaluating the fit of the model, and posterior. \emph{How good? Sensitivity to assumptions?}
\end{itemize}

\subsection{General notation for statistical inference}

Population, sample, estimates, parameters, etc.

\subsubsection*{Parameters, data, and predictions}

Denote $\theta$ as unobservable parameter vector, \(y\) as the observed data. \(\tilde{y}\) as unknown but observable data.

\subsubsection*{Observational units and variables}

Data, of \(n\) objects. Write \(y=\Pare{y_1,\dots,y_n}\) or \(\Tran{y}\). Notice \(y_i\) itself could be a vector, then the entire \(y\) is a \(n\) row matrix.

\subsubsection*{Exchangeability}

\(n\) values \(y_i\) may be regarded as exchangeable. Then the joint pdf \(p(y_1,\dots,y_n\) is invariant to permutations of indexes.

\subsubsection*{Explanatory variables}

Or \emph{covariates}. Use \(X\) to denote the entire set of explanatory variables for all \(n\) units. If there're \(k\) explanatory variables, then \(X\) is a matrix of \(n\times k\).

\subsubsection*{Hierarchical modeling}
Or \emph{multilevel models}. It's possible here to assume the exchangeability at each level of units.

\subsection{Bayesian inference}
Conclude about a parameter vector \(\theta\) or unobserved data \(\tilde{ y}\) in probability statements, usually denoted as \(p(\theta\mid y) \) or \(p(\tilde{ y}\mid y)\). And also implicitly condition on the known values \(x\).

\subsubsection*{Probability notation}
\(p(\cdot\mid\cdot)\) denotes a conditional pdf w/ the arguments determined by the context. \(p(\cdot)\) usually denotes a marginal distribution. And if for example \(\theta\sim\cN(\mu,\sigma^2)\), we also write \(p(\theta)=\cN(\theta\mid\mu,\sigma^2)\).

The geometric mean is \(\exp\Pare{\Exp{\log\theta}}\)

\subsubsection*{Bayes' rule}

Of prior \(p(\theta)\) and sample distribution \(p(y\mid\theta)\), we have \[p(\theta,y)=p(\theta)p(y\mid\theta).\] Then by Bayes' rule we have the \emph{posterior}: 
\begin{equation}\label{1.1}
    p(\theta\mid y)=\frac{p(\theta,y)}{p(y)}=\frac{p(\theta)p(y\mid\theta)}{p(y)},    
\end{equation}

where \(p(y)=\sum_\theta p(\theta)p(y\mid \theta)=\int p(\theta)p(y\mid \theta) \dif\theta\) is the total probability. Usually we write above in the following form
\begin{equation}\label{1.2}
    p(\theta\mid y)\propto p(\theta)p(y\mid\theta).
\end{equation}

\subsubsection*{Prediction}

The \emph{\Red{prior} predictive distribution} is 
\begin{equation}\label{1.3}
    p(y)=\sum_\theta p(y,\theta)= \sum_\theta p(\theta)p(y\mid \theta)=\int p(y,\theta)\dif\theta=\int p(\theta)p(y\mid \theta) \dif\theta.
\end{equation}

Then we predict an observable \(\tilde{ y}\). Then its distribution is \emph{\Red{posterior} predictive distribution}, with formula
\begin{align}
    p(\tilde{y}\mid y) &=\int p(\tilde y,\theta\mid y)\dif \theta\nonumber\\
    &=\int p(\tilde{y}\mid\theta,y)p(\theta\mid y)\dif\theta\Bspace\textrm{Given $\theta$, $y$ and $\tilde y$ are independent}\nonumber\\
    &=\int p(\tilde{y}\mid \theta)p(\theta\mid y)\dif\theta\label{1.4}
\end{align}

\subsubsection*{Likelihood}

From above \ref{1.4}, data \(y\) affect the posterior only through \(p(y\mid\theta)\), i.e., the likelihood function when \(y\) is fixed. This is the likelihood principle.

\subsubsection*{Likelihood and odds ratio}

Define \emph{posterior odds} for two parameters \(\theta_1\) and \(\theta_2\) to be 
\begin{equation}\label{1.5}
    \frac{p(\theta_1\mid y)}{p(\theta_2\mid y)}=\frac{p(\theta_1)p(y\mid\theta_1)/p(y)}{p(\theta_2)p(y\mid\theta_2)/p(y)}=\frac{p(\theta_1)p(y\mid\theta_1)}{p(\theta_2)p(y\mid\theta_2)},
\end{equation}

The later part is \emph{likelihood ratio} thus we have: \emph{posterior odds=prior odds times likelihood ratio}

\subsection{Discrete examples: genetics and spell checking}

2 examples,
\subsection{Probability as a measure of uncertainty}

Basically, the idea is the bayesian methods are more subjective due to the reliance on a prior distribution.
\subsection{Example: probability from football point spreads}
\subsection{Example: calibration for record linkage}
\subsection{Some useful results from probability theory}

Regarding the joint density, we have the following
\begin{align*}
    p(u)&=\int p(u,v)\dif v\\
    p(u,v,w)&=p(u\mid v,w)p(v\mid w)p(w)\\
    p(u,v\mid w)&=p(v\mid u,w)P(u\mid w)=p(u\mid v,w)p(v\mid w)
\end{align*}

In vector calculus, we define covariance matrix as
\[\Cov{u}=\int(u-\Exp{u})\Tran{(u-\Exp u)}p(u)\dif u\]

And conditional expectation is a function of conditioned variables. For example \(\Exp{u\mid v}\) is a function of \(v\). And we have the following formula
\begin{align}
    \Exp{u}&=\Exp{\Exp{u\mid v}}\\
    \Exp{u}&=\int\int u\cdot p(u,v)\dif u\dif v=\int\int u\cdot p(u\mid v)\dif u\; p(v)\dif v\\
        &=\int \Exp{u\mid v}p(v)\dif v\\
    \Var{u}&=\Exp{\Var{u\mid v}}+\Var{\Exp{u\mid v}}
\end{align}

\subsubsection*{Transformation of variables}
Denote \(p_u(u)\) the density for \(u\) and transformation is \(v=f(u)\). If \(p_u\) is discrete and \(f\) is one-to-one, then \(p_v(v)=p_u(f^{-1}(v))\). And if \(f\) is many-to-one, then we need to sum those probabilities of same value of \(f(u)\).

And if \(p_u\) is continuous, and \(f\) is one-to-one, then \(p_v(v)=\abs{J}p_u(f^{-1}(v))\) where \(\abs{J}\) is the absolute value of the determinant of Jacobian, and can be denoted as \(\frac{\partial u}{\partial v}\) even in vector form.

A useful 1-d function, the logarithm
\begin{equation}\label{1.10}
    \logit(u)=\log(\frac{u}{1-u})
\end{equation}

with the inverse \(\logit^{-1}(v)=\frac{e^v}{1+e^v}\).

Another useful function is the probit transformation \(\Phi^{-1}(u)\) where \(\Phi\) is the standard normal cdf.

\subsection{Computation and software}
\subsubsection*{Summarizing inferences by simulation}
\subsubsection*{Sampling using the inverse cumulative distribution function}

For 1-d distribution \(p(v)\) with cdf \(F(v)\), the inverse cdf \(F^{-1}\) can be used to obtain random samples from the distribution \(p\).

\begin{enumerate}
    \item Draw a random value \(U\) from standard uniform
    \item \(v=F^{-1}(U)\) and this \(v\) will be a random draw from \(p\).
\end{enumerate}

\subsubsection*{Simulation of posterior and posterior predictive quantities}

\subsection{Bayesian inference in applied statistics}
\subsection{Selected Exercises}
\section{Single-parameter models}
\subsection{Estimating a probability from binomial data}






















\newpage
\appendix
\section*{Appendices}
\addcontentsline{toc}{section}{Appendices}
\section{Standard probability distribution}
\subsection{Continuous distribution}
\subsubsection*{Uniform}

Standard uniform \(U(0,1)\), equal possibilities. If \(u\sim U(0,1)\), then \(\theta=a+(b-a)u\sim U(a,b)\). A noninformative distribution is obtained in the limit as \(a\to\infty\) and \(b\to\infty\).
\subsubsection*{Univariate normal}

Standard normal \(\cN(0,1)\). If \(z\sim\cN(0,1)\) then \(\theta=\mu+\sigma z\sim \cN(\mu,\sigma^2)\). A noninformative (flat distribution) is obtained in the limit as \(\sigma\to\infty\). And \(\sigma=0\) corresponds to point mass at \(\theta\).

Useful properties: If two independent \(\theta_1\sim\cN(\mu_1,\sigma_1^2)\) and \(\theta_2\sim\cN(\mu_2,\sigma_2^2)\), then \(\theta_1+\theta_2\sim\cN(\mu_1+\mu_2,\sigma_1^2+\sigma_2^2)\). And mixture property states that if \(\theta_1\mid\theta_2\sim\cN(\theta_2,\sigma_1^2)\) and \(\theta_2\sim\cN(\mu_2,\sigma_2^2)\), then \(\theta_1\sim\cN(\mu_2,\sigma_1^2+\sigma_2^2)\).

\subsubsection*{Lognormal}

When \(\log\theta\sim\cN(\mu,\sigma^2)\), \(\theta\) is log normal. Using transformation, its density is $$p(\theta)=\Pare{\sqrt{2\pi}\sigma\theta}^{-1}\exp\Pare{\frac{-1}{2\sigma^2}\Pare{\log\theta-\mu}^2}.$$ Its mean is \(\exp(\mu+\frac{1}{2}\sigma^2)\) and variance is \(\exp(2\mu)\exp(\sigma^2)(\exp(\sigma^2-1))\), and mode is \(\exp(\mu-\sigma^2)\)


\subsubsection*{Multivariate normal}

Standard Multi-normal \(z=(z_1,\dots,z_d)\sim\cN(0,I_d)\) where \(I_d\) is \(d\times d\) identity matrix. If \(z\sim\cN(0,I_d)\) then \(\theta=\mu+Az\sim\cN(\mu,A\Tran{A})\)












\end{document}
