\documentclass{article}

%%%%%%%%%%%%%%%%% this version: 03-14-2017 %%%%%%%%%%%%%%%%%%%%%%%%%%%%%%%

%%% standard packages
\usepackage{amsthm, amsmath, amssymb, amsfonts, graphicx, epsfig}
%\graphicspath{ {D:/Notes/others/assets/} }

\usepackage{algorithm, algpseudocode}

\allowdisplaybreaks
\usepackage{setspace}
%\usepackage{accents}
%%%
%\usepackage{kbordermatrix}

%%%%%%%%%%%%%%%%%%%%%%% ADDITIONAL FONTS %%%%%%%%%%%%%%%%%%%%%%%%%%%%%%
%%% Package to make \mathbbm, in particular to have 1 as for indicator function
\usepackage{bbm}
%%% Package to make special curl fonts, by using \mathscr{F}
\usepackage{mathrsfs}
%%% dsfont sometimes used instead of mathbb or mathbbm
%\usepackage{dsfont}
%%%%%%%%%%%%%%%%%%%%%%%%%%%%%%%%%%%%%%%%%%%%%%%%%%%%%%%%%%%%%%%%%%%%%%%


%%%%%%%%%%%%%%%%%%%%%%%%%%%%%%%%%%%%%%%%%%%%%%%%%%%%%%%%%%%%%%%%%%%%%%
%%% The ulem package provides various types of underlining that can
%%% stretch between words and be broken across lines. Convenient for editing. \sout{xxxx}
%%% http://ctan.unixbrain.com/macros/latex/contrib/ulem/ulem.pdf
%%%% Remark: if used with natbib, then the bibliography comes underlined. Comment this package at last compilation
%\usepackage{ulem}
%%%%%%%%%%%%%%%%%%%%%%%%%%%%%%%%%%%%%%%%%%%%%%%%%%%%%%%%%%%%%%%%%%%%%%

%%%%%%%%%%%%%%%%%%%%%%%%%%%%%%%%%%%%%%%%%%%%%%%%%%%%%%%%%%%%%%%%%%%%%%
%% Cancel is used to cross out in math mode. \xcancel, \cancel Convenient for editting.
%% http://ctan.math.utah.edu/ctan/tex-archive/macros/latex/contrib/cancel/cancel.pdf
%\usepackage{cancel}
%%%%%%%%%%%%%%%%%%%%%%%%%%%%%%%%%%%%%%%%%%%%%%%%%%%%%%%%%%%%%%%%%%%%%%


%%%%%%%%%%%%%%%%%%%%%%%%%%%%%%%%%%%%%%%%%%%%%%%%%%%%%%%%%%%%%%%%%%%%%%%
%This packages adds support of handling eps images to package graphics
%or graphicx with option pdftex. If an eps image is detected, epstopdf is
%automatically called to convert it to pdf format.
\usepackage{epstopdf}
%%%%%%%%%%%%%%%%%%%%%%%%%%%%%%%%%%%%%%%%%%%%%%%%%%%%%%%%%%%%%%%%%%%%%%%


%%%%%%%%%%%%%%%%%%%%%%%%%%%%%%%%%%%%%%%%%%%%%%%%%%%%%%%%%%%%%%%%%%%%%%%
% various features for using graphics, including subfigure, captions, subcaptions etc.
\usepackage{graphicx}
%\usepackage{caption}
\usepackage[font=sl,labelfont=bf]{caption}
\usepackage{subcaption}
%%%%%%%%%%%%%%%%%%%%%%%%%%%%%%%%%%%%%%%%%%%%%%%%%%%%%%%%%%%%%%%%%%%%%%%


%%%%%%%%%%%%%%%%%%%%%%%%%%%%%%%%%%%%%%%%%%%%%%%%%%%%%%%%%%%%%%%%%%%%%%%
% This package improves the interface for defining floating objects such
% as figures and tables in LaTeX.
% http://www.ctan.org/pkg/float
\usepackage{float}
\restylefloat{table}
%%%%%%%%%%%%%%%%%%%%%%%%%%%%%%%%%%%%%%%%%%%%%%%%%%%%%%%%%%%%%%%%%%%%%%%


%%% use for diagonals in the table's cells
%\usepackage{slashbox}  %Removed by Ares

%%%%%%%%%%%%%%%%%%%%%%%%%%%%%%%%%%%%%%%%%%%%%%%%%%%%%%%%%%%%%%%%%%%%%%%


%%%%%%%%%%%%%%%%%%%%%%%%%%%%%%%%%%%%%%%%%%%%%%%%%%%%%%%%%%%%%%%%%%%%%%%
% This package gives the enumerate environment an optional argument
% which determines the style in which the counter is printed.
% http://www.ctex.org/documents/packages/table/enumerate.pdf
\usepackage{enumerate}
%%%%%%%%%%%%%%%%%%%%%%%%%%%%%%%%%%%%%%%%%%%%%%%%%%%%%%%%%%%%%%%%%%%%%%%


%%%%%%%%%%%%%%%%%%%%%%%%%%%%%%%%%%%%%%%%%%%%%%%%%%%%%%%%%%%%%%%%%%%%%%%
%selectively in/exclude pieces of text: the user can determine new comment versions,
%and each is controlled separately. Special comments can be determined where the
%user specifies the action that is to be taken with each comment line.
% http://get-software.net/macros/latex/contrib/comment/comment.pdf
%\usepackage{comment}
%%%%%%%%%%%%%%%%%%%%%%%%%%%%%%%%%%%%%%%%%%%%%%%%%%%%%%%%%%%%%%%%%%%%%%%


%%%%%%%%%%%%%%%%%%%%%%%%%%%%%%%%%%%%%%%%%%%%%%%%%%%%%%%%%%%%%%%%%%%%%%
%% To display the labels used in a tex file in the dvi file (for example, if a theorem is labelled with the \label command) use the package
%% http://www.ctan.org/pkg/showkeys

%%%%%% Use Option 1
%%%%%% This will display all labels (for example, in the case of a labelled theorem, the label of the theorem will occur in the margin of the dvi file.
%%%%%% The command above by itself not only displays labels when they are first named but also when they are cited (or referenced).
%\usepackage{showkeys}

%%%%%% Use Option 2
%%%%%% It will NOT display citations and references. Only Labels
%\usepackage[notref, notcite]{showkeys}

%%%%%% Use Options 3
%%%%%% formats to smaller fonts the labels etc.
%\usepackage[usenames,dvipsnames]{color}
%\providecommand*\showkeyslabelformat[1]{{\normalfont \tiny#1}}
%\usepackage[notref,notcite,color]{showkeys}
%\definecolor{labelkey}{rgb}{0,0,1}
%%%%%%%%%%%%%%%%%%%%%%%%%%%%%%%%%%%%%%%%%%%%%%%%%%%%%%%%%%%%%%%%%%%%%%


%%%%%%%%%%%%%%%%%%%%%%%%%%%%%%%%%%%%%%%%%%%%%%%%%%%%%%%%%%%%%%%%%%%%%%%
% It extends the functionality of all the LATEX cross-referencing commands (including the table of contents, bibliographies etc) to produce \special commands which a driver can turn into hypertext links; it also provides new commands to allow the user to write ad hoc hypertext links, including those to external documents and URLs.
% http://www.tug.org/applications/hyperref/manual.html
\usepackage[colorlinks=true, pdfstartview=FitV, linkcolor=blue,
            citecolor=blue, urlcolor=blue]{hyperref}
\usepackage[usenames]{color}
%%%%%%%%%%%%%%%%%%%%%%%%%%%%%%%%%%%%%%%%%%%%%%%%%%%%%%%%%%%%%%%%%%%%%%%%
% custom textbox
\usepackage{tcolorbox}
\tcbuselibrary{breakable}
\tcbuselibrary{skins}
\tcbset{my left line/.style={
          enhanced, frame hidden, borderline west = {0.5pt}{0pt}{black}, % specify border
          opacityframe=0, opacityback=0,opacityfill=0, % color
          arc = 0mm, left skip=1em, % border
          left = 1mm,top=0mm,bottom=0mm,boxsep=1mm,middle=1mm, % text and border
}}

\newtcbox{\mybox}[1][]{my left line, #1}
\newtcolorbox{myleftlinebox}[1][breakable]{my left line, #1}
% usage
% use \mybox[on line]{your TEXT} for inline box otherwise forcing line breaks   
% for whole break, use \begin{myleftlinebox
%%%%%%%%%%%%%%%%%%%%%%%%%%%%%%%%%%%%%%%%%%%%%%%%%%%%%%%%%%%%%%%%%%%%%%%
\usepackage[utf8]{inputenc}


%%%%%%%%%%%%%%%%%%%%%%%%%%%%%%%%%%%%%%%%%%%%%%%%%%%%%%%%%%%%%%%%%%%%%%%
% a good looking way to format urls
% http://mirror.its.uidaho.edu/pub/tex-archive/help/Catalogue/entries/url.html
\usepackage{url}
% Define a new 'leo' style for the package that will use a smaller font.
\makeatletter\def\url@leostyle{%
 \@ifundefined{selectfont}{\def\UrlFont{\sf}}{\def\UrlFont{\scriptsize\ttfamily}}} \makeatother\urlstyle{leo}
%%%%%%%%%%%%%%%%%%%%%%%%%%%%%%%%%%%%%%%%%%%%%%%%%%%%%%%%%%%%%%%%%%%%%%%


%%%%%%%%%%%%%%%%%%%%%%%%%%%%%%%%%%%%%%%%%%%%%%%%%%%%%%%%%%%%%%%%%%%%%%%%%%%%
%%%%%% The present package defines the environment mdframed which automatically deals with page breaks in framed text.
%%%%%% http://mirrors.ibiblio.org/CTAN/macros/latex/contrib/mdframed/mdframed.pdf
\usepackage[framemethod=default]{mdframed}
%%%%%%%%%%%%%%%%%%%%%%%%%%%%%%%%%%%%%%%%%%%%%%%%%%%%%%%%%%%%%%%%%%%%%%%%%%%%

%%%%%%%%%%%%%%%%%%%%%%%%%%%%%%%%%%%%%%%%%%%%%%%%%%%%%%%%%%%%%%%%%%%%%%%%%%%%%%%%
%%%%%% An extended implementation of the array and tabular environments
%%%%%% which extends the options for column formats, and provides "programmable" format specifications.
%\usepackage{array}
%%%%%%%%%%%%%%%%%%%%%%%%%%%%%%%%%%%%%%%%%%%%%%%%%%%%%%%%%%%%%%%%%%%%%%%%%%%

%%%%%%%%%%%%%%%%%%%%%%%%%%%%%%%%%%%%%%%%%%%%%%%%%%%%%%%%%%%%%%%%%%%%%%%%%%%%
%%% The package enhances the quality of tables in LATEX, providing extra
%%% commands as well as behind-the-scenes optimization. Guidelines
%%% are given as to what constitutes a good table in this context.
%%% From version 1.61, the package offers long table compatibility.
%\usepackage{booktabs}
%%%%%%%%%%%%%%%%%%%%%%%%%%%%%%%%%%%%%%%%%%%%%%%%%%%%%%%%%%%%%%%%%%%%%%%%%%%%%%


\usepackage{tikz}




%% or use GEOMETRY package
\usepackage[margin=1.0in, letterpaper]{geometry}
%\def\baselinestretch{1.1}


%%%%%%%%%%%%% OR USE EXACT DIMENSIONS %
%\setlength{\textwidth}{6.5in}     %%
%\setlength{\oddsidemargin}{0in}   %%
%\setlength{\evensidemargin}{0in}  %%
%\setlength{\textheight}{8.5in}    %%
%\setlength{\topmargin}{0in}       %%
%\setlength{\headheight}{0in}      %%
%\setlength{\headsep}{.3in}         %%
%\setlength{\footskip}{.5in}       %%
%%%%%%%%%%%%%%%%%%%%%%%%%%%%%%%%%%%%%%%%%%%%%%%%%%%%%%%%%%%%%%%%%%%%%%%


%%%%%%%%%%%%%%%%%%%%%%%%%%%%%%%%%%%%%%%%%%%%%%%%%%%%%%%%%%%%%%%%%%%%%%%
%%%%%%%%%%%%%%%%%%%%%%%% NUMBERING %%%%%%%%%%%%%
\newtheorem{theorem}{Theorem}
\newtheorem{conjecture}{Conjecture}
\newtheorem{conclusion}{Conclusion}
\newtheorem{proposition}[theorem]{Proposition}
\newtheorem{lemma}[theorem]{Lemma}
\newtheorem{corollary}[theorem]{Corollary}
\newtheorem{assumption}{Assumption}
\newtheorem{condition}{C}
\theoremstyle{definition}
\newtheorem{definition}[theorem]{Definition}
\newtheorem{example}[theorem]{Example}
\theoremstyle{remark}
\newtheorem{remark}[theorem]{Remark}
\newtheorem{question}[theorem]{Question}
\newtheorem{problem}[theorem]{Problem}
\newtheorem{NB}[theorem]{Nota Bene}

\numberwithin{equation}{section}
\numberwithin{theorem}{section}
%\renewcommand{\labelitemi}{ {\small $\rhd$}}
%%%%%%%%%%%%%%%%%%%%%%%%%%%%%%%%%%%%%%%%%%%%%%%%%%%%%%%%%%%%%%%%%%%%%%%

%%%%%%%%%%%%%%%%%%%%%%%%%%%%%%%%%%%%%%%%%%%%%%%%%%%%%%%%%%%%%%%%%%%%%%%
\definecolor{Red}{rgb}{0.9,0,0.0}
\definecolor{Blue}{rgb}{0,0.0,1.0}
%%%%%%%%%%%%%%%%%%%%%%%%%%%%%%%%%%%%%
%%% used for editing and making comments in color
% Example \ig{Remarks and Commets}
\newcommand{\ig}[1]{\textcolor[rgb]{0.00, 0.0, 1.0}{{\tiny \textsuperscript{[\textrm{IC:Rem}]}} \ #1}}
\newcommand{\igAdd}[1]{\textcolor[rgb]{0.7, 0.0, 0.0}{{\tiny \textsuperscript{[\textrm{IC:Add}]}} \ #1}}
\newcommand{\igEdit}[1]{\textcolor[rgb]{0.7, 0.0, 0.0}{{\tiny \textsuperscript{[\textrm{IC:Edit}]}} \  #1}}

\newcommand{\trb}[1]{\begin{color}[rgb]{0.98, 0.0, 0.98}{TRB: #1} \end{color}}
\newcommand{\ti}[1]{\begin{color}[rgb]{1.00,0.00,0.00}{TI: #1} \end{color}}
\newcommand{\Red}[1]{\textcolor{Red}{#1}}
%%%%%%%%%%%%%%%%%%%%%%%%%%%%%%%%%%%%%

%%%%%%%%%%%%%%%%%%%%%%%%%%%%%%%%%%%%%
%%%     Igor's macros
%% \mathcal Letters
\def\cA{\mathcal{A}}
\def\cB{\mathcal{B}}
\def\cC{\mathcal{C}}
\def\cD{\mathcal{D}}
\def\cE{\mathcal{E}}
\def\cF{\mathcal{F}}
\def\cG{\mathcal{G}}
\def\cH{\mathcal{H}}
\def\cI{\mathcal{I}}
\def\cJ{\mathcal{J}}
\def\cK{\mathcal{K}}
\def\cL{\mathcal{L}}
\def\cM{\mathcal{M}}
\def\cN{\mathcal{N}}
\def\cO{\mathcal{O}}
\def\cP{\mathcal{P}}
\def\cQ{\mathcal{Q}}
\def\cR{\mathcal{R}}
\def\cS{\mathcal{S}}
\def\cT{\mathcal{T}}
\def\cU{\mathcal{U}}
\def\cV{\mathcal{V}}
\def\cW{\mathcal{W}}
\def\cX{\mathcal{X}}
\def\cY{\mathcal{Y}}
\def\cZ{\mathcal{Z}}

%% \mathbb Letters
\def\bA{\mathbb{A}}
\def\bB{\mathbb{B}}
\def\bC{\mathbb{C}}
\def\bD{\mathbb{D}}
\def\bE{\mathbb{E}}
\def\bF{\mathbb{F}}
\def\bG{\mathbb{G}}
\def\bH{\mathbb{H}}
\def\bI{\mathbb{I}}
\def\bJ{\mathbb{J}}
\def\bK{\mathbb{K}}
\def\bL{\mathbb{L}}
\def\bM{\mathbb{M}}
\def\bN{\mathbb{N}}
\def\bO{\mathbb{O}}
\def\bP{\mathbb{P}}
\def\bQ{\mathbb{Q}}
\def\bR{\mathbb{R}}
\def\bS{\mathbb{S}}
\def\bT{\mathbb{T}}
\def\bU{\mathbb{U}}
\def\bV{\mathbb{V}}
\def\bW{\mathbb{W}}
\def\bX{\mathbb{X}}
\def\bY{\mathbb{Y}}
\def\bZ{\mathbb{Z}}

%% \mathscr Letters, for filtration, sigma algebras etc
\def\sA{\mathscr{A}}
\def\sB{\mathscr{B}}
\def\sC{\mathscr{C}}
\def\sD{\mathscr{D}}
\def\sE{\mathscr{E}}
\def\sF{\mathscr{F}}
\def\sG{\mathscr{G}}
\def\sH{\mathscr{H}}
\def\sI{\mathscr{I}}
\def\sJ{\mathscr{J}}
\def\sK{\mathscr{K}}
\def\sL{\mathscr{L}}
\def\sM{\mathscr{M}}
\def\sN{\mathscr{N}}
\def\sO{\mathscr{O}}
\def\sP{\mathscr{P}}
\def\sQ{\mathscr{Q}}
\def\sR{\mathscr{R}}
\def\sS{\mathscr{S}}
\def\sT{\mathscr{T}}
\def\sU{\mathscr{U}}
\def\sV{\mathscr{V}}
\def\sW{\mathscr{W}}
\def\sX{\mathscr{X}}
\def\sY{\mathscr{Y}}
\def\sZ{\mathscr{Z}}


%%%% \mathsf for Matrices
\def\mA{\mathsf{A}}
\def\mB{\mathsf{B}}
\def\mC{\mathsf{C}}
\def\mD{\mathsf{D}}
\def\mE{\mathsf{E}}
\def\mF{\mathsf{F}}
\def\mG{\mathsf{G}}
\def\mH{\mathsf{H}}
\def\mI{\mathsf{I}}
\def\mJ{\mathsf{J}}
\def\mK{\mathsf{K}}
\def\mL{\mathsf{L}}
\def\mM{\mathsf{M}}
\def\mN{\mathsf{N}}
\def\mO{\mathsf{O}}
\def\mP{\mathsf{P}}
\def\mQ{\mathsf{Q}}
\def\mR{\mathsf{R}}
\def\mS{\mathsf{S}}
\def\mT{\mathsf{T}}
\def\mU{\mathsf{U}}
\def\mV{\mathsf{V}}
\def\mW{\mathsf{W}}
\def\mX{\mathsf{X}}
\def\mY{\mathsf{Y}}
\def\mZ{\mathsf{Z}}


%%%% \mathbf for Matrices or vectors
\def\bfB{\boldsymbol{B}}
\def\bfC{\boldsymbol{C}}
\def\bfD{\boldsymbol{D}}
\def\bfA{\boldsymbol{A}}
\def\bfE{\boldsymbol{E}}
\def\bfF{\boldsymbol{F}}
\def\bfG{\boldsymbol{G}}
\def\bfH{\boldsymbol{H}}
\def\bfI{\boldsymbol{I}}
\def\bfJ{\boldsymbol{J}}
\def\bfK{\boldsymbol{K}}
\def\bfL{\boldsymbol{L}}
\def\bfM{\boldsymbol{M}}
\def\bfN{\boldsymbol{N}}
\def\bfO{\boldsymbol{O}}
\def\bfP{\boldsymbol{P}}
\def\bfQ{\boldsymbol{Q}}
\def\bfR{\boldsymbol{R}}
\def\bfS{\boldsymbol{S}}
\def\bfT{\boldsymbol{T}}
\def\bfU{\boldsymbol{U}}
\def\bfV{\boldsymbol{V}}
\def\bfW{\boldsymbol{W}}
\def\bfX{\boldsymbol{X}}
\def\bfY{\boldsymbol{Y}}
\def\bfZ{\boldsymbol{Z}}

%%%% \mathbf for Matrices
\def\bfa{\boldsymbol{a}}
\def\bfb{\boldsymbol{b}}
\def\bfc{\boldsymbol{c}}
\def\bfd{\boldsymbol{d}}
\def\bfe{\boldsymbol{e}}
\def\bff{\boldsymbol{f}}
\def\bfg{\boldsymbol{g}}
\def\bfh{\boldsymbol{h}}
\def\bfi{\boldsymbol{i}}
\def\bfj{\boldsymbol{j}}
\def\bfk{\boldsymbol{k}}
\def\bfl{\boldsymbol{l}}
\def\bfm{\boldsymbol{m}}
\def\bfn{\boldsymbol{n}}
\def\bfo{\boldsymbol{o}}
\def\bfp{\boldsymbol{p}}
\def\bfq{\boldsymbol{q}}
\def\bfr{\boldsymbol{r}}
\def\bfs{\boldsymbol{s}}
\def\bft{\boldsymbol{t}}
\def\bfu{\boldsymbol{u}}
\def\bfv{\boldsymbol{v}}
\def\bfw{\boldsymbol{w}}
\def\bfx{\boldsymbol{x}}
\def\bfy{\boldsymbol{y}}
\def\bfz{\boldsymbol{z}}

%%%% \boldsymbol for lower case greek letter in math mode
\def\bfalpha{\boldsymbol{\alpha}}
\def\bfbeta{\boldsymbol{\beta}}
\def\bfgamma{\boldsymbol{\gamma}}
\def\bfdelta{\boldsymbol{\delta}}
\def\bfepsilon{\boldsymbol{\epsilon}}
\def\bfzeta{\boldsymbol{\zeta}}
\def\bfeta{\boldsymbol{\eta}}
\def\bftheta{\boldsymbol{\theta}}
\def\bfiota{\boldsymbol{\iota}}
\def\bfkappa{\boldsymbol{\kappa}}
\def\bflambda{\boldsymbol{\lambda}}
\def\bfmu{\boldsymbol{\mu}}
\def\bfnu{\boldsymbol{\nu}}
\def\bfomicron{\boldsymbol{\omicron}}
\def\bfpi{\boldsymbol{\pi}}
\def\bfrho{\boldsymbol{\rho}}
\def\bfsigma{\boldsymbol{\sigma}}
\def\bftau{\boldsymbol{\tau}}
\def\bfupsilon{\boldsymbol{\upsilon}}
\def\bfphi{\boldsymbol{\phi}}
\def\bfchi{\boldsymbol{\chi}}
\def\bfpsi{\boldsymbol{\psi}}
\def\bfomega{\boldsymbol{\omega}}

%%%% \boldsymbol for upper case greek letter in math mode
\def\bfPhi{\boldsymbol{\Phi}}
\def\bfTheta{\boldsymbol{\Theta}}
\def\bfPsi{\boldsymbol{\Psi}}
\def\bfOmega{\boldsymbol{\Omega}}
\def\bfSigma{\boldsymbol{\Sigma}}


%%%%%%%%%%%%%%%%%% Shortcuts %%%%%%%%%%%%%%%%%%%%%%%%%%%%%%%%%%%%%%%%%%%
\newcommand{\wt}{\widetilde}
\newcommand{\wh}{\widehat}

%%%%%%%%%%%%%%%%%%%%%%%%%%%%%%%%%%%%%%%%%%%%%%%%%%%%%%%%%%%%%%%%%%%%%%%%%%%%%%%%%
%%%%%%%%%%%%%%%%%%   Nonstandard notations  %%%%%%%%%%%%%%%%%%%%%%%%%%%%%%%%%%%%%
\newcommand{\pd}[1]{\partial_{#1}}      % partial derivative
\newcommand{\1}{\mathbbm{1}}            % preferable way of writing indicator function
\newcommand{\set}[1]{\{#1\}}            % set: {xyz} to be used for inline formulas
\newcommand{\Set}[1]{\left\{#1\right\}} % set: {xyz} to be used for separate (not inline) formulas
\renewcommand{\mid}{\,|\,}              % mid bar with small spaces before and after: x | y
\newcommand{\Mid}{\,\Big | \,}          % big bar with small spaces before and after:
\newcommand{\norm}[1]{ \| #1 \| }       % mid bar with small spaces before and after: x | y
\newcommand{\abs}[1]{\left\vert#1\right\vert}   % absolute value
\newcommand{\CB}[1]{\left\{ #1 \right\}}
\newcommand{\SB}[1]{\left[ #1 \right]}
\newcommand{\Pare}[1]{\left( #1 \right)}
\newcommand{\AB}[1]{\left \langle #1 \right \rangle}
\newcommand{\given}[1]{\left.#1\right|}
\newcommand{\givenAlt}[1]{\left.#1\right]}
\newcommand{\Tran}[1]{{#1}^\top}
\newcommand{\bsde}{BS$\Delta$E}         % BS\DeltaE
\newcommand{\bsdes}{BS$\Delta$Es}       % BS\DeltaE
\newcommand{\ow}{\text{otherwise}}
\newcommand{\imblies}{\Longleftarrow}
\newcommand{\Exp}[1]{\mathrm{E}\left[ #1 \right]}
\newcommand{\Var}[1]{\mathrm{Var}\left[ #1 \right]}
\newcommand{\Cov}[1]{\mathrm{Cov}\left[ #1 \right]}
\newcommand{\Bspace}{\;\;\;\;}
\newcommand{\dif}{\,\mathrm{d}}        % used for differential, same as in commath.sty

\newcommand{\LHS}{\text{LHS}} % left hand side
\newcommand{\RHS}{\text{RHS}} % right hand side
\newcommand{\Adj}{\text{Adj}} %adjoint matrix

\newcommand{\RNum}[1]{\uppercase\expandafter{\romannumeral #1\relax}} % romanian numerals

\DeclareMathOperator{\logit}{logit}
\DeclareMathOperator*{\esssup}{ess\,sup} % ess sup
\DeclareMathOperator*{\essinf}{ess\,inf} % ess inf
\DeclareMathOperator*{\esslimsup}{ess\,\limsup}
\DeclareMathOperator*{\essliminf}{ess\,\liminf}
\DeclareMathOperator*{\argmin}{arg\,min} % argmin
\DeclareMathOperator*{\diag}{diag} % diag
\DeclareMathOperator*{\argmax}{arg\,max} % argmax
\DeclareMathOperator*{\Arg}{Arg} % arguments
\DeclareMathOperator*{\rank}{rank\,} % argmax
\DeclareMathOperator*{\KL}{KL} % KL divergence
\DeclareMathOperator*{\Proj}{Proj} % Projection
\DeclareMathOperator{\Std}{\mathrm{Std}} % \std for Standard deviation
\DeclareMathOperator{\sgn}{\mathrm{sgn}} % sign of a variable
\DeclareMathOperator{\tr}{\mathrm{trace}} % matrix trace
% trigonometric and hyperbolic functions
\DeclareMathOperator{\sech}{sech}
\DeclareMathOperator{\csch}{csch}
\DeclareMathOperator{\arcsec}{arcsec}
\DeclareMathOperator{\arccot}{arcCot}
\DeclareMathOperator{\arccsc}{arcCsc}
\DeclareMathOperator{\arccosh}{arcCosh}
\DeclareMathOperator{\arcsinh}{arcsinh}
\DeclareMathOperator{\arctanh}{arctanh}
\DeclareMathOperator{\arcsech}{arcsech}
\DeclareMathOperator{\arccsch}{arcCsch}
\DeclareMathOperator{\arccoth}{arcCoth} 

%%%%%%%%%%%%%%%%%%%%%%%%%%%%%%% FINANCE %%%%%%%%%%%%%%%%%%%%%%%%%%%%%%%%%%%%%%%%%%%%%%%%%%
\DeclareMathOperator{\var}{\mathrm{V}@\mathrm{R}}           % \V@R Value-at-risk
\DeclareMathOperator{\tce}{\mathrm{TCE}}                    % Tail Conditional Expectation
\DeclareMathOperator{\tvar}{\mathrm{TV}@\mathrm{R}}         % \TV@R tail Value-at-risk
\DeclareMathOperator{\avar}{\mathrm{AV}@\mathrm{R}}         % \AV@R average Value-at-risk
\DeclareMathOperator{\ent}{\mathrm{Ent}}                    % \ent = Entropic Risk Measure
\DeclareMathOperator{\glr}{\mathrm{GLR}}                    % \glr = gain to loss ratio

\DeclareMathOperator{\ES}{\mathrm{ES}}

\newcommand{\ask}{\text{ask}}           % ask price
\newcommand{\bid}{\text{bid}}           % bid price

%%%%%%%%%%%%%%%%%%%%%%%%%%%%%%%%%%%%%%%%%%%%%%%%%%%%%%%
% \bibliographystyle{amsplain}
% \bibliographystyle{alpha} % standard LaTeX bibliography format. Preferable to be used
% \bibliography{MathFinanceMaster-12-28-2014}
%\bibliography{D:/_research/latex/lib_igor/igor_bib_mathfinance}
% help on how to use several bit files \bibliography{videogames,comics,interface,theory}
%%%%%%%%%%%%%%%%%%%%%%%%%%%%%%%%%%%%%%%%%%%%%%%%%%%%%%%




\title{Gleanings on Engineering Analysis I}
\author{Yuanxing Cheng, A20453410, MMAE-501-f22}


\begin{document}

\maketitle

\section{On Diagonalization of Matrices}

\(Q\) diagonalizes \(A\) is \(Q^\top A Q =D\) is a diagonal matrix. \(Q\) is the called {\it modal matrix}. A general procedure for real symmetric matrix \(A_{N\times N}\).

\begin{enumerate}
    \item notice for real symmetric matrix, it has \(N\) different eigenvalues, here we obtain them and name them as \(\lambda_i\)
    \item get normalized eigenvectors \(q_i\)
    \item construct orthogonal modal matrix \(Q=[q_1,\dots,q_N]\)
    \item then \(D = \diag(\lambda_1,\dots,\lambda_N)\)
\end{enumerate}

Then, we write \(x=Qy\) and then write the {\it canonical form} of the quadratic form.

\[x^\top A x=y^\top Q^\top AQ y = \sum_N \lambda_i y_i^2  \]

Then clearly, \(Q\) and \(A\) share the same eigenvalues ({\it spectral theorem}), and then we call \(D = Q^\top A Q\) the similarity transformation.


\begin{theorem}[Spectral theorem in finite space]
    for normal matrix, it's diagonalizable using its orthogonal model matrix. The result share the same eigenvalues and they are similar.
\end{theorem}

\begin{remark}
    For real and symmetric matrix \(\bfA\) with distinct eigenvalues, the eigenvectors are mutually orthogonal and we have decomposition: \(\bfA = \bfQ \bfD \bfQ^\top\) where \(\bfQ\) is the modal matrix comprised of the orthonormal eigenvectors of \(\bfA\) and \(\bfD\) is a diagonal matrix having the eigenvalues of \(\bfA\) along its diagonal; for nonsymmetric matrix with distinct eigenvalues, the eigenvectors are mutually linearly independent, but not necessarily orthogonal. So we can have the decomposition: \(\bfA = \bfU \bfD \bfU^{-1}\) and \(\bfU\) is the modal matrix containing the linearly-independent eigenvectors of \(\bfA\). For nonsymmetric matrix with repeated eigenvalues, it ends up {\it Jordan canonical form} and we write \(\bfA = \bfU\bfJ\bfU^{-1}\) where \(\bfJ\) is the nearly-diagonal Jordan canonical form.
\end{remark}



\section{On solving DEs}

\subsection{System of N coupled first-order linear ordinary differential equations}

Consider

\begin{equation*}
    \begin{split}
        \dot u_1(t) &= A_{11} u_1(t) + A_{12} u_2(t) + \cdots+ A_{1N} u_N(t) + f_1(t)\\
        \dot u_2(t) &= A_{21} u_1(t) + A_{22} u_2(t) + \cdots+ A_{2N} u_N(t) + f_2(t)\\
         & \vdots\\
        \dot u_N(t) &= A_{N1} u_1(t) + A_{N2} u_2(t) + \cdots+ A_{NN} u_N(t) + f_N(t)\\
    \end{split}
\end{equation*}

In matrix form, we write \(\dot \bfu(t) = \bfA \bfu(t)+\bff(t)\). If \(\bff\equiv 0\) then the system is homogeneous.

To solve this, we first diagonalize \(\bfA\).

\begin{enumerate}
    \item Diagonalize \(\bfA\) and obtain \(D = U^{-1}\bfA U\)
    \item Consider \(\dot \bfu(t) = U \dot \bfv(t)\)
    \item then we have \(\dot\bfv(t) = U^{-1}AU\bfv(t) + U^{-1}\bff(t)=D\bfv(t)+U^{-1}\bff(t)\)
    \item we then obtain the uncoupled system and the homogeneous system will have an obvious soution: \(v_i (t)= c_i e^{\lambda_i t}\) where \(\lambda_i\) is the \(i\)-th element of the diagonal of matrix \(D\).
    \item For nonhomogeneous system, the particular solution can be found using method of undetermined coefficients.
    \item Finally, \(\bfu(t) = U\bfv(t)\) is the solution we want.
    
\end{enumerate}

\subsection{N-th order Linear ODE}

Consider

\[u^{(N)}=F(t,u,\dot u,\dots,u^{(N-1)})\]

We write it into a system of N first-order differential equations by substitutions \(u_i(t) = u^{(i-1)}(t)\).

\begin{remark}
    For linear equations, both the real and imaginary parts of complex solutions are by themselves solutions of the differential equations, and a linear combination of the real and imaginary parts, is also a solution of the linear equations. The real and imagainary parts is obtained using {\it Euler's formula}: \(e^{\theta it} =\cos(\theta t)+i\sin(\theta t)\)
\end{remark}

\begin{remark}
    Above we are only discussing {\it autonomous} system meaning \(\bfA\) does not depend on time.
\end{remark}

\section{Decompositions}

\subsection{SVD}
For matrix \(A\) of size \(m\times n\) we want to decomposite it to \(A = U\Sigma V^\top\) where \(U\) is \(m\times m\) orthogonal matrix, \(V\) is \(n\times n\) orthogonal matrix, \(\Sigma\) is an \(m \times n\) diagonal matrix. In the following, \(u_i\) and \(v_i\) are column vectors of \(u\) and \(V\). 

\begin{enumerate}
    \item \(AA^\top u_i = \sigma_i^2 u_i = \sigma_i A v_i\), solving left singular vectors of \(A\), and singular values \(\sigma_i\). We also want \(\sigma_1\geq\sigma_2\cdots\sigma_p\geq 0\) where \(p=\min (m,n)\).
    \item Alternatively, we can solve right singular vectors from \(A^\top A v_i = \sigma_i^2 v_i = \sigma_i A^\top u_i\)
    \item obtain full \(U\) and \(V\) and orthonormalizing them.
    \item fill \(\Sigma\) left columns in \(0\).
    \item The final result can be also expressed in the form: \(A = \sum_{i=1}^p \sigma_i u_i v_i^\top\).
\end{enumerate}

\begin{remark}
    in image compression, we might use \(k<p\).
\end{remark}

\subsection{Polar decomposition}

Following SVD, we might futher write \(A = U\Sigma V^\top = UV^\top V\Sigma V^\top\) and break this up into \(A = WR\) where \(W = UV^\top\) and \(R = V\Sigma V^\top\).

Notice \(U\) and \(V\) are both orthogonal so that \(W\) is also orthogonal; and \(R\) is symmetric since \(\Sigma\) is symmetric.

\begin{enumerate}
    \item Notice \(A^\top A = R^\top R = R^2\), we first solve \(R\).
    \item Notice we can diagonalize \(R^2\) and \(R\): \(Q^\top R^2 Q = \diag(\lambda_1^2,\cdots,\lambda_N^2)\), \(Q^\top R Q = \diag(\lambda_1,\cdots,\lambda_N)\)
    \item then \(R = Q\diag(\lambda_1,\cdots,\lambda_N) Q^\top\)
    \item \(W = AR^{-1}\)
\end{enumerate}

\subsection{QR}

Write \(A = [u_1,u_2,\dots, u_S]\) and we want \(A = QR\) where \(Q\) is orthonormal matrix \([q_1,q_2,\dots, q_S]\) and \(R\) is right triangular matrix. Following Gram-schmidt, we are transforming \(u_i\) to \(q_i\). For \(i=2,3,\dots, S\),

\[\hat u_i = u_i - \sum_{k=1}^{i-1} \AB{u_i,q_k} q_k\]

then we normalize them \(q_i = \frac{\hat u_i}{\norm{\hat u_i}}\). And this GS orthogonalization is actually  a matrix decomposition for \(A\) because we actually have \(\hat u_i = \norm{\hat u_i} q_i\), \(\norm{\hat u_i}=\AB{\hat u_i,q_i}\), and 

\[u_i = \norm{\hat u_i} q_i + \sum_{k=1}^{i-1} \AB{u_i,q_k}q_k\]

Above in a nicer way:

\begin{equation*}
    \begin{split}
        i=1& : u_1 =\AB{u_1,q_1} q_1,\\
        i=2& : u_2 =\AB{u_2,q_1}q_1,+\AB{u_2,q_2}q_2,\\
        \vdots &
    \end{split}
\end{equation*}

or in matrix form: \(A=QR\) where \(Q=[q_1,q_2,\dots,q_S ]\) and 

\[R=\begin{bmatrix}
    \AB{u_1,q_1} & \AB{u_2,q_1} & \AB{u_3,q_1} & \cdots &\AB{u_S,q_1}\\
    0            & \AB{u_2,q_2} & \AB{u_3,q_2} & \cdots &\AB{u_S,q_2}\\
    \vdots & &&& \AB{u_S,q_S}
\end{bmatrix}\]

\subsection{LU decomposition}

\subsection{CHolesky decomposition}

\section{Differential Eigenproblems}

\subsection{Function Spaces, bases, and orthogonalization}

\begin{definition}
    A set of functions \(u_i(x)\) for \(i=1,2,\dots,M\) are {\it linearly independent} over an interval \(a\leq x\leq b\) is their linear combination is zero is all coefficients are zero \(\forall x\in [a,b]\).
\end{definition}

\begin{remark}
    For any piecewise continuous function \(u(x)\) we have expression:
    \[u(x)=\sum_{i=1}^\infty a_n u_n(x)\] where \(a_n\in\bR \) and \(a_n\) are not all zeros, \(u_n(x)\) are linear-independent {\it basis functions}. When there are actually infinite sum, we say \(u(x)\) is {\it infinite dimensional}.
\end{remark}

\begin{definition}
    We also want basis functions mutually orthogonal, so here we define the inner product of two functions.
    \[
        \AB{u_m(x),u_n(x)}=\int_a^b w(x)u_m(x)u_n(x)\dif x
    \]
    orthogonality means  \(\AB{u_m(x),u_n(x)}\equiv 0\forall x\in [a,b]\), and \(w(x)\) is a weight function. \(w(x)\equiv 1\) unless stated otherwise.
\end{definition}

\begin{definition}[Legendre polynomials]
    Obtained from Gram-schmidt process on \(u_i(x)=x^i\) for \(i=0,1,\dots\) and \(-1\leq x\leq 1\), the first several terms are \(\tilde u_i(x) = 1,x,\frac{1}{2}(3x^2-1),\dots\). Then any piecewise continuous function \(f(x)\) in the interval \(-1\leq x\leq 1\) can be written as 
    \[f(x) = \sum_{i=1}^\infty \AB{f(x),\tilde{u}_i(x)} \tilde{u}_i(x)\]
\end{definition}

\begin{remark}
    When considering bases \(\cos(ix),\sin(ix)\) for \(i=0,1,\dots\) and \(0\leq x\leq 2\pi\), we end up with {\it Fourier series},
    \begin{equation*}
        \begin{split}
            f(x) &= \frac{a_0}{\sqrt{2\pi}}+\sum_{i=1}^\infty a_i \frac{\cos(ix)}{\sqrt{\pi}}+\sum_{i=1}^\infty b_i \frac{\sin(ix)}{\sqrt{\pi}}\\
            a_0 &= \AB{f(x),\frac{1}{\sqrt{2\pi}}}=\frac{1}{\sqrt{2\pi}} \int_0^{2\pi} f(x)\dif x\\
            a_i &= \AB{f(x),\frac{\cos(ix)}{\sqrt{\pi}}}=\frac{1}{\sqrt{\pi}} \int_0^{2\pi} f(x)\cos(ix)\dif x\\
            b_i &= \AB{f(x),\frac{\sin(ix)}{\sqrt{\pi}}}=\frac{1}{\sqrt{\pi}} \int_0^{2\pi} f(x)\sin(ix)\dif x
        \end{split}
    \end{equation*}
\end{remark}

\begin{definition}[Eigenproblems]
    Consider \(n\)-th order differential equation \(\cL u(x)=f(x)\) where 
    \[\cL = a_0+a_1\frac{\dif}{\dif x} + a_2 \frac{\dif^2}{\dif x^2}+\cdots+a_n \frac{\dif^n}{\dif x^n}\]
    We then tend to find \(\lambda_i\) so that  \(\cL u_i(x) = \lambda_i u_i(x)\). We call the nontrivial solution \(u_i(x)\) the eigenvalue of \(\cL\) and \(u_i(x)\) the {\it eigenfunction} of \(\cL\).
\end{definition}

Given that the differential operator \(\cL\) is self-adjoint (defined below) and boundary conditions,

\begin{enumerate}
    \item Solve \(\cL u_i(x) = \lambda_i u_i(x)\) as a \(N\)-th order Linear order ODE,  excluding the trivial solution.
    \item Usually we end up in a characteristic equation with roots being the eigenvalues
    \item And when solving for coefficients of general solutions, make sure the norm of eigenfunctions being \(1\), resulting in normalized eigenfunctions.
    \item write \(f(x) = \sum_{i=1}^\infty \hat a_i u_i(x)\) where \(\hat a_i=\AB{f(x),u_i(x)}\)
    \item write \(u(x) = \sum_{i=1}^\infty  a_i u_i(x)\) where \(a_i = \frac{\hat a_i}{\lambda_i}\), so that \(\cL u(x)=f(x)\)
    \item The final solution can be written as 
    \[u(x) = \sum_{i=1}^\infty \frac{\AB{f(x),u_i(x)}}{\lambda_i} u_i(x)\]
\end{enumerate}

\section{Adjointness}

\begin{definition}[from matrix transpose to adjoint operator]
    Consider matrix \(A\) of size \(n\times n\), arbitrary vectors \(u\) and \(v\) or size \(n\times 1\). We define \(A^\top\) the matrix such that \(\AB{v,Au}=\AB{A^\top v, u}\). And in real case, \(\AB{v,Au}=\AB{u, A^\top v}\). Likewise, for operator \(\cL\) we define its adjoint \(\cL^*\) such that 
    \[\AB{v,\cL u} = \AB{u,\cL^* v}\]
    where \(u(x)\) and \(v(x)\) are arbitrary functions with homogeneous boundary conditions.
\end{definition}

\begin{remark}
    An easy way to find an operator's adjoint is to use test function, then do integration by parts.
\end{remark}

\begin{example}[Second order linear differential equation with variable coefficients]
    \[\cL u =\frac{1}{w(x)}\Pare{a_0(x)u''(x)+a_1(x)u'(x)+a_2(x)u(x)}=0\]
    where \(x\in [a,b]\) and \(w(x)\) is a weight function.
    \begin{enumerate}
        \item Consider test function \(v(x)\) and find
        \begin{equation*}
            \begin{split}
                \AB{v,\cL u}&=\int_a^b w(x)v(x) \Pare{\frac{1}{w(x)}\Pare{a_0(x)u''(x)+a_1(x)u'(x)+a_2(x)u(x)}}\dif x\\
                &= \int_a^b \Pare{a_0vu''+a_1 vu'+a_2vu}\dif x
            \end{split}
        \end{equation*}
        \item using integration by parts to down-order \(u\): \(\int p\dif q=pq-\int q\dif p\), we obtain the following:
        \begin{itemize}
            \item \(\int_a^b a_0vu''\dif x= \int_a^b a_0v\dif u'= \given{a_0vu'}_a^b -\int_a^b u'(a_0v)'\dif x =\given{a_0vu'}_a^b -\given{(a_0v)'u}_a^b +\int_a^b u(a_0v)''\dif x \)
            \item \(\int_a^b a_1 vu'\dif x = \int_a^b a_1 v\dif u=\given{a_1vu}_a^b-\int_A^b u(a_1v)'\dif x\)
            \item \(a_2(x)\) term with \(u(x)\) so no need to do integration by parts
        \end{itemize}
        \item combine these we end up with
        \[\AB{v,\cL u}=\Pare{a_0 vu'-(a_0v)'u+a_1vu}_a^b +\int_a^b w(x)u(x)\Pare{\frac{1}{w(x)}\Pare{(a_0v)''-(a_1v)'+a_2v}}\dif x \]
        \item above should equal to \(\AB{u,\cL^* u}\) so given homogeneous BC, \(\cL^*v=\frac{1}{w(x)}\Pare{(a_0v)''-(a_1v)'+a_2v}\)
    \end{enumerate}
\end{example}
\begin{remark}
    variable coefficients move inside the derivatives, and the odd-order derivatives change sign as compared to \(\cL\).
\end{remark}

\begin{definition}
    A differential operator \(\cL\) is {\it self-adjoint}, or Hermitian, if \(\cL=\cL^*\).
\end{definition}
\begin{remark}
    Similar to symmetric matrix, its eigenvectors are mutually orthogonal if its eigenvalues are distinct, self-adjoint operator's eigenfunctions are mutually orthogonal if its eigenvalues are distinct.
\end{remark}

\subsection{Sturm-Liouville differential operator}

For differential equations in the previous example, use product rule we have

\[\cL^* v=\frac{1}{w(x)}\Pare{(a_0v)''-(a_1v)'+a_2v}=\frac{1}{w(x)}\Pare{a_0v''+(2a_0'-a_1)v'+(a_0''-a_1'+a_2)v}\]

Letting \(\cL=\cL^*\), we want
\begin{align*}
    a_1 &= 2a_0'-a_1 \implies a_1 = a_0'\\
    a_2 &= a_0''-a_1'+a_2
\end{align*}

So a self-adjoint differential operator may have the form:

\begin{align*}
    \cL u &= \frac{1}{w(x)} \Pare{a_0u''+a_0'u'+a_2u}\\
    \cL &= \frac{1}{w(x)} \Pare{\frac{\dif}{\dif x}\Pare{p(x)\frac{\dif}{\dif x}}+q(x)}
\end{align*}

where \(p(x)>0\) and \(w(x)>0\) in \(a\leq x\leq b\) and the boundary conditions are homogeneous.
\begin{remark}
    Similarly, we have the forth-order Sturm-Liouville differential operator
    \[\cL = \frac{1}{w(x)}\Pare{\frac{\dif^2}{\dif x^2}\Pare{s(x)\frac{\dif^2}{\dif x^2}} +\frac{\dif}{\dif x}\Pare{p(x)\frac{\dif}{\dif x}}+q(x) }\]
\end{remark}

\begin{remark}[Eigenfunctions of Sturm-Liouville Operators]
    Still we assume homogeneous boundary conditions, and consider the eigenproblrm \(w(x)\cL u_i(x)=-\lambda_i w(x) u_i(x)\). We have
    \begin{itemize}
        \item the eigenvalues are distinct and nonnegative
        \item the eigenfunctions \(u_i(x)\) are orthogonal wrt the weight function \(w(x)\).
    \end{itemize}
\end{remark}

\begin{remark}
    some special cases
    \begin{itemize}
        \item taking \(p\equiv 1\) and \(q\equiv 0\) on \(x\in[0,1]\)  produces eigenfunctions. Either fourier sine series or fourier cosine series, for different set of boundary conditions.
        \item taking \(p(x)=x\), \(q(x)=\frac{-\nu^2}{x}\), \(w(x)=x\) on \(x\in[0,1]\) produces Bessel functions, with \(\lambda_i=\mu_i^2\) 
        \item taking \(p(x)=1-x^2\), \(q(x)=0\), \(w(x)=1\) on \(x\in[-1,1]\) produces Legendre Polynomials, with \(\lambda_i=\nu(\nu+1)\)
        \item Legendre polynomials have recursion relation: 
        \[(\nu+1)P_{\nu+1}(x)-(2\nu+1)xP_\nu(x)+\nu P_{\nu-1}(x)=0\]
        \item taking \(p(x)=\sqrt{1-x^2}\), \(q(x)=0\), \(w(x)=\frac{1}{\sqrt{1-x^2}}\) on \(x\in[-1,1]\) produces Chebyshev polynomials, with \(\lambda_i=\nu^2\)
        \item Chebyshev polynomials have recursion relation:
        \[T_{\nu+1}(x)-2xT_\nu(x)+T_{\nu-1}(x)=0\]
    \end{itemize}
\end{remark}

\begin{remark}
    for nonhomogeneous boundary conditions, we can apply certain transformation.
\end{remark}

\section{Separation of Variables}

A new way to solve PDEs, by assume solution \(u(x,t)=\phi(x)\psi(t)\).

\subsection{General hyperbolic partial differential equation}

Consider \(\cM \ddot u+\cK u=0\) where \(\cM\) and \(\cK\) are linear differential operators in space, and dots denote differential in time.

Substituting into the governing equation gives

\[\ddot \psi \cM \phi + \psi\cK \phi=0\implies \frac{\cK \phi}{\cM \phi}=-\frac{\ddot\psi}{\psi}=\lambda\]

it's a constant since LHS only on space and RHS only on time.

Then we solve uncoupled differential equations

\begin{align*}
    \cK \phi(x) &=\lambda\cM \phi(x)\\
    \ddot\psi(t) &= -\lambda \psi(t)
\end{align*}
\subsection{Electrmagnetics}
\subsection{Schr\"odinger Equation}



\end{document}