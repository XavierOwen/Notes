\documentclass{article}

%%%%%%%%%%%%%%%%% this version: 03-14-2017 %%%%%%%%%%%%%%%%%%%%%%%%%%%%%%%

%%% standard packages
\usepackage{amsthm, amsmath, amssymb, amsfonts, graphicx, epsfig}
%\graphicspath{ {D:/Notes/others/assets/} }

\usepackage{algorithm, algpseudocode}

\allowdisplaybreaks
\usepackage{setspace}
%\usepackage{accents}
%%%
%\usepackage{kbordermatrix}

%%%%%%%%%%%%%%%%%%%%%%% ADDITIONAL FONTS %%%%%%%%%%%%%%%%%%%%%%%%%%%%%%
%%% Package to make \mathbbm, in particular to have 1 as for indicator function
\usepackage{bbm}
%%% Package to make special curl fonts, by using \mathscr{F}
\usepackage{mathrsfs}
%%% dsfont sometimes used instead of mathbb or mathbbm
%\usepackage{dsfont}
%%%%%%%%%%%%%%%%%%%%%%%%%%%%%%%%%%%%%%%%%%%%%%%%%%%%%%%%%%%%%%%%%%%%%%%


%%%%%%%%%%%%%%%%%%%%%%%%%%%%%%%%%%%%%%%%%%%%%%%%%%%%%%%%%%%%%%%%%%%%%%
%%% The ulem package provides various types of underlining that can
%%% stretch between words and be broken across lines. Convenient for editing. \sout{xxxx}
%%% http://ctan.unixbrain.com/macros/latex/contrib/ulem/ulem.pdf
%%%% Remark: if used with natbib, then the bibliography comes underlined. Comment this package at last compilation
%\usepackage{ulem}
%%%%%%%%%%%%%%%%%%%%%%%%%%%%%%%%%%%%%%%%%%%%%%%%%%%%%%%%%%%%%%%%%%%%%%

%%%%%%%%%%%%%%%%%%%%%%%%%%%%%%%%%%%%%%%%%%%%%%%%%%%%%%%%%%%%%%%%%%%%%%
%% Cancel is used to cross out in math mode. \xcancel, \cancel Convenient for editting.
%% http://ctan.math.utah.edu/ctan/tex-archive/macros/latex/contrib/cancel/cancel.pdf
%\usepackage{cancel}
%%%%%%%%%%%%%%%%%%%%%%%%%%%%%%%%%%%%%%%%%%%%%%%%%%%%%%%%%%%%%%%%%%%%%%


%%%%%%%%%%%%%%%%%%%%%%%%%%%%%%%%%%%%%%%%%%%%%%%%%%%%%%%%%%%%%%%%%%%%%%%
%This packages adds support of handling eps images to package graphics
%or graphicx with option pdftex. If an eps image is detected, epstopdf is
%automatically called to convert it to pdf format.
\usepackage{epstopdf}
%%%%%%%%%%%%%%%%%%%%%%%%%%%%%%%%%%%%%%%%%%%%%%%%%%%%%%%%%%%%%%%%%%%%%%%


%%%%%%%%%%%%%%%%%%%%%%%%%%%%%%%%%%%%%%%%%%%%%%%%%%%%%%%%%%%%%%%%%%%%%%%
% various features for using graphics, including subfigure, captions, subcaptions etc.
\usepackage{graphicx}
%\usepackage{caption}
\usepackage[font=sl,labelfont=bf]{caption}
\usepackage{subcaption}
%%%%%%%%%%%%%%%%%%%%%%%%%%%%%%%%%%%%%%%%%%%%%%%%%%%%%%%%%%%%%%%%%%%%%%%


%%%%%%%%%%%%%%%%%%%%%%%%%%%%%%%%%%%%%%%%%%%%%%%%%%%%%%%%%%%%%%%%%%%%%%%
% This package improves the interface for defining floating objects such
% as figures and tables in LaTeX.
% http://www.ctan.org/pkg/float
\usepackage{float}
\restylefloat{table}
%%%%%%%%%%%%%%%%%%%%%%%%%%%%%%%%%%%%%%%%%%%%%%%%%%%%%%%%%%%%%%%%%%%%%%%


%%% use for diagonals in the table's cells
%\usepackage{slashbox}  %Removed by Ares

%%%%%%%%%%%%%%%%%%%%%%%%%%%%%%%%%%%%%%%%%%%%%%%%%%%%%%%%%%%%%%%%%%%%%%%


%%%%%%%%%%%%%%%%%%%%%%%%%%%%%%%%%%%%%%%%%%%%%%%%%%%%%%%%%%%%%%%%%%%%%%%
% This package gives the enumerate environment an optional argument
% which determines the style in which the counter is printed.
% http://www.ctex.org/documents/packages/table/enumerate.pdf
\usepackage{enumerate}
%%%%%%%%%%%%%%%%%%%%%%%%%%%%%%%%%%%%%%%%%%%%%%%%%%%%%%%%%%%%%%%%%%%%%%%


%%%%%%%%%%%%%%%%%%%%%%%%%%%%%%%%%%%%%%%%%%%%%%%%%%%%%%%%%%%%%%%%%%%%%%%
%selectively in/exclude pieces of text: the user can determine new comment versions,
%and each is controlled separately. Special comments can be determined where the
%user specifies the action that is to be taken with each comment line.
% http://get-software.net/macros/latex/contrib/comment/comment.pdf
%\usepackage{comment}
%%%%%%%%%%%%%%%%%%%%%%%%%%%%%%%%%%%%%%%%%%%%%%%%%%%%%%%%%%%%%%%%%%%%%%%


%%%%%%%%%%%%%%%%%%%%%%%%%%%%%%%%%%%%%%%%%%%%%%%%%%%%%%%%%%%%%%%%%%%%%%
%% To display the labels used in a tex file in the dvi file (for example, if a theorem is labelled with the \label command) use the package
%% http://www.ctan.org/pkg/showkeys

%%%%%% Use Option 1
%%%%%% This will display all labels (for example, in the case of a labelled theorem, the label of the theorem will occur in the margin of the dvi file.
%%%%%% The command above by itself not only displays labels when they are first named but also when they are cited (or referenced).
%\usepackage{showkeys}

%%%%%% Use Option 2
%%%%%% It will NOT display citations and references. Only Labels
%\usepackage[notref, notcite]{showkeys}

%%%%%% Use Options 3
%%%%%% formats to smaller fonts the labels etc.
%\usepackage[usenames,dvipsnames]{color}
%\providecommand*\showkeyslabelformat[1]{{\normalfont \tiny#1}}
%\usepackage[notref,notcite,color]{showkeys}
%\definecolor{labelkey}{rgb}{0,0,1}
%%%%%%%%%%%%%%%%%%%%%%%%%%%%%%%%%%%%%%%%%%%%%%%%%%%%%%%%%%%%%%%%%%%%%%


%%%%%%%%%%%%%%%%%%%%%%%%%%%%%%%%%%%%%%%%%%%%%%%%%%%%%%%%%%%%%%%%%%%%%%%
% It extends the functionality of all the LATEX cross-referencing commands (including the table of contents, bibliographies etc) to produce \special commands which a driver can turn into hypertext links; it also provides new commands to allow the user to write ad hoc hypertext links, including those to external documents and URLs.
% http://www.tug.org/applications/hyperref/manual.html
\usepackage[colorlinks=true, pdfstartview=FitV, linkcolor=blue,
            citecolor=blue, urlcolor=blue]{hyperref}
\usepackage[usenames]{color}
%%%%%%%%%%%%%%%%%%%%%%%%%%%%%%%%%%%%%%%%%%%%%%%%%%%%%%%%%%%%%%%%%%%%%%%%
% custom textbox
\usepackage{tcolorbox}
\tcbuselibrary{breakable}
\tcbuselibrary{skins}
\tcbset{my left line/.style={
          enhanced, frame hidden, borderline west = {0.5pt}{0pt}{black}, % specify border
          opacityframe=0, opacityback=0,opacityfill=0, % color
          arc = 0mm, left skip=1em, % border
          left = 1mm,top=0mm,bottom=0mm,boxsep=1mm,middle=1mm, % text and border
}}

\newtcbox{\mybox}[1][]{my left line, #1}
\newtcolorbox{myleftlinebox}[1][breakable]{my left line, #1}
% usage
% use \mybox[on line]{your TEXT} for inline box otherwise forcing line breaks   
% for whole break, use \begin{myleftlinebox
%%%%%%%%%%%%%%%%%%%%%%%%%%%%%%%%%%%%%%%%%%%%%%%%%%%%%%%%%%%%%%%%%%%%%%%
\usepackage[utf8]{inputenc}


%%%%%%%%%%%%%%%%%%%%%%%%%%%%%%%%%%%%%%%%%%%%%%%%%%%%%%%%%%%%%%%%%%%%%%%
% a good looking way to format urls
% http://mirror.its.uidaho.edu/pub/tex-archive/help/Catalogue/entries/url.html
\usepackage{url}
% Define a new 'leo' style for the package that will use a smaller font.
\makeatletter\def\url@leostyle{%
 \@ifundefined{selectfont}{\def\UrlFont{\sf}}{\def\UrlFont{\scriptsize\ttfamily}}} \makeatother\urlstyle{leo}
%%%%%%%%%%%%%%%%%%%%%%%%%%%%%%%%%%%%%%%%%%%%%%%%%%%%%%%%%%%%%%%%%%%%%%%


%%%%%%%%%%%%%%%%%%%%%%%%%%%%%%%%%%%%%%%%%%%%%%%%%%%%%%%%%%%%%%%%%%%%%%%%%%%%
%%%%%% The present package defines the environment mdframed which automatically deals with page breaks in framed text.
%%%%%% http://mirrors.ibiblio.org/CTAN/macros/latex/contrib/mdframed/mdframed.pdf
\usepackage[framemethod=default]{mdframed}
%%%%%%%%%%%%%%%%%%%%%%%%%%%%%%%%%%%%%%%%%%%%%%%%%%%%%%%%%%%%%%%%%%%%%%%%%%%%

%%%%%%%%%%%%%%%%%%%%%%%%%%%%%%%%%%%%%%%%%%%%%%%%%%%%%%%%%%%%%%%%%%%%%%%%%%%%%%%%
%%%%%% An extended implementation of the array and tabular environments
%%%%%% which extends the options for column formats, and provides "programmable" format specifications.
%\usepackage{array}
%%%%%%%%%%%%%%%%%%%%%%%%%%%%%%%%%%%%%%%%%%%%%%%%%%%%%%%%%%%%%%%%%%%%%%%%%%%

%%%%%%%%%%%%%%%%%%%%%%%%%%%%%%%%%%%%%%%%%%%%%%%%%%%%%%%%%%%%%%%%%%%%%%%%%%%%
%%% The package enhances the quality of tables in LATEX, providing extra
%%% commands as well as behind-the-scenes optimization. Guidelines
%%% are given as to what constitutes a good table in this context.
%%% From version 1.61, the package offers long table compatibility.
%\usepackage{booktabs}
%%%%%%%%%%%%%%%%%%%%%%%%%%%%%%%%%%%%%%%%%%%%%%%%%%%%%%%%%%%%%%%%%%%%%%%%%%%%%%


\usepackage{tikz}




%% or use GEOMETRY package
\usepackage[margin=1.0in, letterpaper]{geometry}
%\def\baselinestretch{1.1}


%%%%%%%%%%%%% OR USE EXACT DIMENSIONS %
%\setlength{\textwidth}{6.5in}     %%
%\setlength{\oddsidemargin}{0in}   %%
%\setlength{\evensidemargin}{0in}  %%
%\setlength{\textheight}{8.5in}    %%
%\setlength{\topmargin}{0in}       %%
%\setlength{\headheight}{0in}      %%
%\setlength{\headsep}{.3in}         %%
%\setlength{\footskip}{.5in}       %%
%%%%%%%%%%%%%%%%%%%%%%%%%%%%%%%%%%%%%%%%%%%%%%%%%%%%%%%%%%%%%%%%%%%%%%%


%%%%%%%%%%%%%%%%%%%%%%%%%%%%%%%%%%%%%%%%%%%%%%%%%%%%%%%%%%%%%%%%%%%%%%%
%%%%%%%%%%%%%%%%%%%%%%%% NUMBERING %%%%%%%%%%%%%
\newtheorem{theorem}{Theorem}
\newtheorem{conjecture}{Conjecture}
\newtheorem{conclusion}{Conclusion}
\newtheorem{proposition}[theorem]{Proposition}
\newtheorem{lemma}[theorem]{Lemma}
\newtheorem{corollary}[theorem]{Corollary}
\newtheorem{assumption}{Assumption}
\newtheorem{condition}{C}
\theoremstyle{definition}
\newtheorem{definition}[theorem]{Definition}
\newtheorem{example}[theorem]{Example}
\theoremstyle{remark}
\newtheorem{remark}[theorem]{Remark}
\newtheorem{question}[theorem]{Question}
\newtheorem{problem}[theorem]{Problem}
\newtheorem{NB}[theorem]{Nota Bene}

\numberwithin{equation}{section}
\numberwithin{theorem}{section}
%\renewcommand{\labelitemi}{ {\small $\rhd$}}
%%%%%%%%%%%%%%%%%%%%%%%%%%%%%%%%%%%%%%%%%%%%%%%%%%%%%%%%%%%%%%%%%%%%%%%

%%%%%%%%%%%%%%%%%%%%%%%%%%%%%%%%%%%%%%%%%%%%%%%%%%%%%%%%%%%%%%%%%%%%%%%
\definecolor{Red}{rgb}{0.9,0,0.0}
\definecolor{Blue}{rgb}{0,0.0,1.0}
%%%%%%%%%%%%%%%%%%%%%%%%%%%%%%%%%%%%%
%%% used for editing and making comments in color
% Example \ig{Remarks and Commets}
\newcommand{\ig}[1]{\textcolor[rgb]{0.00, 0.0, 1.0}{{\tiny \textsuperscript{[\textrm{IC:Rem}]}} \ #1}}
\newcommand{\igAdd}[1]{\textcolor[rgb]{0.7, 0.0, 0.0}{{\tiny \textsuperscript{[\textrm{IC:Add}]}} \ #1}}
\newcommand{\igEdit}[1]{\textcolor[rgb]{0.7, 0.0, 0.0}{{\tiny \textsuperscript{[\textrm{IC:Edit}]}} \  #1}}

\newcommand{\trb}[1]{\begin{color}[rgb]{0.98, 0.0, 0.98}{TRB: #1} \end{color}}
\newcommand{\ti}[1]{\begin{color}[rgb]{1.00,0.00,0.00}{TI: #1} \end{color}}
\newcommand{\Red}[1]{\textcolor{Red}{#1}}
%%%%%%%%%%%%%%%%%%%%%%%%%%%%%%%%%%%%%

%%%%%%%%%%%%%%%%%%%%%%%%%%%%%%%%%%%%%
%%%     Igor's macros
%% \mathcal Letters
\def\cA{\mathcal{A}}
\def\cB{\mathcal{B}}
\def\cC{\mathcal{C}}
\def\cD{\mathcal{D}}
\def\cE{\mathcal{E}}
\def\cF{\mathcal{F}}
\def\cG{\mathcal{G}}
\def\cH{\mathcal{H}}
\def\cI{\mathcal{I}}
\def\cJ{\mathcal{J}}
\def\cK{\mathcal{K}}
\def\cL{\mathcal{L}}
\def\cM{\mathcal{M}}
\def\cN{\mathcal{N}}
\def\cO{\mathcal{O}}
\def\cP{\mathcal{P}}
\def\cQ{\mathcal{Q}}
\def\cR{\mathcal{R}}
\def\cS{\mathcal{S}}
\def\cT{\mathcal{T}}
\def\cU{\mathcal{U}}
\def\cV{\mathcal{V}}
\def\cW{\mathcal{W}}
\def\cX{\mathcal{X}}
\def\cY{\mathcal{Y}}
\def\cZ{\mathcal{Z}}

%% \mathbb Letters
\def\bA{\mathbb{A}}
\def\bB{\mathbb{B}}
\def\bC{\mathbb{C}}
\def\bD{\mathbb{D}}
\def\bE{\mathbb{E}}
\def\bF{\mathbb{F}}
\def\bG{\mathbb{G}}
\def\bH{\mathbb{H}}
\def\bI{\mathbb{I}}
\def\bJ{\mathbb{J}}
\def\bK{\mathbb{K}}
\def\bL{\mathbb{L}}
\def\bM{\mathbb{M}}
\def\bN{\mathbb{N}}
\def\bO{\mathbb{O}}
\def\bP{\mathbb{P}}
\def\bQ{\mathbb{Q}}
\def\bR{\mathbb{R}}
\def\bS{\mathbb{S}}
\def\bT{\mathbb{T}}
\def\bU{\mathbb{U}}
\def\bV{\mathbb{V}}
\def\bW{\mathbb{W}}
\def\bX{\mathbb{X}}
\def\bY{\mathbb{Y}}
\def\bZ{\mathbb{Z}}

%% \mathscr Letters, for filtration, sigma algebras etc
\def\sA{\mathscr{A}}
\def\sB{\mathscr{B}}
\def\sC{\mathscr{C}}
\def\sD{\mathscr{D}}
\def\sE{\mathscr{E}}
\def\sF{\mathscr{F}}
\def\sG{\mathscr{G}}
\def\sH{\mathscr{H}}
\def\sI{\mathscr{I}}
\def\sJ{\mathscr{J}}
\def\sK{\mathscr{K}}
\def\sL{\mathscr{L}}
\def\sM{\mathscr{M}}
\def\sN{\mathscr{N}}
\def\sO{\mathscr{O}}
\def\sP{\mathscr{P}}
\def\sQ{\mathscr{Q}}
\def\sR{\mathscr{R}}
\def\sS{\mathscr{S}}
\def\sT{\mathscr{T}}
\def\sU{\mathscr{U}}
\def\sV{\mathscr{V}}
\def\sW{\mathscr{W}}
\def\sX{\mathscr{X}}
\def\sY{\mathscr{Y}}
\def\sZ{\mathscr{Z}}


%%%% \mathsf for Matrices
\def\mA{\mathsf{A}}
\def\mB{\mathsf{B}}
\def\mC{\mathsf{C}}
\def\mD{\mathsf{D}}
\def\mE{\mathsf{E}}
\def\mF{\mathsf{F}}
\def\mG{\mathsf{G}}
\def\mH{\mathsf{H}}
\def\mI{\mathsf{I}}
\def\mJ{\mathsf{J}}
\def\mK{\mathsf{K}}
\def\mL{\mathsf{L}}
\def\mM{\mathsf{M}}
\def\mN{\mathsf{N}}
\def\mO{\mathsf{O}}
\def\mP{\mathsf{P}}
\def\mQ{\mathsf{Q}}
\def\mR{\mathsf{R}}
\def\mS{\mathsf{S}}
\def\mT{\mathsf{T}}
\def\mU{\mathsf{U}}
\def\mV{\mathsf{V}}
\def\mW{\mathsf{W}}
\def\mX{\mathsf{X}}
\def\mY{\mathsf{Y}}
\def\mZ{\mathsf{Z}}


%%%% \mathbf for Matrices or vectors
\def\bfB{\boldsymbol{B}}
\def\bfC{\boldsymbol{C}}
\def\bfD{\boldsymbol{D}}
\def\bfA{\boldsymbol{A}}
\def\bfE{\boldsymbol{E}}
\def\bfF{\boldsymbol{F}}
\def\bfG{\boldsymbol{G}}
\def\bfH{\boldsymbol{H}}
\def\bfI{\boldsymbol{I}}
\def\bfJ{\boldsymbol{J}}
\def\bfK{\boldsymbol{K}}
\def\bfL{\boldsymbol{L}}
\def\bfM{\boldsymbol{M}}
\def\bfN{\boldsymbol{N}}
\def\bfO{\boldsymbol{O}}
\def\bfP{\boldsymbol{P}}
\def\bfQ{\boldsymbol{Q}}
\def\bfR{\boldsymbol{R}}
\def\bfS{\boldsymbol{S}}
\def\bfT{\boldsymbol{T}}
\def\bfU{\boldsymbol{U}}
\def\bfV{\boldsymbol{V}}
\def\bfW{\boldsymbol{W}}
\def\bfX{\boldsymbol{X}}
\def\bfY{\boldsymbol{Y}}
\def\bfZ{\boldsymbol{Z}}

%%%% \mathbf for Matrices
\def\bfa{\boldsymbol{a}}
\def\bfb{\boldsymbol{b}}
\def\bfc{\boldsymbol{c}}
\def\bfd{\boldsymbol{d}}
\def\bfe{\boldsymbol{e}}
\def\bff{\boldsymbol{f}}
\def\bfg{\boldsymbol{g}}
\def\bfh{\boldsymbol{h}}
\def\bfi{\boldsymbol{i}}
\def\bfj{\boldsymbol{j}}
\def\bfk{\boldsymbol{k}}
\def\bfl{\boldsymbol{l}}
\def\bfm{\boldsymbol{m}}
\def\bfn{\boldsymbol{n}}
\def\bfo{\boldsymbol{o}}
\def\bfp{\boldsymbol{p}}
\def\bfq{\boldsymbol{q}}
\def\bfr{\boldsymbol{r}}
\def\bfs{\boldsymbol{s}}
\def\bft{\boldsymbol{t}}
\def\bfu{\boldsymbol{u}}
\def\bfv{\boldsymbol{v}}
\def\bfw{\boldsymbol{w}}
\def\bfx{\boldsymbol{x}}
\def\bfy{\boldsymbol{y}}
\def\bfz{\boldsymbol{z}}

%%%% \boldsymbol for lower case greek letter in math mode
\def\bfalpha{\boldsymbol{\alpha}}
\def\bfbeta{\boldsymbol{\beta}}
\def\bfgamma{\boldsymbol{\gamma}}
\def\bfdelta{\boldsymbol{\delta}}
\def\bfepsilon{\boldsymbol{\epsilon}}
\def\bfzeta{\boldsymbol{\zeta}}
\def\bfeta{\boldsymbol{\eta}}
\def\bftheta{\boldsymbol{\theta}}
\def\bfiota{\boldsymbol{\iota}}
\def\bfkappa{\boldsymbol{\kappa}}
\def\bflambda{\boldsymbol{\lambda}}
\def\bfmu{\boldsymbol{\mu}}
\def\bfnu{\boldsymbol{\nu}}
\def\bfomicron{\boldsymbol{\omicron}}
\def\bfpi{\boldsymbol{\pi}}
\def\bfrho{\boldsymbol{\rho}}
\def\bfsigma{\boldsymbol{\sigma}}
\def\bftau{\boldsymbol{\tau}}
\def\bfupsilon{\boldsymbol{\upsilon}}
\def\bfphi{\boldsymbol{\phi}}
\def\bfchi{\boldsymbol{\chi}}
\def\bfpsi{\boldsymbol{\psi}}
\def\bfomega{\boldsymbol{\omega}}

%%%% \boldsymbol for upper case greek letter in math mode
\def\bfPhi{\boldsymbol{\Phi}}
\def\bfTheta{\boldsymbol{\Theta}}
\def\bfPsi{\boldsymbol{\Psi}}
\def\bfOmega{\boldsymbol{\Omega}}
\def\bfSigma{\boldsymbol{\Sigma}}


%%%%%%%%%%%%%%%%%% Shortcuts %%%%%%%%%%%%%%%%%%%%%%%%%%%%%%%%%%%%%%%%%%%
\newcommand{\wt}{\widetilde}
\newcommand{\wh}{\widehat}

%%%%%%%%%%%%%%%%%%%%%%%%%%%%%%%%%%%%%%%%%%%%%%%%%%%%%%%%%%%%%%%%%%%%%%%%%%%%%%%%%
%%%%%%%%%%%%%%%%%%   Nonstandard notations  %%%%%%%%%%%%%%%%%%%%%%%%%%%%%%%%%%%%%
\newcommand{\pd}[1]{\partial_{#1}}      % partial derivative
\newcommand{\1}{\mathbbm{1}}            % preferable way of writing indicator function
\newcommand{\set}[1]{\{#1\}}            % set: {xyz} to be used for inline formulas
\newcommand{\Set}[1]{\left\{#1\right\}} % set: {xyz} to be used for separate (not inline) formulas
\renewcommand{\mid}{\,|\,}              % mid bar with small spaces before and after: x | y
\newcommand{\Mid}{\,\Big | \,}          % big bar with small spaces before and after:
\newcommand{\norm}[1]{ \| #1 \| }       % mid bar with small spaces before and after: x | y
\newcommand{\abs}[1]{\left\vert#1\right\vert}   % absolute value
\newcommand{\CB}[1]{\left\{ #1 \right\}}
\newcommand{\SB}[1]{\left[ #1 \right]}
\newcommand{\Pare}[1]{\left( #1 \right)}
\newcommand{\AB}[1]{\left \langle #1 \right \rangle}
\newcommand{\given}[1]{\left.#1\right|}
\newcommand{\givenAlt}[1]{\left.#1\right]}
\newcommand{\Tran}[1]{{#1}^\top}
\newcommand{\bsde}{BS$\Delta$E}         % BS\DeltaE
\newcommand{\bsdes}{BS$\Delta$Es}       % BS\DeltaE
\newcommand{\ow}{\text{otherwise}}
\newcommand{\imblies}{\Longleftarrow}
\newcommand{\Exp}[1]{\mathrm{E}\left[ #1 \right]}
\newcommand{\Var}[1]{\mathrm{Var}\left[ #1 \right]}
\newcommand{\Cov}[1]{\mathrm{Cov}\left[ #1 \right]}
\newcommand{\Bspace}{\;\;\;\;}
\newcommand{\dif}{\,\mathrm{d}}        % used for differential, same as in commath.sty

\newcommand{\LHS}{\text{LHS}} % left hand side
\newcommand{\RHS}{\text{RHS}} % right hand side
\newcommand{\Adj}{\text{Adj}} %adjoint matrix

\newcommand{\RNum}[1]{\uppercase\expandafter{\romannumeral #1\relax}} % romanian numerals

\DeclareMathOperator{\logit}{logit}
\DeclareMathOperator*{\esssup}{ess\,sup} % ess sup
\DeclareMathOperator*{\essinf}{ess\,inf} % ess inf
\DeclareMathOperator*{\esslimsup}{ess\,\limsup}
\DeclareMathOperator*{\essliminf}{ess\,\liminf}
\DeclareMathOperator*{\argmin}{arg\,min} % argmin
\DeclareMathOperator*{\diag}{diag} % diag
\DeclareMathOperator*{\argmax}{arg\,max} % argmax
\DeclareMathOperator*{\Arg}{Arg} % arguments
\DeclareMathOperator*{\rank}{rank\,} % argmax
\DeclareMathOperator*{\KL}{KL} % KL divergence
\DeclareMathOperator*{\Proj}{Proj} % Projection
\DeclareMathOperator{\Std}{\mathrm{Std}} % \std for Standard deviation
\DeclareMathOperator{\sgn}{\mathrm{sgn}} % sign of a variable
\DeclareMathOperator{\tr}{\mathrm{trace}} % matrix trace
% trigonometric and hyperbolic functions
\DeclareMathOperator{\sech}{sech}
\DeclareMathOperator{\csch}{csch}
\DeclareMathOperator{\arcsec}{arcsec}
\DeclareMathOperator{\arccot}{arcCot}
\DeclareMathOperator{\arccsc}{arcCsc}
\DeclareMathOperator{\arccosh}{arcCosh}
\DeclareMathOperator{\arcsinh}{arcsinh}
\DeclareMathOperator{\arctanh}{arctanh}
\DeclareMathOperator{\arcsech}{arcsech}
\DeclareMathOperator{\arccsch}{arcCsch}
\DeclareMathOperator{\arccoth}{arcCoth} 

%%%%%%%%%%%%%%%%%%%%%%%%%%%%%%% FINANCE %%%%%%%%%%%%%%%%%%%%%%%%%%%%%%%%%%%%%%%%%%%%%%%%%%
\DeclareMathOperator{\var}{\mathrm{V}@\mathrm{R}}           % \V@R Value-at-risk
\DeclareMathOperator{\tce}{\mathrm{TCE}}                    % Tail Conditional Expectation
\DeclareMathOperator{\tvar}{\mathrm{TV}@\mathrm{R}}         % \TV@R tail Value-at-risk
\DeclareMathOperator{\avar}{\mathrm{AV}@\mathrm{R}}         % \AV@R average Value-at-risk
\DeclareMathOperator{\ent}{\mathrm{Ent}}                    % \ent = Entropic Risk Measure
\DeclareMathOperator{\glr}{\mathrm{GLR}}                    % \glr = gain to loss ratio

\DeclareMathOperator{\ES}{\mathrm{ES}}

\newcommand{\ask}{\text{ask}}           % ask price
\newcommand{\bid}{\text{bid}}           % bid price

%%%%%%%%%%%%%%%%%%%%%%%%%%%%%%%%%%%%%%%%%%%%%%%%%%%%%%%
% \bibliographystyle{amsplain}
% \bibliographystyle{alpha} % standard LaTeX bibliography format. Preferable to be used
% \bibliography{MathFinanceMaster-12-28-2014}
%\bibliography{D:/_research/latex/lib_igor/igor_bib_mathfinance}
% help on how to use several bit files \bibliography{videogames,comics,interface,theory}
%%%%%%%%%%%%%%%%%%%%%%%%%%%%%%%%%%%%%%%%%%%%%%%%%%%%%%%




\title{Gleanings on Engineering Analysis I}
\author{Yuanxing Cheng, A20453410, MMAE-501-f22}


\begin{document}

\maketitle

\section{On Diagonalization of Matrices}

\(Q\) diagonalizes \(A\) is \(Q^\top A Q =D\) is a diagonal matrix. \(Q\) is the called {\it modal matrix}. A general procedure for real symmetric matrix \(A_{N\times N}\).

\begin{enumerate}
    \item notice for real symmetric matrix, it has \(N\) different eigenvalues, here we obtain them and name them as \(\lambda_i\)
    \item get normalized eigenvectors \(q_i\)
    \item construct orthogonal modal matrix \(Q=[q_1,\dots,q_N]\)
    \item then \(D = \diag(\lambda_1,\dots,\lambda_N)\)
\end{enumerate}

Then, we write \(x=Qy\) and then write the {\it canonical form} of the quadratic form.

\[x^\top A x=y^\top Q^\top AQ y = \sum_N \lambda_i y_i^2  \]

Then clearly, \(Q\) and \(A\) share the same eigenvalues ({\it spectral theorem}), and then we call \(D = Q^\top A Q\) the similarity transformation.


\begin{theorem}[Spectral theorem in finite space]
    for normal matrix, it's diagonalizable using its orthogonal model matrix. The result share the same eigenvalues and they are similar.
\end{theorem}

\begin{remark}
    For real and symmetric matrix \(\bfA\) with distinct eigenvalues, the eigenvectors are mutually orthogonal and we have decomposition: \(\bfA = \bfQ \bfD \bfQ^\top\) where \(\bfQ\) is the modal matrix comprised of the orthonormal eigenvectors of \(\bfA\) and \(\bfD\) is a diagonal matrix having the eigenvalues of \(\bfA\) along its diagonal; for nonsymmetric matrix with distinct eigenvalues, the eigenvectors are mutually linearly independent, but not necessarily orthogonal. So we can have the decomposition: \(\bfA = \bfU \bfD \bfU^{-1}\) and \(\bfU\) is the modal matrix containing the linearly-independent eigenvectors of \(\bfA\). For nonsymmetric matrix with repeated eigenvalues, it ends up {\it Jordan canonical form} and we write \(\bfA = \bfU\bfJ\bfU^{-1}\) where \(\bfJ\) is the nearly-diagonal Jordan canonical form.
\end{remark}



\section{On solving DEs}

\subsection{System of N coupled first-order linear ordinary differential equations}

Consider

\begin{equation*}
    \begin{split}
        \dot u_1(t) &= A_{11} u_1(t) + A_{12} u_2(t) + \cdots+ A_{1N} u_N(t) + f_1(t)\\
        \dot u_2(t) &= A_{21} u_1(t) + A_{22} u_2(t) + \cdots+ A_{2N} u_N(t) + f_2(t)\\
         & \vdots\\
        \dot u_N(t) &= A_{N1} u_1(t) + A_{N2} u_2(t) + \cdots+ A_{NN} u_N(t) + f_N(t)\\
    \end{split}
\end{equation*}

In matrix form, we write \(\dot \bfu(t) = \bfA \bfu(t)+\bff(t)\). If \(\bff\equiv 0\) then the system is homogeneous.

To solve this, we first diagonalize \(\bfA\).

\begin{enumerate}
    \item Diagonalize \(\bfA\) and obtain \(D = U^{-1}\bfA U\)
    \item Consider \(\dot \bfu(t) = U \dot \bfv(t)\)
    \item then we have \(\dot\bfv(t) = U^{-1}AU\bfv(t) + U^{-1}\bff(t)=D\bfv(t)+U^{-1}\bff(t)\)
    \item we then obtain the uncoupled system and the homogeneous system will have an obvious soution: \(v_i (t)= c_i e^{\lambda_i t}\) where \(\lambda_i\) is the \(i\)-th element of the diagonal of matrix \(D\).
    \item For nonhomogeneous system, the particular solution can be found using method of undetermined coefficients.
    \item Finally, \(\bfu(t) = U\bfv(t)\) is the solution we want.
    
\end{enumerate}

\subsection{N-th order Linear ODE}

Consider

\[u^{(N)}=F(t,u,\dot u,\dots,u^{(N-1)})\]

We write it into a system of N first-order differential equations by substitutions \(u_i(t) = u^{(i-1)}(t)\).

\begin{remark}
    For linear equations, both the real and imaginary parts of complex solutions are by themselves solutions of the differential equations, and a linear combination of the real and imaginary parts, is also a solution of the linear equations. The real and imagainary parts is obtained using {\it Euler's formula}: \(e^{\theta it} =\cos(\theta t)+i\sin(\theta t)\)
\end{remark}

\begin{remark}
    Above we are only discussing {\it autonomous} system meaning \(\bfA\) does not depend on time.
\end{remark}

\section{Decompositions}

\subsection{SVD}
For matrix \(A\) of size \(m\times n\) we want to decomposite it to \(A = U\Sigma V^\top\) where \(U\) is \(m\times m\) orthogonal matrix, \(V\) is \(n\times n\) orthogonal matrix, \(\Sigma\) is an \(m \times n\) diagonal matrix. In the following, \(u_i\) and \(v_i\) are column vectors of \(u\) and \(V\). 

\begin{enumerate}
    \item \(AA^\top u_i = \sigma_i^2 u_i = \sigma_i A v_i\), solving left singular vectors of \(A\), and singular values \(\sigma_i\). We also want \(\sigma_1\geq\sigma_2\cdots\sigma_p\geq 0\) where \(p=\min (m,n)\).
    \item Alternatively, we can solve right singular vectors from \(A^\top A v_i = \sigma_i^2 v_i = \sigma_i A^\top u_i\)
    \item obtain full \(U\) and \(V\) and orthonormalizing them.
    \item fill \(\Sigma\) left columns in \(0\).
    \item The final result can be also expressed in the form: \(A = \sum_{i=1}^p \sigma_i u_i v_i^\top\).
\end{enumerate}

\begin{remark}
    in image compression, we might use \(k<p\).
\end{remark}

\subsection{Polar decomposition}

Following SVD, we might futher write \(A = U\Sigma V^\top = UV^\top V\Sigma V^\top\) and break this up into \(A = WR\) where \(W = UV^\top\) and \(R = V\Sigma V^\top\).

Notice \(U\) and \(V\) are both orthogonal so that \(W\) is also orthogonal; and \(R\) is symmetric since \(\Sigma\) is symmetric.

\begin{enumerate}
    \item Notice \(A^\top A = R^\top R = R^2\), we first solve \(R\).
    \item Notice we can diagonalize \(R^2\) and \(R\): \(Q^\top R^2 Q = \diag(\lambda_1^2,\cdots,\lambda_N^2)\), \(Q^\top R Q = \diag(\lambda_1,\cdots,\lambda_N)\)
    \item then \(R = Q\diag(\lambda_1,\cdots,\lambda_N) Q^\top\)
    \item \(W = AR^{-1}\)
\end{enumerate}

\subsection{QR}

Write \(A = [u_1,u_2,\dots, u_S]\) and we want \(A = QR\) where \(Q\) is orthonormal matrix \([q_1,q_2,\dots, q_S]\) and \(R\) is right triangular matrix. Following Gram-schmidt, we are transforming \(u_i\) to \(q_i\). For \(i=2,3,\dots, S\),

\[\hat u_i = u_i - \sum_{k=1}^{i-1} \AB{u_i,q_k} q_k\]

then we normalize them \(q_i = \frac{\hat u_i}{\norm{\hat u_i}}\). And this GS orthogonalization is actually  a matrix decomposition for \(A\) because we actually have \(\hat u_i = \norm{\hat u_i} q_i\), \(\norm{\hat u_i}=\AB{\hat u_i,q_i}\), and 

\[u_i = \norm{\hat u_i} q_i + \sum_{k=1}^{i-1} \AB{u_i,q_k}q_k\]

Above in a nicer way:

\begin{equation*}
    \begin{split}
        i=1& : u_1 =\AB{u_1,q_1} q_1,\\
        i=2& : u_2 =\AB{u_2,q_1}q_1,+\AB{u_2,q_2}q_2,\\
        \vdots &
    \end{split}
\end{equation*}

or in matrix form: \(A=QR\) where \(Q=[q_1,q_2,\dots,q_S ]\) and 

\[R=\begin{bmatrix}
    \AB{u_1,q_1} & \AB{u_2,q_1} & \AB{u_3,q_1} & \cdots &\AB{u_S,q_1}\\
    0            & \AB{u_2,q_2} & \AB{u_3,q_2} & \cdots &\AB{u_S,q_2}\\
    \vdots & &&& \AB{u_S,q_S}
\end{bmatrix}\]

\subsection{LU decomposition}

\subsection{CHolesky decomposition}

\section{Differential Eigenproblems}

\subsection{Function Spaces, bases, and orthogonalization}

\begin{definition}
    A set of functions \(u_i(x)\) for \(i=1,2,\dots,M\) are {\it linearly independent} over an interval \(a\leq x\leq b\) is their linear combination is zero is all coefficients are zero \(\forall x\in [a,b]\).
\end{definition}

\begin{remark}
    For any piecewise continuous function \(u(x)\) we have expression:
    \[u(x)=\sum_{i=1}^\infty a_n u_n(x)\] where \(a_n\in\bR \) and \(a_n\) are not all zeros, \(u_n(x)\) are linear-independent {\it basis functions}. When there are actually infinite sum, we say \(u(x)\) is {\it infinite dimensional}.
\end{remark}

\begin{definition}
    We also want basis functions mutually orthogonal, so here we define the inner product of two functions.
    \[
        \AB{u_m(x),u_n(x)}=\int_a^b w(x)u_m(x)u_n(x)\dif x
    \]
    orthogonality means  \(\AB{u_m(x),u_n(x)}\equiv 0\forall x\in [a,b]\), and \(w(x)\) is a weight function. \(w(x)\equiv 1\) unless stated otherwise.
\end{definition}

\begin{definition}[Legendre polynomials]
    Obtained from Gram-schmidt process on \(u_i(x)=x^i\) for \(i=0,1,\dots\) and \(-1\leq x\leq 1\), the first several terms are \(\tilde u_i(x) = 1,x,\frac{1}{2}(3x^2-1),\dots\). Then any piecewise continuous function \(f(x)\) in the interval \(-1\leq x\leq 1\) can be written as 
    \[f(x) = \sum_{i=1}^\infty \AB{f(x),\tilde{u}_i(x)} \tilde{u}_i(x)\]
\end{definition}

\begin{remark}
    When considering bases \(\cos(ix),\sin(ix)\) for \(i=0,1,\dots\) and \(0\leq x\leq 2\pi\), we end up with {\it Fourier series},
    \begin{equation*}
        \begin{split}
            f(x) &= \frac{a_0}{\sqrt{2\pi}}+\sum_{i=1}^\infty a_i \frac{\cos(ix)}{\sqrt{\pi}}+\sum_{i=1}^\infty b_i \frac{\sin(ix)}{\sqrt{\pi}}\\
            a_0 &= \AB{f(x),\frac{1}{\sqrt{2\pi}}}=\frac{1}{\sqrt{2\pi}} \int_0^{2\pi} f(x)\dif x\\
            a_i &= \AB{f(x),\frac{\cos(ix)}{\sqrt{\pi}}}=\frac{1}{\sqrt{\pi}} \int_0^{2\pi} f(x)\cos(ix)\dif x\\
            b_i &= \AB{f(x),\frac{\sin(ix)}{\sqrt{\pi}}}=\frac{1}{\sqrt{\pi}} \int_0^{2\pi} f(x)\sin(ix)\dif x
        \end{split}
    \end{equation*}
\end{remark}

\begin{definition}[Eigenproblems]
    Consider \(n\)-th order differential equation \(\cL u(x)=f(x)\) where 
    \[\cL = a_0+a_1\frac{\dif}{\dif x} + a_2 \frac{\dif^2}{\dif x^2}+\cdots+a_n \frac{\dif^n}{\dif x^n}\]
    We then tend to find \(\lambda_i\) so that  \(\cL u_i(x) = \lambda_i u_i(x)\). We call the nontrivial solution \(u_i(x)\) the eigenvalue of \(\cL\) and \(u_i(x)\) the {\it eigenfunction} of \(\cL\).
\end{definition}

Given that the differential operator \(\cL\) is self-adjoint (defined below) and boundary conditions,

\begin{enumerate}
    \item Solve \(\cL u_i(x) = \lambda_i u_i(x)\) as a \(N\)-th order Linear order ODE,  excluding the trivial solution.
    \item Usually we end up in a characteristic equation with roots being the eigenvalues
    \item And when solving for coefficients of general solutions, make sure the norm of eigenfunctions being \(1\), resulting in normalized eigenfunctions.
    \item write \(f(x) = \sum_{i=1}^\infty \hat a_i u_i(x)\) where \(\hat a_i=\AB{f(x),u_i(x)}\)
    \item write \(u(x) = \sum_{i=1}^\infty  a_i u_i(x)\) where \(a_i = \frac{\hat a_i}{\lambda_i}\), so that \(\cL u(x)=f(x)\)
    \item The final solution can be written as 
    \[u(x) = \sum_{i=1}^\infty \frac{\AB{f(x),u_i(x)}}{\lambda_i} u_i(x)\]
\end{enumerate}

\section{Adjointness}

\begin{definition}[from matrix transpose to adjoint operator]
    Consider matrix \(A\) of size \(n\times n\), arbitrary vectors \(u\) and \(v\) or size \(n\times 1\). We define \(A^\top\) the matrix such that \(\AB{v,Au}=\AB{A^\top v, u}\). And in real case, \(\AB{v,Au}=\AB{u, A^\top v}\). Likewise, for operator \(\cL\) we define its adjoint \(\cL^*\) such that 
    \[\AB{v,\cL u} = \AB{u,\cL^* v}\]
    where \(u(x)\) and \(v(x)\) are arbitrary functions with homogeneous boundary conditions.
\end{definition}

\begin{remark}
    An easy way to find an operator's adjoint is to use test function, then do integration by parts.
\end{remark}

\begin{example}[Second order linear differential equation with variable coefficients]
    \[\cL u =\frac{1}{w(x)}\Pare{a_0(x)u''(x)+a_1(x)u'(x)+a_2(x)u(x)}=0\]
    where \(x\in [a,b]\) and \(w(x)\) is a weight function.
    \begin{enumerate}
        \item Consider test function \(v(x)\) and find
        \begin{equation*}
            \begin{split}
                \AB{v,\cL u}&=\int_a^b w(x)v(x) \Pare{\frac{1}{w(x)}\Pare{a_0(x)u''(x)+a_1(x)u'(x)+a_2(x)u(x)}}\dif x\\
                &= \int_a^b \Pare{a_0vu''+a_1 vu'+a_2vu}\dif x
            \end{split}
        \end{equation*}
        \item using integration by parts to down-order \(u\): \(\int p\dif q=pq-\int q\dif p\), we obtain the following:
        \begin{itemize}
            \item \(\int_a^b a_0vu''\dif x= \int_a^b a_0v\dif u'= \given{a_0vu'}_a^b -\int_a^b u'(a_0v)'\dif x =\given{a_0vu'}_a^b -\given{(a_0v)'u}_a^b +\int_a^b u(a_0v)''\dif x \)
            \item \(\int_a^b a_1 vu'\dif x = \int_a^b a_1 v\dif u=\given{a_1vu}_a^b-\int_A^b u(a_1v)'\dif x\)
            \item \(a_2(x)\) term with \(u(x)\) so no need to do integration by parts
        \end{itemize}
        \item combine these we end up with
        \[\AB{v,\cL u}=\Pare{a_0 vu'-(a_0v)'u+a_1vu}_a^b +\int_a^b w(x)u(x)\Pare{\frac{1}{w(x)}\Pare{(a_0v)''-(a_1v)'+a_2v}}\dif x \]
        \item above should equal to \(\AB{u,\cL^* u}\) so given homogeneous BC, \(\cL^*v=\frac{1}{w(x)}\Pare{(a_0v)''-(a_1v)'+a_2v}\)
    \end{enumerate}
\end{example}
\begin{remark}
    variable coefficients move inside the derivatives, and the odd-order derivatives change sign as compared to \(\cL\).
\end{remark}

\begin{definition}
    A differential operator \(\cL\) is {\it self-adjoint}, or Hermitian, if \(\cL=\cL^*\).
\end{definition}
\begin{remark}
    Similar to symmetric matrix, its eigenvectors are mutually orthogonal if its eigenvalues are distinct, self-adjoint operator's eigenfunctions are mutually orthogonal if its eigenvalues are distinct.
\end{remark}

\subsection{Sturm-Liouville differential operator}

For differential equations in the previous example, use product rule we have

\[\cL^* v=\frac{1}{w(x)}\Pare{(a_0v)''-(a_1v)'+a_2v}=\frac{1}{w(x)}\Pare{a_0v''+(2a_0'-a_1)v'+(a_0''-a_1'+a_2)v}\]

Letting \(\cL=\cL^*\), we want
\begin{align*}
    a_1 &= 2a_0'-a_1 \implies a_1 = a_0'\\
    a_2 &= a_0''-a_1'+a_2
\end{align*}

So a self-adjoint differential operator may have the form:

\begin{align*}
    \cL u &= \frac{1}{w(x)} \Pare{a_0u''+a_0'u'+a_2u}\\
    \cL &= \frac{1}{w(x)} \Pare{\frac{\dif}{\dif x}\Pare{p(x)\frac{\dif}{\dif x}}+q(x)}
\end{align*}

where \(p(x)>0\) and \(w(x)>0\) in \(a\leq x\leq b\) and the boundary conditions are homogeneous.
\begin{remark}
    Similarly, we have the forth-order Sturm-Liouville differential operator
    \[\cL = \frac{1}{w(x)}\Pare{\frac{\dif^2}{\dif x^2}\Pare{s(x)\frac{\dif^2}{\dif x^2}} +\frac{\dif}{\dif x}\Pare{p(x)\frac{\dif}{\dif x}}+q(x) }\]
\end{remark}

\begin{remark}[Eigenfunctions of Sturm-Liouville Operators]
    Still we assume homogeneous boundary conditions, and consider the eigenproblrm \(w(x)\cL u_i(x)=-\lambda_i w(x) u_i(x)\). We have
    \begin{itemize}
        \item the eigenvalues are distinct and nonnegative
        \item the eigenfunctions \(u_i(x)\) are orthogonal wrt the weight function \(w(x)\).
    \end{itemize}
\end{remark}

\begin{remark}
    some special cases
    \begin{itemize}
        \item taking \(p\equiv 1\) and \(q\equiv 0\) on \(x\in[0,1]\)  produces eigenfunctions. Either fourier sine series or fourier cosine series, for different set of boundary conditions.
        \item taking \(p(x)=x\), \(q(x)=\frac{-\nu^2}{x}\), \(w(x)=x\) on \(x\in[0,1]\) produces Bessel functions, with \(\lambda_i=\mu_i^2\) 
        \item taking \(p(x)=1-x^2\), \(q(x)=0\), \(w(x)=1\) on \(x\in[-1,1]\) produces Legendre Polynomials, with \(\lambda_i=\nu(\nu+1)\)
        \item Legendre polynomials have recursion relation: 
        \[(\nu+1)P_{\nu+1}(x)-(2\nu+1)xP_\nu(x)+\nu P_{\nu-1}(x)=0\]
        \item taking \(p(x)=\sqrt{1-x^2}\), \(q(x)=0\), \(w(x)=\frac{1}{\sqrt{1-x^2}}\) on \(x\in[-1,1]\) produces Chebyshev polynomials, with \(\lambda_i=\nu^2\)
        \item Chebyshev polynomials have recursion relation:
        \[T_{\nu+1}(x)-2xT_\nu(x)+T_{\nu-1}(x)=0\]
    \end{itemize}
\end{remark}

\begin{remark}
    for nonhomogeneous boundary conditions, we can apply certain transformation.
\end{remark}

\section{Separation of Variables}

A new way to solve PDEs, by assume solution \(u(x,t)=\phi(x)\psi(t)\).

\subsection{General hyperbolic partial differential equation}

Consider \(\cM \ddot u+\cK u=0\) where \(\cM\) and \(\cK\) are linear differential operators in space, and dots denote differential in time.

Substituting into the governing equation gives

\[\ddot \psi \cM \phi + \psi\cK \phi=0\implies \frac{\cK \phi}{\cM \phi}=-\frac{\ddot\psi}{\psi}=\lambda\]

it's a constant since LHS only on space and RHS only on time.

Then we solve uncoupled differential equations

\begin{align*}
    \cK \phi(x) &=\lambda\cM \phi(x)\\
    \ddot\psi(t) &= -\lambda \psi(t)
\end{align*}
\subsection{Electrmagnetics}
\subsection{Schr\"odinger Equation}

\section{Complex variables}

\subsection{Complex functions}

On trigonometric functions, using taylor series

\begin{align*}
    \sin z &= z-\frac{z^3}{3!}+\frac{z^5}{5!}-\cdots = \sum_{n=0}^\infty (-1)^n \frac{z^{2n+1}}{(2n+1)!}=\frac{e^{iz}-e^{-iz}}{2i}\\
    \cos z &= 1-\frac{z^2}{2!}+\frac{z^4}{4!}-\cdots = \sum_{n=0}^\infty (-1)^n \frac{z^{2n}}{(2n)!}=\frac{e^{iz}+e^{-iz}}{2}\\
    e^{iz} &= \cos z+i\sin z
\end{align*}

\begin{theorem}[De Moivre's Theorem]
    \[z^n = (re^{i\theta})^n r^n (\cos(n\theta)+i\sin(n\theta))\]
\end{theorem}

On hyperbolic functions, similarly we have

\begin{align*}
    \sinh z &= z+\frac{z^3}{3!}+\frac{z^5}{5!}+\cdots = \sum_{n=0}^\infty\frac{z^{2n+1}}{(2n+1)!}=\frac{e^{z}-e^{-z}}{2i}\\
    \cosh z &= 1+\frac{z^2}{2!}+\frac{z^4}{4!}+\cdots = \sum_{n=0}^\infty \frac{z^{2n}}{(2n)!}=\frac{e^{z}+e^{-z}}{2}\\
    \sinh(iz) &= i\sin z, \sin(iz)=i\sinh(z)\\
    \cosh(iz) &= \cos z, \cos(iz) =\cosh(z)
\end{align*}

On expenentials, for \(z=x+iy\)

\begin{align*}
    e^z &= 1+z+\frac{z^2}{2!}+\frac{z^3}{3!}+\cdots = \sum_{n=0}^\infty \frac{z^n}{n!}\\
    e^z &= e^x(\cos y+i\sin y)
\end{align*}

On Logarithms,  for \(z=re^{i\theta}\neq 0\)

\[\log z = \log r+i\theta = \log\abs{z}+i(\theta_0+2k\pi)\]

a multi-valued function. We define the pricinple argument where \(-\pi<\theta_0\leq\pi\), with \(k\in \bZ\).

on fractional powers

\begin{align*}
    z^{n/m} &= \exp\CB{\log z^{n/m}}\\
    &= \exp\CB{(n/m)\log (r+i(\theta_0+2k\pi))}\\
    &= r^{n/m}\exp\CB{i(n/m)(\theta_0+2k\pi)}, k=0,1,2,\dots,m-1
\end{align*}

on inverse trigonometric functions, these values are solved from quadratic equation. For example with \(z=\sin w = \dfrac{e^{iw}-e^{-iw}}{2i}\) we have \((e^{iw})^2-2iz(e^{iw})-1=0\) and thus \(w = \sin^{-1} z = \frac{1}{i}\log(iz+(1-z^2)^{1/2})\). Similarly we have the following results

\begin{align*}
    \sin^{-1} z &= \frac{1}{i}\log(iz+(1-z^2)^{1/2})\\
    \cos^{-1} z &= \frac{1}{i}\log(z+(z^2-1)^{1/2})\\
    \tan^{-1} z &= \frac{1}{2i} \log\frac{i-z}{i+z}\\
    \sinh^{-1}z &= \log(z+(1+z^2)^{1/2})\\
    \cosh^{-1}z &= \log(z+(z^2-1)^{1/2})\\
    \tanh^{-1}z &= \frac{1}{2}\log\frac{1+z}{1-z}
\end{align*}

\subsection{Analytic functions}

\begin{definition}[Analyticity]
    A complex function \(f(z)\) is anlytic at a point \(z=z_0\) if \(f(z)\) is {\em differentiable}  at \(z_0\) and in some neighbourhood of \(z_0\).
    \begin{itemize}
        \item and is {\em entire} if analytic in the entire complex plane.
        \item and points where a function is not analytic, are {\em singular points}.
    \end{itemize}
\end{definition}

\begin{theorem}[Goursat's theorem]
    If \(f(z)\) is analytic at a point \(z_0\), then \(f'(z)\) is continuous at \(z_o\).
\end{theorem}

\begin{theorem}[Cauchy-Riemann equations]
    In order for a function \(w=f(z)=u(x,y)+iv(x,y)\) to be analytic in some region in \(\bR^2\), the following should be satisfied:
    \[\frac{\partial u}{\partial x}=\frac{\partial v}{\partial y}, \frac{\partial v}{\partial x}=-\frac{\partial u}{\partial y}\]
    Notice with this we further have the continuity of these partial derivatives. Thus we might conclude: Cauchy-Riemann equations \(\implies f(z)\) being analytic \(\implies\) continuity of partial derivatives.

    We can also derive its expression in polar coordinates. Using change of variables we end up with
    \[\frac{\partial u}{\partial r}=\frac{1}{r}\frac{\partial v}{\partial \theta}, \frac{\partial v}{\partial r}=-\frac{1}{r}\frac{\partial u}{\partial \theta}\]
\end{theorem}

\begin{corollary}
    Then the derivatives have the forms
    \begin{align*}
        \frac{\dif w}{\dif z}=f'(z) &= \frac{\partial u}{\partial x} +i\frac{\partial v}{\partial x}\\
         &= \frac{\partial v}{\partial y}-i\frac{\partial u}{\partial y}\\
         &= \Pare{\frac{\partial u}{\partial r}+i\frac{\partial v}{\partial r}}(\cos\theta-i\sin\theta)\\
         &= \Pare{\frac{\partial u}{\partial \theta}+i\frac{\partial v}{\partial \theta}}\frac{-i}{r}(\cos\theta-i\sin\theta)
    \end{align*}
\end{corollary}

On the slopes of \(u\) and \(v\). Consider curve \(u(x,y)=U_1\) then
\[\dif u=0=\frac{\partial u}{\partial x}\dif x+\frac{\partial u}{\partial y}\dif y\implies \Pare{\frac{\dif y}{\dif x}}_{u=U_1} = -\frac{\partial u/\partial x}{\partial u/\partial y}\]

Similarly we have \[\Pare{\frac{\dif y}{\dif x}}_{v=V_1} = \frac{\partial u/\partial y}{\partial u/\partial x} = -\frac{1}{ \Pare{\frac{\dif y}{\dif x}}_{u=U_1}}\].

Also notice that we have \(\nabla^2 u=\nabla^2 v=0\) and we call the solutions of laplace's equation the harmonic functions. And call \(u(x,y)\) and \(v(x,y)\) the harmonic conjugates of one another.

\subsection{Branch points and branch cuts}

Consider the multi-valued function 
\[\log z = \log \abs z + i(\theta_0+2k\pi)\]

where the principle argument is in range \(-\pi < \theta_0 \leq \pi \) and \(k\in\bZ\).

\begin{definition}[branch]
    \begin{itemize}
        \item branch is a single-valued and analytic portion of a multi-valued function, say we can fix \(k=0\), which is the principle branch.
        \item  The branch point is a point that must be inside any patch that is necessary to change branches of a function. In the example, \(z=0\) is a branch point of \(\log z\).
        \item The branch cut is a curve along which the function is discontinuous if it is to remain single-valued. For example \(y=0, x\leq 0\) the negative real axis ia a branch cut of \(\log z\).
    \end{itemize}
\end{definition}

\subsection{Conformal mapping and boundary value problems}

First, we view complex function \(w=f(z)=u(x,y)+iv(x,y)\) as a mapping from \(x,y\)-plane to \(u,v\)-plane. We want the mapping to be one-to-one and conformal.

\begin{theorem}[one-to-one mapping]
    A mapping \(w=f(z)\) is one-to-one at \(z_0\) if 
    \begin{itemize}
        \item \(f(z)\) is analytic at \(z_0\) 
        \item \(f'(z_0) \neq 0\) (points where \(f'(z)=0\) are called the critical points)
    \end{itemize}
    And then a unique inverse mapping function \(z=F(w)\) exists at \(z_0\).
\end{theorem}

Sketch of the proof. Notice we have

\begin{equation*}
    \begin{bmatrix}
    \frac{\partial u}{\partial x} & \frac{\partial v}{\partial x}\\
    \frac{\partial u}{\partial y} & \frac{\partial v}{\partial y}
    \end{bmatrix} \begin{bmatrix}
        \frac{\partial}{\partial u}\\
        \frac{\partial}{\partial v}
    \end{bmatrix} = \begin{bmatrix}
        \frac{\partial}{\partial x}\\
        \frac{\partial}{\partial y}
    \end{bmatrix}    
\end{equation*}

Above linear system has unique solution if {\em Jacobian} is not zero.

\begin{equation*}
    J = \begin{vmatrix}
        \frac{\partial u}{\partial x} & \frac{\partial v}{\partial x}\\
        \frac{\partial u}{\partial y} & \frac{\partial v}{\partial y}
    \end{vmatrix} = \frac{\partial u}{\partial x} \frac{\partial v}{\partial y}- \frac{\partial v}{\partial x}\frac{\partial u}{\partial y} \neq 0
\end{equation*}

Thus if further given the condition that the function is analytic, we can use the Cauchy-Riemann equation and obtain that \(J = \abs{f'(z)}^2\neq 0\).

\begin{theorem}[conformal mapping]
    conformal means locally angle-preserving. We can prove that the condition for being conformal is same for being one-to-one.
\end{theorem}

Sketch of the proof. For mapping \(w=f(z)\) and two curves \(C,T\) where \(C\) is in \(x,y\)-plane and is mapped to \(T\) in \(u,v\)-plane. We consider points \(z_0\) and \(z_o+\dif z\) on curve \(C\) and their image \(w_0\) and \(w_0+\dif w\). We also define \(\theta\) and \(\eta\) to be the angles of the tangents to the curves \(C\) and \(T\) at \(z_0\) and \(w_0\).

We have the following relations
\begin{align*}
    \dif z &= \abs{\dif z} e^{i\theta}\\
    \dif w &= \abs{\dif w} e^{i\eta}\\
    f'(z_0)&= \abs{f'(z_0)}e^{i\Arg(f'(z_0))}
\end{align*}

Then we plug in above into \(\dif w = f'(z_0)\dif z\), we have
\begin{align*}
    \abs{\dif w} &= \abs{f'(z_0)}\abs{\dif z} \exp{i\Arg(f'(z_0)+\theta)}\\
    \eta &= \Arg(f'(z_0))+\theta
\end{align*}

This means that mapping \(f(z)\) stretches the infinitesimal curve segment \(\dif z\) by a factor \(\abs{f'(z_0)}\) and rotates it by an angle \(\Arg(f'(z_0))\). So to preserve angles



\begin{remark}
    conformality \(\iff\) analyticity and non-zero derivatives
\end{remark}

\begin{theorem}
    Subject to conformal mapping, Laplace's equation in the \(x,y\)-plane transforms to Laplace's equation in the \(u,v\)-plane.
\end{theorem}

With this theorem, we can simplify some problem by using conformal mapping first and solve the problem in the new plane where the conditions are in a simpler form. Here's a useful transformation: Linear Fractional (Bilinear or M\"obius) Transformation

\[w=f(z) = \frac{az+b}{cz+d}\implies z=F(w)=\frac{-dw+b}{cw-a}\]
We can use this to map three specified points \(z_1,z_2,z_3\) in the \(x,y\)-plane into three specified image points \(w_1,w_2,w_3\) in the \(u,v\)-plane, the constants are determined by solving:
\[\frac{(w-w_1)(w_3-w_2)}{(w-w_2)(w_3-w_1)}=\frac{(z-z_1)(z_3-z_2)}{(z-z_2)(z_3-z_1)}\]
And when a points is speficied to be at infinity, we set the fators that includes that point equal to \(1\). For example, if \(w_1=\infty\), then \((w-w_1)/(w_3-w_1)=1\).

This map will transform lines and circles to lines or circles.

Other conformal mappings are elementary mappings: translation, stretching, rotation, inversion. If only working in a restricted region, exponentials, trigonometrics can also be choices of conformal mapping. 

\begin{myleftlinebox}
    \begin{example}[temperature distribution, dirichilet type]
        Consider Laplace's equation \(\frac{\partial ^2 T }{\partial x^2}+\frac{\partial^2 T}{\partial y^2}=0\) subject to the boundary condition in the complex plane \(z=x+iy\)
        \begin{align*}
            T(x,y) = 0,& \abs{z-\frac{1}{2}}=\frac{1}{2}\\
            T(x,y) = 1,& \abs{z-\frac{1}{4}}=\frac{1}{4}
        \end{align*}
        It's an area between two isothermal circles that pass through the origin.
    \end{example}
    \tcbline
    We consider transform the circle boundary to line boundary. Consider mapping:
    \begin{align*}
        z_1=0 &\to w_1 = \infty\\
        z_2=\frac{1}{2} &\to w_2 = 0\\
        z_3=1 &\to w_3 = 1
    \end{align*}
    Plug in above to the formula we end up with
    \[w=f(z)=\frac{1-z}{z}\]
    And the boundaies are mapped as follows
    \begin{align*}
        T(x,y) = 0,& \abs{z-\frac{1}{2}}=\frac{1}{2}\to u=1\\
        T(x,y) = 1,& \abs{z-\frac{1}{4}}=\frac{1}{4}\to u=0
    \end{align*}
    In the new plane, we have

    \[\frac{\partial^2 \hat T^2}{\partial u^2}+\frac{\partial ^2 \hat T}{\partial v^2} = 0\] with the above boundary conditions.

    Notice that there's no change in the \(v\)-direction. So we assume \(\hat T = \hat T(u)\). Then we have \(\frac{\dif ^2\hat T}{\dif u^2}=0\implies \hat T(u,v)=Au+B\). With the boundary condition, we obtain \(\hat T (u,v)=u\).

    Next we find the inverse map to get the solution in the original plane. From \(w=\frac{1-z}{z}\) we end up with \(u(x,y)=\frac{x}{x^2+y^2}-1\), thus
    \[T(x,y)=\hat T(u(x,y),v(x,y))=\frac{x}{x^2+y^2}-1\]
\end{myleftlinebox}

\subsection{Diricilet Problem in a Circular Disk}

The Laplace's equation in acylindrical coordinates is
\[\frac{\partial^2 \phi}{\partial r^2}+\frac{1}{r}\frac{\partial \phi}{\partial r}+\frac{1}{r^2}\frac{\partial ^2\phi}{\partial \theta^2}=0\]

with the boundary conditions on \(r=R\): 
\[\phi(r,\theta)=\phi(R,\sigma)=f(\sigma)\]

\subsubsection*{method 1: seperation of variables}
We assume that \(\phi(r,\theta)=P(r)Q(\theta)\), then substute it into the equation and obtain

\[\frac{r^2}{p}\frac{\dif ^2 P}{\dif r^2}+\frac{r}{P}\frac{\dif P}{\dif r}=\frac{-1}{Q}\frac{\dif ^2Q }{\dif \theta^2}=\lambda^2\]

It's now an eigenproblem.

\begin{align*}
    r^2 \frac{\dif ^2 P}{\dif r^2}+r\frac{\dif P}{\dif r}-\lambda^2 P=0\\
    {\dif ^2Q }{\dif \theta^2}+\lambda^2 Q = 0
\end{align*}

Solving these we have \(P(r) = C_1r^\lambda + C_2 r^{-\lambda}\), \(\lambda\neq 0\); \(P(\lambda) = C_3+C_4 \log r\), \(\lambda=0\), and \(Q(\theta) = C_5\cos(\lambda\theta)+C_6\sin(\lambda\theta)\), \(\lambda\neq 0\); \(Q(\theta) = C_7+C_8\theta\), \(\lambda=0\).

So the general solution is 

\[\phi(r,\theta)=(C_1 r^\lambda + C_2 r^{-\lambda})(C_5\cos(\lambda\theta)+C_6\sin(\lambda\theta))+(C_3+C_4 \log r)(C_7+C_8\theta)\]

Notice the function must be bounded thus \(C_2=C_4=0\) and the function must be \(2\pi\)-periodic to be single-valued, thus \(C_8=0\). So we end up with

\[\phi(r,\theta) = c_1 + r^\lambda(c_3 \cos(\lambda\theta)+c_4 \sin(\lambda\theta))\]

Next to make sure the \(\cos\) and \(\sin\) terms have period \(2\pi\), we plug in \(\theta+2\pi\) and end up with \(\cos(2\pi\lambda)=1\) and \(\sin(2\pi\lambda)=0\). So the eigenvalues must be \(\lambda=1,2,3,\dots\). And thus the eigenfunction solution is 

\[\phi(r,\theta) = A_0+\sum_{n=1}^\infty r^n \Pare{A_n \cos(n\theta)+B_n\sin(n\theta)}\]

Then apply the boundary condition at \(r=R\) we have

\[f(\sigma) = \phi(R,\sigma) =  A_0+\sum_{n=1}^\infty R^n \Pare{A_n \cos(n\sigma)+B_n\sin(n\sigma)}\]

And this is just the fourier series expansion of \(f(\sigma)\) where we let

\begin{align*}
    A_0 &= \frac{1}{2\pi}    \int_0^{2\pi} f(\sigma )\dif\sigma\\
    A_n &= \frac{1}{\pi R^n} \int_0^{2\pi} f(\sigma) \cos(n\sigma)\dif\sigma\\
    B_n &= \frac{1}{\pi R^n} \int_0^{2\pi} f(\sigma) \sin(n\sigma)\dif\sigma
\end{align*}

Finally the simplified solution is:

\[\phi(r,\theta) = \frac{1}{\pi}\int_0^{2\pi} f(\sigma)\Pare{\frac{1}{2}+\sum_{n=1}^\infty \Pare{\frac{r}{R}}^n \cos(n(\sigma-\theta))}\dif\sigma\]

Next we rewrite this using the following relations:

\begin{align*}
    \sum_{n=1}^\infty \Pare{\frac{r}{R}}^n \cos(n(\sigma-\theta)) &= \Re\Pare{\sum_{n=1}^\infty \Pare{\frac{r}{R}e^{i(\sigma-\theta)}}^n }\\
    \sum_{n=1}^\infty x^n &= -1+\frac{1}{1-x}
\end{align*}

We end up with the following:

\begin{theorem}[Poisson integral formula for Dirichlet problem in a circular disk]
    Suppose the boundary condition to be \(\phi(R,\sigma)=f(\sigma)\) then the Laplace's equation have solution:

    \[\phi(r,\theta) = \frac{R^2-r^2}{2\pi} \int_0^{2\pi} \phi(R,\sigma) \frac{1}{R^2+r^2-2Rr\cos(\sigma-\theta)}\dif\sigma\]

    where \(0\leq r\leq R\) and \(\theta\in \bR\). And \(\phi(R,\sigma) = f(\sigma)\) are the values of \(\phi(r,\theta)\) on the circular boundary at \(r=R\).
\end{theorem}

With above formula we obtain following two theorem:

\begin{theorem}[Gauss' mean value theorem]
    If \(C\) is a circle that is completely within a region \(R\) in which \(\phi(z)\) is analytic, then \(\phi(z)\) at the center of the circle is the average of the value of \(\phi(z)\) along \(C\).
\end{theorem}
\begin{theorem}[Maximum and minimum modulus theorem]
    Let \(\phi(z)\) be continuous and nonconstant throughout a closed bounded region \(R\) and analytic within the interior of \(R\). Then the maximum value of \(\abs{\phi(z)}\) occurs on the boundary of \(R\). IF \(\phi(z)\) is nowhere zero in \(R\), then the minimum value of \(\abs{\phi(z)}\) must also occur on the boundary.
\end{theorem}
\subsubsection*{method 2: conformal map}

we will build a conformal map and solve the problem in upper half plane. Consider the mapping:

\begin{align*}
    z_1=-1 &\to w_1 = 0\\
    z_2=-i &\to w_2 = 1\\
    z_3=1  &\to w_3 = \infty
\end{align*}

We solve it in similar method: \(w =f(z) = i\frac{1+z}{1-z}\) and \(z=e^{i\sigma}=F(w) =\frac{w-i}{w+i}\). We can check that it indeed map the unit circle to upper half plane.

Consider some point on the unit circle \(e^{i\sigma}\) whose image in the upper half plane is \(\xi\). To obtain the formula in the new plane, we first rewrite the formula. 

\begin{align*}
    \phi(z) &=  \frac{R^2-r^2}{2\pi} \int_0^{2\pi} \phi(R,\sigma) \frac{1}{R^2+r^2-2Rr\cos(\sigma-\theta)}\dif\sigma\\
    &= \frac{1}{2\pi}  \int_0^{2\pi} \phi(Re^{1\sigma}) \frac{R^2-r^2}{\abs{Re^{i\sigma}-re^{i\theta}}^2}\dif\sigma\\
    &= \frac{1}{2\pi} \int_0^{2\pi} \phi(e^{i\sigma}) \frac{1-\abs{z}^2}{\abs{e^{i\sigma}-z}^2} \dif\sigma, \text{if } R=1
\end{align*}

Notice the unit circle is mapped to the real axis and thus: \(\hat \phi(e^{i\sigma}) = \hat\phi(F(\xi))=\hat\phi(\xi,0)\). Following this we can also derive the following.

\begin{align*}
    1-\abs{z}^2 &= 1-\abs{\frac{(u+iv)-i}{(u+iv)+i}}^2 = \frac{4v}{u^2+(a+v)^2} \\
    \abs{e^{i\sigma}-z}^2 &= \abs{\frac{\xi-i}{\xi+i}-\frac{w-i}{w+i}}^2 = 4\frac{(u-\xi)^2+v^2}{(1+\xi^2)(u^2+(1+v)^2)}\\
    ie^{i\sigma}\dif\sigma &= \Pare{\frac{1}{\xi+i}-\frac{\xi-i}{(\xi+i)^2}}\dif\xi\\
    \dif\sigma &= \frac{2}{\xi^2+1} \dif\xi
\end{align*}

Substituting these into the formula gives

\begin{align*}
    \hat\phi(u,v) &= \frac{1}{2\pi}\int_{-\infty}^\infty \hat\phi(\xi,0) \frac{4v}{u^2+(1+v)^2} \frac{1}{4}\frac{(1+\xi^2)(u^2+(1+v)^2)}{(\xi-u)^2+v^2}\frac{2}{1+\xi^2}\dif\xi\\
    &= \frac{v}{\pi}\int_{-\infty}^\infty \hat\phi(\xi,0) \frac{1}{(\xi-u)^2+v^2}\dif\xi, v>0
\end{align*}    

where \(\hat\phi(\xi,0)\) being the values on the boundary with \(v=0\).

\begin{remark}
    Fourier transform may also be used to derive this.
\end{remark}

\subsection{Application to Fluid Flows}

\begin{definition}[2D incompressible flow condition]
    Continuity equation:
    \[\nabla\cdot \bfV = \frac{\partial v_x}{\partial x}+\frac{\partial v_y}{\partial y} = 0\]
    where \(\bfV=v_x(x,y)i+v_y(x,y)j\) is the velocity vector, \(v_x(x,y)\) here denotes the velocity components and \(i\) denotes the unit vector in \(x\)-direction.
\end{definition}

\begin{definition}[Streamfunction]
    Consider streamfunction \(\psi(x,y)\) defined by 
    \[v_x = \frac{\partial\psi}{\partial y}, v_y = -\frac{\partial \psi}{\partial x}\]
    so that the flow is automatically incompressible. We also call the contours of constant streamfunction streamlines. We can also see that the slop of the streamline is:

    \[\frac{\dif y}{\dif x} = \frac{-\partial\psi/\partial x}{\partial \psi/\partial y} = \frac{v_y}{v_x}\]
    it's the direction of the velocity vector thus streamlines can indicate the local flow direction (velocity direction).
\end{definition}

\begin{definition}[vorticity]
    Vorticity is the curl of the velocity field, in 2D cartesian coordinates, we write
    \[\omega = \nabla\times \bfV = \frac{\partial v_y}{\partial x}-\frac{\partial v_x}{\partial y}\]
    And in 3D cartesian coordinates, we write
    \[\nabla\times\bfV = \Pare{\frac{\partial v_z}{\partial y}-\frac{\partial v_y}{\partial z}, \frac{\partial v_x}{\partial z}-\frac{\partial v_z}{\partial x}, \frac{\partial v_y}{\partial x}-\frac{\partial v_x}{\partial y}}\]
    If the flow is irrotational, we have \(\omega=0\) in which case there exists a scalar function \(\phi(x,y)\) for which \(\nabla\times (\nabla\phi)=0\).
\end{definition}

\begin{remark}
    The curl of the gradient of any scalar function is zero. In 2D case we have
    \[\nabla\times (\nabla \phi) = \frac{\partial (\partial \phi/\partial y)}{\partial x} - \frac{\partial(\partial \phi/\partial x)}{\partial y}=0\]
    And in 3d case we have
    \[\nabla\times (\nabla \phi) = \Pare{\frac{\partial ^2\phi}{\partial y\partial z}-\frac{\partial ^2\phi}{\partial z\partial y},\frac{\partial ^2\phi}{\partial z\partial x}-\frac{\partial ^2\phi}{\partial x\partial z},\frac{\partial ^2\phi}{\partial x\partial y}-\frac{\partial ^2\phi}{\partial y\partial x}} =0\]
\end{remark}

\begin{definition}[velocity potential]
    We write the velocity vector as a gradient of a scalar function \(\phi(x,y)\) according to \(\bfV = \nabla \phi\) where the components are \( v_x = \frac{\partial\phi}{\partial x}\). We call this scalar function \(\phi\) the velocity potential.
\end{definition}

\begin{corollary}
    From the incompressible condition, we have \(\nabla\cdot\bfV = 0\) and thus
    \[\frac{\partial^2 \phi}{\partial x^2}+\frac{\partial^2\phi}{\partial y^2} = 0\]
    Also the streamfunction has the property (2D case):
    \[w=\nabla\times \bfV = \frac{\partial v_y}{\partial x}-\frac{\partial v_x}{\partial y} = -\frac{\partial^2\psi}{\partial x^2}- \frac{\partial^2\psi}{\partial y^2}\]
\end{corollary}

\begin{definition}[potential flow]
    Ideal flow: two-dimensional, incompressible, inviscid, irrotational.
\end{definition}

\begin{corollary}
    For the ideal flow, the streamfunction, velocity potential are homonic functions, and are harmonic conjugates of one another, i.e.:
    \[\nabla^2 \phi = 0, \nabla^2\psi = -\omega=0, v_x = \frac{\partial \phi}{\partial x}=\frac{\partial \psi}{\partial y}, v_y = \frac{\partial \phi}{\partial y}=-\frac{\partial \psi}{\partial x}\]
    And thus here we define the complex potential function:
    \[\Phi(z) = \phi(x,y)+i\psi(x,y)\]
    where \(\Phi(z)\) is an analytic function. Correspondingly we have the complex velocity by taking derivatives.
    \[\frac{\dif\Phi}{\dif z} = \frac{\partial \phi}{\partial x}+i\frac{\partial psi}{\partial x} = \frac{\partial \psi}{\partial y}-i\frac{\partial \phi}{\partial y} = v_x-iv_y = \bar W(z)\]
    The velocity potential drives the flow along the streamlines from higher to lower potential.
    We also define the Stagnation points in the potential flow where the velocity \(\bfV = 0\), or \(\bar W = \frac{\dif \Phi}{\dif z} = 0\).
\end{corollary}

In polar coordinates, we have the following results:

\begin{align*}
    \bfV &= v_r(r,\theta)\bfe_r + v_\theta(r,\theta)\bfe_\theta\\
    v_r &= \frac{\partial \phi}{\partial r} = \frac{1}{r}\frac{\partial \psi}{\partial \theta}\\
    v_\theta &= \frac{1}{r}\frac{\partial \phi}{\partial \theta} = -\frac{\partial\psi}{\partial r}\\
    \nabla^2\phi &= \frac{1}{r} \frac{\partial}{\partial r}\Pare{r\frac{\partial \phi}{\partial r}} + \frac{1}{r^2} \frac{\partial ^2\phi}{\partial \theta^2} = 0\\
    \bar W(z) &= \frac{\dif \Phi}{\dif z} = (v_r-iv_\theta)e^{-i\theta}
\end{align*}

Next, we define basic flows by specifying their complex potentials. 

Uniform flow: of speed \(U\) at an angle \(\alpha\) with positive \(x\)-axis.

\begin{align*}
    \Phi(z) &= Ue^{-i\alpha}z\\
    \bar W(z) &= Ue^{-i\alpha}=U(\cos \alpha-i\sin\alpha)\\
    v_x &= U\cos \alpha\\
    v_y &= U\sin\alpha
\end{align*}

Sources and sinks: \(Q\) being the volumetric flow rate. If \(Q>0\), it's a source and sink if negative.

\begin{align*}
    \Phi(z) &= \frac{Q}{2\pi} \log z\\
    \Phi(z) &= \frac{Q}{2\pi}( \log r+i\theta)\\
    \phi(x,y) &= \Re(\Phi)=\frac{Q}{2\pi} \log r\\
    \psi(x,y) &= \Im(\Phi) =\frac{Q}{2\pi}\theta\\
    v_r &= \frac{\partial \phi}{\partial r}= \frac{1}{r}\frac{\partial\psi}{\partial \theta}= \frac{Q}{2\pi r}\\
    v_\theta &= \frac{1}{r}\frac{\partial \phi}{\partial \theta} = -\frac{\partial \psi}{\partial r} = 0
\end{align*}

Point Vortex: \(\Gamma\) being the circulation strength. If \(\Gamma>0\), it's counterclockwise rotation.

\[\Phi(z) = -i\frac{\Gamma}{2\pi} \log z\]

Doublet: obtained by placing a source and a sink a distance \(\epsilon\) apart and taking \(\epsilon \to 0\) in a particular manner. The resulting velocity potential is \(\Phi = \frac{S}{z}\) where \(S\) is the strength of the doublet.

Corner and Wedge flows: 

\begin{align*}
    \Phi(z) &= \frac{A}{n}z^n, n\geq \frac{1}{2}\\
    \phi(r,\theta) &= \frac{A}{n} r^n \cos(n\theta)\\
    \psi(r,\theta) &= \frac{A}{n} r^n \sin(n\theta)\\
    \bar W(z) &= Az^{n-1} = Ar^{n-1} e^{in\theta}e^{-i\theta}\\
    v_r(r,\theta) &= Ar^{n-1} \cos(n\theta)\\
    v_\theta(r,\theta) &=- Ar^{n-1} \sin(n\theta)
\end{align*}

Next we can consider using conformal mapping on the flows.

\begin{align*}
    \bar W(z) &= \frac{\dif\Phi}{\dif z}\\
    \hat{\bar W}(w) &= \frac{\dif \hat \Phi}{\dif w}\\
    \Phi&=\hat\Phi\\
    \frac{\dif\Phi}{\dif z} &= \frac{\dif \hat\Phi}{\dif w}\frac{\dif w}{\dif z} = \frac{\dif \hat\Phi}{\dif w} f'(z)\\
    \hat W(z) &= f'(z) \hat{\bar W}(w)
\end{align*}

The magnitudes of the velocities differ by a factor of \(f'(z)\) in the \(z\)- and \(w\)-planes, while the values of \(\phi\) and \(\psi\) are the same at image points, i.e. \(\Phi(z) = \hat\Phi(w)\).

The critical points of the mapping, i.e. where \(f'(z) = 0\) are stagnation points in the physical domain i.e. \(\bar W(z)=0\).

Two useful general purpose mappings:

Schwarz-Christoffel Transformation: it maps the interior of a polygon with \(n\) vertices in the \(z\)-plane to the upper half of the \(w\)-plane. The inverse transformation is

\[z=F(w)=A\int (w-u_1)^{k_1}(w-u_2)^{k_2}\cdots (w-u_n)^{k_1n}\dif w+B\]

where \(k_i=\frac{\alpha_i}{\pi}-1\)

Joukowski Transformation: the mapped image is a cylinder. We have

\begin{align*}
    z &= F(w)=w+\frac{c^2}{w}\\
    w &= f(z)=\frac{1}{2}z\pm {\Pare{\frac{z}{2}}^2-c^2}^{1/2}
\end{align*}

\section{Integration in the complex plane and residue theory}
\section{Variational Methods}

We use differential calculus to find extrema of functions, and use variatinoal calculus to find extrema of functionals, say a definite integral involving an unknown function and its derivatives The result should be a function.

\subsection{Integration by parts}

starting from product rule of two functions \(p(x)\) and \(q(x)\), \(\frac{\dif}{\dif}(pq) = p\frac{\dif q}{\dif x}+q\frac{\dif p}{\dif x}\) and thus integration
\[pq = \int p\frac{\dif q}{\dif x} \dif x+\int q\frac{\dif q}{\dif x}\dif x\implies \int p\dif q = pq-\int q\dif p\]

Alternatively, for a definite integral over \(x_0\leq x\leq x_1\), we have
\[\int_{x_0}^{x_1} p\dif q = \given{pq}_{x_0}^{x_1}-\int_{x_0}^{x_1} q\dif p\]

In the two-dimensional context, integration by parts is replaced by the divergence theorem.

\begin{theorem}[2D divergence theorem]
    \[\iint_A \nabla\cdot \bfv\dif A = \oint_C\bfv\cdot \bfn\dif s\]
    where \(\bfv\) is any vector, \(\bfn\) is the outward facing unit normal to the curve \(C\) bounding the area \(A\). Notice that \(\dif s\) proceeds in the counterclockwise direction along the boundary.
\end{theorem}

\begin{corollary}
    Consider apply divergence theorem to \(\bfv = \psi\nabla\phi\) where \(\phi(x,y)\) and \(\phi(x,y)\) are arbitrary 2D scalar functions. Then
    \begin{align*}
        \nabla \cdot\bfv &= \nabla\cdot(\psi\nabla\phi)=\psi\nabla^2\phi+\nabla\psi\cdot \nabla\phi\\
        \bfv\cdot\bfn &= \psi (\nabla\phi\cdot \bfn)
    \end{align*}
    Then in the Green's first theorem, we have
    \[\iint_A\nabla\psi\cdot\nabla\phi\dif A = \oint_C\psi (\nabla\phi\cdot\bfn)\dif s-\iint_A\psi\nabla^2\phi\dif A\]
    In one dimension, with \(\psi=\psi(x)\) and \(\phi=\phi(x)\) we have \(\nabla=\frac{\dif}{\dif x}i\) and \(\bfn_0=-i\), \(\bfn_1=i\) and \(\dif A = \dif x\) thus
    \[\int_{x_0}^{x_1} \frac{\dif\psi}{\dif x}\frac{\dif\phi}{\dif x}\dif x = \given{\psi\frac{\dif\phi}{\dif x}_{x=x_1}}-\given{\psi\frac{\dif\phi}{\dif x}}_{x=x_0} -\int_{x_0}^{x_1} \psi\frac{\dif^2\phi}{\dif x^2}\dif x\]
    Consider \(\psi(x)=q(x)\) and \(\phi(x)=\int p(x)\dif x\) we have the integration by parts formula in 1D. For 2D case, we have \(\bfv=P(x,y)i+Q(x,y)j\), \(\bfn=\frac{\dif y}{\dif s}i-\frac{\dif x}{\dif s}j\) and \(\nabla=\frac{\partial}{\partial x}i+\frac{\partial }{\partial y}j\). The formula goes to:
    \[\iint_A \Pare{\frac{\dif P}{\dif x}+\frac{\dif Q}{\dif y}}\dif x\dif y=\oint_C\Pare{P\frac{\dif y}{\dif x}-Q\frac{\dif x}{\dif x}}\dif s=\oint_C (P\dif y-Q\dif x)\]
\end{corollary}

\begin{lemma}[fundamental lemma of the calculus of variations]
    If \(g(x)\) is a continuous function in \(x_0\leq x\leq x_1\) and if 
    \[\int_{x_0}^{x_1} g(x)h(x)\dif x=0\]
    where \(h(x)\) is an arbitrary function in the same interval with \(h(x_0)=h(x_1)=0\) then \(g(x)=0\) at every point in the inverval \(x_0\leq x\leq x_1\).
\end{lemma}

can be proved by contradiction.

\subsection{Calculus of variations}

\begin{definition}[Varation of a function]
    consider the functional \(I[u(x)]\) of the function \(u(x)\) of the form 
    \[I[u(x)] = \int_{x_0}^{x_1} F(x,u(x),u'(x))\dif x\]

    where \(u(x)\) is smooth and is the family of curves passing through the speficied end points \(u(x_0)=u_0\), \(u(x_1)=u_1\). 
    
    Then consider \(\hat u(x) = u(x) +\Pare{\given{\frac{\partial u}{\partial \epsilon}}_{\epsilon=0}}\epsilon+O(\epsilon^2)\). And the variation of \(u\) is
    \[O(\epsilon) = \delta u = \Pare{\given{\frac{\partial u}{\partial \epsilon}}_{\epsilon=0}}\epsilon:= \epsilon \eta(x)\]
    where \(\eta(x_0) = \eta(x_1)=0\).
\end{definition}

Next we derive the definition of functional derivative, starting from 
\begin{align*}
    \given{\frac{\dif I[u(x)]}{\dif\epsilon}}_{\epsilon=0} &= \lim_{\epsilon\to 0} \frac{I[\hat u(x)]-I[u(x)]}{\epsilon}\\
    &= \lim_{\epsilon\to 0} \frac{I[u+\epsilon\eta]-I[u]}{\epsilon}\\
    &= \lim_{\epsilon\to 0}\frac{1}{\epsilon} \Pare{ \int_{x_0}^{x_1} F(x,u+\epsilon \eta,u'+\epsilon\eta')\dif x-\int_{x_0}^{x_1} F(x,u,u')\dif x }
\end{align*}

Then using Talor series expansion for the first integral, we have
\begin{align*}
    \given{\frac{\dif I}{\dif\epsilon}}_{\epsilon=0} &=\lim_{\epsilon\to 0}\frac{1}{\epsilon} \Pare{ \int_{x_0}^{x_1} F(x,u,u') + \epsilon\eta\frac{\partial F}{\partial u}+\epsilon \frac{\dif \eta}{\dif x}\frac{\partial F}{\partial u'}+O(\epsilon^2)\dif x -\int_{x_0}^{x_1} F(x,u,u')\dif x }\\
    &= \int_{x_0}^{x_1} \eta\frac{\partial F}{\partial u}+\frac{\dif\eta}{\dif x} \frac{\partial F}{\partial u'}\dif x
\end{align*}
Applying the integration by parts to the second part by letting \(p=\frac{\partial F}{\partial u'}\) and \(q=\eta\), we end up with
\[\given{\frac{\dif I}{\dif\epsilon}}_{\epsilon=0} = \int_{x_0}^{x_1} \eta\frac{\partial F}{\partial u} \dif x + \given{\Pare{\frac{\partial F}{\partial u'}\eta}}_{x_0}^{x_1}- \int_{x_0}^{x_1} \eta\frac{\dif }{\dif x} \Pare{\frac{\partial F}{\partial u'}}\dif x =\int_{x_0}^{x_1}\Pare{\frac{\partial F}{\partial u}-\frac{\dif }{\dif x} \Pare{\frac{\partial F}{\partial u'}}} \eta\dif x\]

Thus we define the functional derivative

\begin{definition}[The functional derivative]
    
    
    \[\frac{\delta I}{\delta u} = \frac{\partial F}{\partial u}-\frac{\dif }{\dif x} \Pare{\frac{\partial F}{\partial u'}}\]
    so that we have \(\given{\frac{\dif I}{\dif\epsilon}}_{\epsilon=0} = \int_{x_0}^{x_1}\frac{\delta I}{\delta u}  \eta\dif x\)
\end{definition}

At the extrema of the functional \(I[u(x)]\), we would expect that \(\given{\frac{\dif I}{\dif\epsilon}}_{\epsilon=0}=0\). Nest we find the stationary function of the functional \(I\). We write

\begin{align*}
    \frac{\dif I}{\dif \epsilon}&=\frac{\dif}{\dif\epsilon}\int_{x_0}^{x_1} F\dif x=\int_{x_0}^{x_1} \frac{\dif F}{\dif\epsilon}\dif x\\
    \frac{\dif F}{\dif \epsilon} &= \frac{\partial F}{\partial x}\frac{\dif x}{\dif \epsilon} + \frac{\partial F}{\partial u}\frac{\dif u}{\dif \epsilon} + \frac{\partial F}{\partial u'}\frac{\dif u'}{\dif \epsilon}
\end{align*}

Notice that \(x\) is independent of \(\epsilon\) so the first term vanishes. Next using the definition of \(u(x) = \hat u(x)-\epsilon \eta(x)\) we have

\[ \frac{\dif u}{\dif \epsilon} = -\eta(x), \frac{\dif u'}{\dif \epsilon} = -\eta'(x)\]

and thus to make sure \(\frac{\dif I}{\dif\epsilon}=0\), we want

\[\int_{x_0}^{x_1} \Pare{\frac{\partial F}{\partial u}\eta + \frac{\partial F}{\partial u'}\eta'}\dif x=0\]
Using integrating by parts again, the second term goes to

\[\int_{x_0}^{x_1}\frac{\partial F}{\partial u'}\eta'\dif x = \given{\Pare{\frac{\partial F}{\partial u'}\eta}}_{x_0}^{x_1} - \int_{x_0}^{x_1} \frac{\dif}{\dif x}\Pare{\frac{\partial F}{\partial u'}}\eta\dif x\]

\begin{definition}[Natural boundary condition]
    To continue, here we add the natural boundary condition for integrand \(F(x,u,u')\). It's
    \[\given{\frac{\partial F}{\partial u'}}_{x_0}=\given{\frac{\partial F}{\partial u'}}_{x_1}=0\]
\end{definition}

\begin{remark}
    If \(u(x)\) are fixed at the end points, then \(\delta u=0\) at the end points so we don't need this natural boundary condition.
\end{remark}

Now the first term vanishes, we can plug it in and end up with

\[\int_{x_0}^{x_1} \Pare{\frac{\partial F}{\partial u} -\frac{\dif}{\dif x}\Pare{\frac{\partial F}{\partial u'}}}\eta\dif x=0\]

notice \(\eta\) is an arbitrary function thus we end up with the following equation:

\begin{definition}[Euler-Lagrange equation]
    For an integrand of the form \(F=F(x,u,u')\), the Euler equation is:
    \[\frac{\partial F}{\partial u} -\frac{\dif}{\dif x}\Pare{\frac{\partial F}{\partial u'}}=0\]
    It's necessary, but not sufficient condition for an extremum of \(I[u(x)]=\int F\dif x\).
\end{definition}



\begin{proposition}
    Another form obtained by expanding the terms using total derivative:
    \[\frac{\partial F}{\partial u} - \Pare{\frac{\partial ^2 F}{\partial u'\partial x}+\frac{\partial ^2 F}{\partial u'\partial u}\frac{\dif u}{\dif x}+\frac{\partial ^2 F}{\partial u'^2 }\frac{\dif^2u}{\dif x^2}}=0\]
\end{proposition}

\begin{corollary}
    A special case when \(F=F(u,u')\) that does not explicitly depend on the independent variable \(x\). We can do the following: Supposing \(x(u)\) exists.
    \begin{align*}
        I[u(x)] &= \int_{x_0}^{x_1} F(u,u') \dif x\\
        &= \int_{u_0}^{u_1} F\Pare{u, \Pare{\frac{\dif x}{\dif u}}^{-1}}\frac{\dif x}{\dif u}\dif u\\
        &= \int_{u_0}^{u_1}  \bar F(u,x')\dif u\\
        \bar F(u,x') &= x'F(u,(x')^{-1})
    \end{align*}
    Then since here \(\partial \bar F/\partial x=0\) the Euler equation is reduced to 
    \[-\frac{\dif }{\dif u}\Pare{\frac{\partial \bar F}{\partial x'}}=0\implies \frac{\partial \bar F}{\partial x'}=c\]
    It can be alternatively derived in another way and obtained in a different form, with a different name: Beltrami identity:
    \[F-\frac{\partial F}{\partial u'}\frac{\dif u}{\dif x}=c\]
\end{corollary}

\begin{corollary}
    This form can also be obtained when encountering end points on different curves. Supposing \(u(0)=f(x_0)\) and \(u(1)=g(x_1)\). We can actually show that the Euler equation is of the usual form except at the ends, where we have the transversality conditions:
    \begin{align*}
        F+(f'-u')\frac{\partial F}{\partial u'}=0, &x=x_0\\
        F+(g'-u')\frac{\partial F}{\partial u'}=0, &x=x_1
    \end{align*}
\end{corollary}

\subsection{Summary on notation and definitions}

\begin{itemize}
    \item the functional of interest: \(I[u(x)] = \int_{x_0}^{x_1} F(x,u,u')\dif x\)
    \item variation of function \(u\): \(\delta u=O(\epsilon) = \epsilon\eta(x)\) \item especially here we have \(\delta u'=(\delta u)'\) meaning the variational operator \(\delta\) and the differential operator \(\dif/\dif x\) are commutative if the differentiation is wrt an independent variable.
    \item variation of functional \(I[u]\):
    \[\delta I = \delta \int_{x_0}^{x_1} F(x,u,u')\dif x = \int_{x_0}^{x_1}\delta F(x,u,u') \dif x=0\]
    \item total differential of \(F(x,u,u')\):
    \[\dif F = \frac{\partial F}{\partial x}\dif x+\frac{\partial F}{\partial u}\dif u+\frac{\partial F}{\partial u'}\dif u'\]
    \item (total) variation of \(F(x,u,u')\):
    \[\delta F = \frac{\partial F}{\partial x}\delta x+\frac{\partial F}{\partial u}\delta u+\frac{\partial F}{\partial u'}\delta u'=\frac{\partial F}{\partial u}\delta u+\frac{\partial F}{\partial u'}\delta u'\]
    \item functional derivative: As \(\given{\dfrac{\dif I}{\dif\epsilon}}_{\epsilon=0}=\displaystyle\lim_{\epsilon\to0}\frac{I[u+\delta u]-I[u]}{\epsilon}=\int_{x_0}^{x_1}\Pare{\frac{\partial F}{\partial u}-\frac{\dif }{\dif x}\Pare{\frac{\partial F}{\partial u'}}}\eta\dif x\), we define
    \[\frac{\delta I}{\delta u}=\frac{\partial F}{\partial u}-\frac{\dif }{\dif x}\Pare{\frac{\partial F}{\partial u'}}\]
    \item Euler equation for integrand \(F(x,u,u')\): \(\frac{\partial F}{\partial u}-\frac{\dif }{\dif x}\Pare{\frac{\partial F}{\partial u'}}=0\)
\end{itemize}

Some fun fact about variational principles
\begin{itemize}
    \item straight line: shortest distance between two points
    \item circle: shortest curve enclosing a given area (Dido's problem)
    \item sphere: smallest area enclosing a given volume
    \item isosceles triangle: a triangle having the shortest perimeter with a given
    base line and area
\end{itemize}

In higher dimension, the results are:

\begin{itemize}
    \item Euler equation for \(F(x,u,u',u'')\)
    \[\frac{\partial F}{\partial u}-\frac{\dif }{\dif x}\Pare{\frac{\partial F}{\partial u'}}+\frac{\dif^2 }{\dif x^2}\Pare{\frac{\partial F}{\partial u''}}=0\]
    \item expanding results in a 4-th order ode, which gives rise to the following
    \item two possible boundary condition sets: first is to specify \(u\) and \(u'\) at the end points so that \(\delta u\) and \(\delta u'\) at \(x_0,x_1\) are both zero. Or use natural boundary conditions, that at \(x_0,x_1\) we have
    \[\frac{\dif }{\dif x}\Pare{\frac{\partial F}{\partial u''}}-\frac{\partial F}{\partial u'}=0, \frac{\partial F}{\partial u''}=0\]
    \item for \(F(x,u,u',\dots,u^{(n)})\), the Euler equation is
    \[\frac{\partial F}{\partial u}-\frac{\dif }{\dif x}\Pare{\frac{\partial F}{\partial u'}}+\frac{\dif^2 }{\dif x^2}\Pare{\frac{\partial F}{\partial u''}}-\cdots+(-1)^n \frac{\dif^n }{\dif x^n}\Pare{\frac{\partial F}{\partial u^{(n)}}} =0\]
\end{itemize}

We'll move to function of more independent variables. 
\subsection{Functional of Two Independent Variables}

Generally, we consider the following functional where \(u(x,y)\) depends on two independent variables \(x\) and \(y\).
\[I[u(x,y)] = \iint_A F(x,y,u,u_x,u_y)\dif x\dif y\]

Here subscripts denote partial differential instead of components. By setting the variation of the functional equal to \(0\), the Euler Equation is obtained via the following steps:

\begin{align*}
    \delta I[u(x,y)] &= 0\\
    &=\iint_A \frac{\partial F}{\partial u}\delta u+\frac{\partial F}{\partial u_x}\delta (u_x) +\frac{\partial F}{\partial u_y}\delta (u_y) \dif x\dif y\\
    &=\iint_A \frac{\partial F}{\partial u}\delta u+\frac{\partial F}{\partial u_x}(\delta u)_x +\frac{\partial F}{\partial u_y}(\delta u)_y \dif x\dif y\\
    &=\iint_A \frac{\partial F}{\partial u}\delta u+ \SB{\frac{\partial }{\partial x}\Pare{\frac{\partial F}{\partial u_x}\delta u}-\frac{\partial}{\partial x}\Pare{\frac{\partial F}{\partial u_x}}\delta u} +\SB{\frac{\partial }{\partial y}\Pare{\frac{\partial F}{\partial u_y}\delta u}-\frac{\partial}{\partial y}\Pare{\frac{\partial F}{\partial u_y}}\delta u}  \dif x\dif y\\
    &=\iint_A \frac{\partial F}{\partial u}\delta u+ \SB{\frac{\partial }{\partial x}\Pare{\frac{\partial F}{\partial u_x}\delta u}+\frac{\partial }{\partial y}\Pare{\frac{\partial F}{\partial u_y}\delta u}} -\SB{\frac{\partial}{\partial x}\Pare{\frac{\partial F}{\partial u_x}}\delta u+\frac{\partial}{\partial y}\Pare{\frac{\partial F}{\partial u_y}}\delta u}  \dif x\dif y\\
\end{align*}

Next we use the divergence theorem on the second term above. Let \(P=\dfrac{\partial F}{\partial u_x}\delta u\) and \(Q=\dfrac{\partial F}{\partial u_y}\delta u\), then we have
\[\iint_A \Pare{\frac{\partial F}{\partial u} -\Pare{\frac{\partial}{\partial x}\Pare{\frac{\partial F}{\partial u_x}}+\frac{\partial}{\partial y}\Pare{\frac{\partial F}{\partial u_y}}} }\delta u \dif x\dif y +\oint_C \Pare{P\frac{\dif y}{\dif x}-Q\frac{\dif x}{\dif x}}\dif s=0 \]

where the second term can be written explicitly as 

\[\oint_C \Pare{\frac{\partial F}{\partial u_x}\frac{\dif y}{\dif x}-\frac{\partial F}{\partial u_y}\frac{\dif x }{\dif s}}\delta u\dif s\]

This results in the following Euler equation.

\begin{theorem}
    Euler's equation for integrand \(F(x,y,u,u_x,u_y)\) is
    \[\frac{\partial F}{\partial u} -\Pare{\frac{\partial}{\partial x}\Pare{\frac{\partial F}{\partial u_x}}+\frac{\partial}{\partial y}\Pare{\frac{\partial F}{\partial u_y}}} =0\]
    So that the first integration above goes to \(0\). For the second integration, either we speficy \(u(x,y)\) at the boundary so that \(\delta u=0\), or we apply the natural boundary condition:
    \[\frac{\partial F}{\partial u_x}\frac{\dif y}{\dif s}-\frac{\partial F}{\partial u_y}\frac{\dif x}{\dif s}=0\]
    In vector form, we write the Euler equation as:
    \[\frac{\partial F}{\partial u}-\nabla\cdot \frac{\partial F}{\partial \nabla u}=0\]
    The natural boundary condiiton is:
    \[\bfn\cdot \frac{\partial F}{\partial\nabla u}=0\]
\end{theorem}

\begin{remark}[arc length formula in 2D]
    smooth function \(u(x,y)\) enclosing an area \(A\) have area:
    \[\iint_A \sqrt{1+\Pare{\frac{\partial u}{\partial x}}^2+\Pare{\frac{\partial u}{\partial y}}^2}\dif x\dif y\]\
    A generalization to the length formula in 1D
    \[\int_{x_0}^{x_1} \sqrt{1+\Pare{\frac{\dif u}{\dif x}}^2}\dif x\]
\end{remark}

\subsection{Inverse problem}

Convert known differential equations into their equivalent variational form. Here's the general steps we can follow.

\begin{enumerate}
    \item Write the differential equation in operator form: \(\cL u=0\)
    \item times \(\delta u\) then take integral and let it be zero: \(\int_\Omega \cL u\delta u\dif \bfx=0\)
    \item using integration by parts and other formula to finally obtain \(\delta I[u]=0\)
\end{enumerate}

Here's an example of the 2D rectangle Dirichlet problem. We begin with the differential equation:

\[\Delta u=\nabla^2 u=\frac{\partial ^2 u}{\partial x^2}+\frac{\partial u^2}{\partial y^2}=0\]

along with the specified boundary condition on the boundary \(C\): \(u(x,y)=u_0(x,y)\).

\begin{equation*}
    \begin{split}
        0 &=  \iint_\Omega \Delta u\delta u\dif x\dif y\\
        &= \iint_\Omega u_{xx} \delta u\dif x\dif y + \iint_\Omega u_{yy} \delta u\dif x\dif y\\
        &= \int_{y_0} ^{y_1} \Pare{\int_{x_0} ^{x_1} \delta u\dif u_x} \dif y+\int_{x_0} ^{x_1} \Pare{\int_{y_0} ^{y_1} \delta u\dif u_y} \dif x\\
        &= \int_{y_0} ^{y_1} \Pare{0- \int_{x_0} ^{x_1} u_x \dif \delta u}\dif y + \int_{x_0} ^{x_1} \Pare{0- \int_{y_0} ^{y_1} u_y \dif \delta u}\dif x\\
        &=  -\int_{y_0} ^{y_1} \int_{x_0} ^{x_1} u_x\delta u_x\dif x \dif y- \int_{x_0} ^{x_1}\int_{y_0} ^{y_1}  u_y\delta u_y\dif y\dif x\\
        &= -\frac{1}{2}\int_\Omega \delta\Pare{u_x^2+u_y^2}\dif x\dif y\\
        &= \delta \int_\Omega \nabla u\cdot \nabla u \dif x\dif y
    \end{split}
\end{equation*}

\begin{remark}
    This process can always be done on linear, self-adjoint differential operators.
\end{remark}

\begin{remark}
    For poisson equation: \(\cL u = f\) similarly we have the inverse problem solution: 
    \[ I[u] = \iint_\Omega \Pare{\Pare{\frac{\partial u}{\partial x}}^2+\Pare{\frac{\partial u}{\partial y}}^2+2fu }\dif x\dif y \]
\end{remark}

\subsection{Functionals of Two Dependent Variables}

\begin{theorem}
    Euler's Equation for \(F(x,y,u,v,u_x,v_x,u_y,v_y)\) are
    \begin{align*}
        \frac{\partial F}{\partial u}-\frac{\partial}{\partial x}\Pare{\frac{\partial F}{\partial u_x}}-\frac{\partial}{\partial y}\Pare{\frac{\partial F}{\partial u_y}}=0\\
        \frac{\partial F}{\partial v}-\frac{\partial}{\partial x}\Pare{\frac{\partial F}{\partial v_x}}-\frac{\partial}{\partial y}\Pare{\frac{\partial F}{\partial v_y}}=0
    \end{align*}

    Either we use fixed boundary conditions or the following natural boundary conditions:

    \begin{align*}
        \frac{\partial F}{\partial u_x}\frac{\dif y}{\dif s}-\frac{\partial F}{\partial u_y}\frac{\dif x}{\dif s}=0\\
        \frac{\partial F}{\partial v_x}\frac{\dif y}{\dif s}-\frac{\partial F}{\partial v_y}\frac{\dif x}{\dif s}=0
    \end{align*}
\end{theorem}

\subsection{Constrained Functionals}

\subsubsection*{Integral constraints}

Use Lagrange multiplier method, notice the multiplier hereis treated as a number not a function.


Consider the functional \(I[u]=\int_{x_0}^{x_1} F(x,u,u')\dif x\) subject to \(\int_{x_0}^{x_1} G(x,u,u')\dif x=C\). We build \(J[u]=\int_{x_0}^{x_1} \tilde F(\lambda, x,u,u')\dif x\) where \(\tilde F(\lambda, x,u,u')=F+\lambda G\) and then solve \(\delta J[u]=0\). We call \(\tilde F\) the augmented integrand. Multiple constraints are treated similarly. We write \(\tilde F=F+\sum_i \lambda_i G_i\). The Euler's equation is simimlar: 

\[\frac{\partial \tilde F}{\partial u}-\frac{\dif }{\dif x}\Pare{\frac{\partial \tilde F}{\partial u'}}=0\]

One specific type of problems: Sturm-Liouville problems, where the integrand has the form: \(F(x,u,u')=q(x)u^2-p(x)(u')^2\), and the constraint has the form \(G(x,u,u')=r(x)u^2\) and \(C=1\). Three functions \(r,p,q\) are known. The simplified Euler's equation is:

\[q(x) u +\lambda r(x)u-\frac{\dif }{\dif x}\Pare{-p(x)u'}=0\]

That's how this type of problem got its name for this is a Sturm-Liouville equation. Notice here \(\lambda\) is an eigenvalue of the Sturm-Liouville differential operator.

\subsubsection*{Algebraic and differential constraints}

Consider the functional \(I[u,v]=\int_{x_0}^{x_1} F(x,u,v,u',v')\dif x\) subject to \(\phi(u,v,u',v')=0\).

Notice in this case to use the multiplier, it's no longer a number but a continuous function of the independent variable(s). We write the augmented integrand as

\[\tilde F = F+\lambda(x)\phi\]

The Euler's equation is the same: \(\frac{\partial \tilde F}{\partial u}-\frac{\dif }{\dif x}\Pare{\frac{\partial \tilde F}{\partial u'}}=0\) and \(\frac{\partial \tilde F}{\partial u}-\frac{\dif }{\dif x}\Pare{\frac{\partial \tilde F}{\partial v'}}=0\).

\section{Hamilton's Principle}

\subsection{Hamilton's Principle for Discrete Systems}

Consider the system to be a single \(O(1)\)-sized rigid particle, a point mass \(m\) acted upon by a force \(\bff(t)\) and \(\bfr(t)\) is the position vector of the system and \(t\) is independent variable.

\begin{theorem}[Hamilton's Variational Principle for a Single Particle]
    In order to determine the actual trajectory \(\bfr(t)\) of the system over the finite time \(t_0\leq t\leq t_1\), Hamilton's principle asserts that the trajectory must be such that \(\bfr(t)\) is a stationary funciton of the functional
    \[\int_{t_0}^{t_1} \delta T+\bff\cdot\delta \bfr \dif t=0\]

    \begin{itemize}
        \item For the first term, \(T\) is the kinetic energy of the system and it equals \(T=\frac{1}{2}m\dot \bfr^2:=\frac{1}{2} m\bfv^2\). It's a scalar.
        \item For the second term, \(\bff\cdot\delta \bfr \) is the virtual work owing to the application of the actual resultant external force \(\bff\) acting on the system through the infinitesimal virtual displacement \(\delta\bfr\). Some examples including the gravitational potential, electrostatic potential, and deformation potential.
    \end{itemize}
\end{theorem}

\begin{remark}
    Notice the principle only asserts that \(\bfr(t)\) is the stationary solution, thus not necessaryily a minimum.
\end{remark}


By variation, the Euler's equation for this system is the newoton's second law.

\[m\ddot \bfr-\bff=0\]

\begin{remark}
    There's a subtle difference betweent the differential form (local law) and the variatonal form (global law). Newton's second law requires two initial conditions at the initial time, position and velocity, thus a initial-value problem. And the Hamilton's principle requires the positions of initial and ending time, thus a boundary-value problem.
\end{remark}

Next we consider the case when nonconservative forces is involved. 

\begin{definition}[Conservative forces]
    \(\bff_c\), forces that the virtual work done \(W_c\) is independent of the path and only depends on the initial and final positions. Such work is recoverable, and the process is reversible.

    If the work done depend on the path, then the force is nonconservative.
\end{definition}

\begin{definition}[Mechanical energy]
    The capacity to do work.
    \begin{itemize}
        \item kinetic energy: from its motion
        \item potential energy: from its position in a conservative force field.
    \end{itemize}
    A system is conservative when its mechanical energy is conserved.
\end{definition}

Thus far, we write 

\begin{align*}
    \bff\cdot\delta\bfr&=\bff_c\cdot\delta \bfr+\bff_{nc}\cdot\delta \bfr\\
    W_c &= \int_{\bfr_1}^{\bfr_2} \bff_c\cdot \delta \bfr = \int_{\bfr_1}^{\bfr_2}\delta\Gamma = \Gamma(\bfr_2)-\Gamma(\bfr_1)\\
    \Gamma&=-V\\
    \bff_c\cdot\delta\bfr &= -\delta V
\end{align*}

In cartisian coordinates, we also have

\[\delta V = \nabla V\cdot\delta\bfr\implies \bff_c=-\nabla V\]

So the Conservative force maybe expressed as the gradient of the scalar potential energy function.


\begin{theorem}[Hamilton's Variational Principle for Conservative and Nonconservative Forces]
    \[\int_{t_0}^{t_1} \delta (T-V)+\bff_{nc}\cdot\delta \bfr \dif t=0\]
    And if there's only conservative force, we may write

    \[\delta \int_{t_0}^{t_1} L\dif t:=\delta\cA=0\]
    where \(L=T-V\) is called the Lagrangian. And \(\cA\) is the action integral. This formula is also referred as the lease action principle (LAP).
\end{theorem}


\subsection{Hamilton's Principle for Continuous Systems}

In the previouis subsection, the particles are nondeformable and thus the virtual work is only that from the external forces acting on the system. Now we turn to continuous systems comprised of deformable bodies in which the mass is distributed throughout the body.

We can apply the discrete version of the principle to differential elements of the continuous system \(\bar V\) for volumn, \(\bar A\) for area and \(\bar S\) for length, and we sum over the entire volumn through integration. Some other changes including

\begin{itemize}
    \item Lagrangian: \(L = \int \cL\dif \bar V\) where \(\cL\) is the Lagrangian density, and it equals the kinetic energy per unit minus the potential energy per unit: \(\cL=\cT-\cV\)
    \item Other than external forces, also put internal forces that deform the continuous system
\end{itemize}
  

\subsection{Euler-Lagrange Equations}

For now we only consider the discrete cases. We begin with following series of definitions and formulae. For a dynamical system with \(n\) degrees of freedom,

\begin{itemize}
    \item Degree of freedom: the minimum number of scalar dependent variables required to fully describe the motion or state of the system.
    \item Generalized coordinates (dependent): \(q_i(t)\), \(i=1,2,\dots, n\);
    \item Generalized velocities: \(\dot q_i(t)\), where dots denote differentiation wrt time;
    \item Virtual displacements: \(\delta q_i\), the infinitesimally small variations.
    \item Kinetic energy: \(T(q_1,\dots,q_n,\dot q_1,\dots,\dot q_n)\),
    \item Potential energy: \(V(q_1,\dots,q_n)\),
    \item Lagrangian: \(L (t,q_i,\dot q_i) = T(t,q_i,\dot q_i)-V(t,q_i) \),
    \item Euler-Lagrange equations for an n-DOF with conservative forces: \(\frac{\partial L}{\partial q_i}-\frac{\dif }{\dif t}\Pare{\frac{\partial L}{\partial \dot q_i}}=0\), \(i=1,2,\dots, n\);
    \item and the equivalent form: \(\frac{\partial L}{\partial q_i}-\frac{\dif }{\dif t}\Pare{\frac{\partial T}{\partial \dot q_i}}=0\) since \(\frac{\partial V}{\partial \dot q_i}=0\)
    \item The virtual work: \(\bff\cdot \delta\bfr = \sum_i \bff\cdot \frac{\partial \bfr}{\partial q_i}\delta q_i = \sum_i Q_i \delta q_i\), and we call \(Q_i=\bff\cdot\frac{\partial \bfr}{\partial q_i} \) the generalized force.
    \item The conservative force: \(Q_i=\bff\cdot\frac{\partial \bfr}{\partial q_i}=-\frac{\partial V}{\partial q_i}\)
    \item The Euler-Lagrangian equations of motion for an n-DOF system with nonconservative forces: \(\frac{\partial L}{\partial q_i}-\frac{\dif }{\dif t}\Pare{\frac{\partial L}{\partial \dot q_i}}=-Q_i\) where \(Q_i\) are the generalized forces owing to nonconservative forces acting on the system, and the potential owing to conservative forces are included in the Lagrangian \(L=T-V\).
    \item Generalized momenta: \(p_i = \frac{\partial T}{\partial \dot q_i} = \frac{\partial L}{\partial \dot q_i}\) and we can use \(\dot p_i\) to simplify the E-L equations: \(\frac{\partial L}{\partial q_i}=\dot p_i\).
\end{itemize}

\begin{proposition}[Invariance of Euler-Lagrangian Equations]
    E-L equations are independent of the coordinate system, say \(q_i\) and \(q_i^*\), \(i=1,2,\dots, n\), under the condition that matrix \(\dfrac{\partial q_j^*}{\partial q_i}\) is non-singular. Then for \(L(t,q_i,\dot q_i)=L^*(t,q_i^*, \dot q_i^*)\) we have\
    \[\frac{\partial L}{\partial q_i}-\frac{\dif }{\dif t}\Pare{\frac{\partial L }{\partial \dot q_i}}=\frac{\partial L^*}{\partial q_i^*}-\frac{\dif }{\dif t}\Pare{\frac{\partial L^* }{\partial \dot q_i^*}}=0\]
\end{proposition}

\begin{corollary}
    During the derivation, a byproduct is that
    \[\frac{\partial \dot q_j^*}{\partial \dot q_i}=\sum_i\frac{\partial q_j^*}{\partial q_i}\]
\end{corollary}

\subsection{From the First Law of Thermodynamics to the Hamilton's Principle}

\begin{theorem}[Conservation of Mass]
    For a closed system, where no matter is moving across the system boundary, then the mass of the system is a constant: \(\dot m=0\).
\end{theorem}

\begin{theorem}[The first law of thermodynamics]
    The conservation of energy: time rate of change of total energy of the system = time rate of energy transfer across the system boundary. And we call a system adiabatic if no hear transfer across the boundary. By assuming no work entering or exiting the system across the boundary, it requires that \(\dot E=0\).
\end{theorem}

To derive the Hamilton's principle, we neglect heat transfer, work crossing the system boundary, and changes in thermodynamic internal energy, thus \(E=T+V-W_{nc}=2T-L-W_{nc}\) is a constant. We also assume the system is autonomous.  The derivation is skipped. Here's a byproduct.

\begin{corollary}
    During the derivation, a byproduct is that
    \[E=2T-L-W_{nc}=\sum_i \dot q_i \frac{\partial L}{\partial \dot q_i}-L-W_{nc}\]
\end{corollary}


Also performing the inverse problem from the E-L equations can also give Hamilton's principle. We consider

\begin{align*}
    0&=\int_{t_0}^{t_1} \sum_i \Pare{\frac{\partial L}{\partial q_i}-\frac{\dif }{\dif t}\Pare{\frac{\partial L}{\partial \dot q_i}}+Q_i}\delta q_i \dif t\\
    &=- \given{\Pare{\sum_i \frac{\partial L}{\partial \dot q_i}\delta q_i}}_{t_0}^{t_1}+\int_{t_0}^{t_1} \sum_i\Pare{\frac{\partial L}{\partial q_i}\delta q_i+\frac{\partial L}{\partial \dot q_i}\delta\dot q_i+Q_i\delta q_i}\dif t\\
    &= 0+\int_{t_0}^{t_1} \Pare{\delta L+\sum_iQ_i \delta q_i}\dif t\\
    &= \int_{t_0}^{t_1} \Pare{\delta L+\bff_{nc}\cdot \delta\bfr}\dif t\\
    &= \int_{t_0}^{t_1} \Pare{\delta T+\bff     \cdot \delta\bfr}\dif t
\end{align*}

\subsection{Conservation of Mechanical Energy and the Hamiltonian}

Consider the system with only conservative forces acting on, \(Q_i=0\), \(W_{nc}=0\). The first law of thermodynamics reads \(E=H=T+V\) being a constant. Here \(H\) is the total mechanical energy of a conservative system. And similarly we have

\[H = \sum_i \dot q_i\frac{\partial L}{\partial \dot q_i}-L=\sum_i \dot q_i p_i -L\]

\begin{proposition}[Hamilton's canonical equations]
    We have following two equivalent expressions for Euler-Lagrangian Equations of motion using Hamiltonian \(i=1,2,\dots,n\), notice \(p_i\) is generalized momenta and \(L\) is assumed to have the form \(L(q_i,\dot q_i)\)
    \[\frac{\partial H}{\partial q_i}=-\dot p_i,\frac{\partial H}{\partial p_i}=\dot q_i\]
\end{proposition}

\begin{remark}
    With above two relations, we can find that \(H\) and \(L\) are the Legendre transform of each other. In ideal condition, for function \(f(x)\), we write its Legendre transform \(f^*(p)=\Pare{px-f(x)}\mid_{x=(f')^{-1}(p)}\)
\end{remark} 


\begin{theorem}[Noether's Theorem]
    This theorem will build connection between Hamilton's principle and Conservation laws. It assets that if the Lagrangian exhibits some mathematical symmetry, for which it is invariant under a transformation in one of the variables on which it depends, denoted by \(s\), then an associated property \(C\) exists that is conserved such that \(\dot C=0\).
\end{theorem}


First example starting from Hamilton's principle for conservative autonomous systems, \(L\) is independent of \(s=t\) and thus invariant, then

\[\dot E=\frac{\dif}{\dif t}\Pare{2T-L-W_{nc}}=\frac{\dif}{\dif t}\Pare{\sum_i \dot q_i \frac{\partial L}{\partial \dot q_i}-L}=0\]

so here \(C=H=T+V\).

Then consider the E-L equations, as we have proved the invariance of E-L equations under \(L=L^*\), so the variable here is \(s=q_i\). And we can also consider the case where \(L\) is independent of \(q_i\) so that \(\frac{\partial}{\partial q_i}=0\), then
 L
\[\frac{\dif}{\dif t} \Pare{\frac{\partial L}{\partial \dot q_i}}=\frac{\partial L}{\partial q_i}=0\]

So we can let \(C=\frac{\partial L}{\partial \dot q_i}=p_i\).

\section{Classical Machanics}

Other than the examples, here're some important definites and formulae.

\begin{itemize}
    \item in three-dimensional cylindrical coordinates we have the kinetic energy: \(T=\frac{1}{2}m\Pare{\dot r^2+(r\dot\theta)^2+\dot z^2}=\frac{m}{2}\Pare{\dot q_1^2+(q_1\dot q_2)^2+\dot q_3^2}\)
    \begin{itemize}
        \item The generalized force \(m\ddot q_1-mq_1\dot q_2^2=Q_1=-\frac{\partial V}{\partial q_1}\) where \(mq_1 \dot q_2^2\) is the centrifugal acceleration
        \item The generalized force \(mq_1^2\ddot q_2+2mq_1 \dot q_1\dot q_2=Q_2=-\frac{\partial Q}{\partial q_2}\) where \(2mq_1 \dot q_1\dot q_2\) is the Coriolis acceleration. 
        \item And \(m\ddot q_3=Q_3=-\frac{\partial V}{\partial q_3}\)
    \end{itemize}
    \item in the dynamics problems with constraints, we may use augmented Lagrangian \(\tilde L = T-V+\lambda(t) \phi(q_i)\). And we call holonomic constraints for those constrain the coordinates of the system.
    \item Principle of virtual work: at the static status, all forces are balanced, system do not move with time, and thus the virtual work must be zero.
    \item In statics problem, the potential energy is the total virtual work (sum or integration of \(\bff\cdot \delta \bfq\)) that would be required to move the system through a virtual displacement \(\delta\bfq\) from its unloaded condition to its fully-loaded condition, where it is in static equilibrium. 
\end{itemize}

\end{document}