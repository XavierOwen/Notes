\documentclass{article}

%%%%%%%%%%%%%%%%% this version: 03-14-2017 %%%%%%%%%%%%%%%%%%%%%%%%%%%%%%%

%%% standard packages
\usepackage{amsthm, amsmath, amssymb, amsfonts, graphicx, epsfig}
%\graphicspath{ {D:/Notes/others/assets/} }

\usepackage{algorithm, algpseudocode}

\allowdisplaybreaks
\usepackage{setspace}
%\usepackage{accents}
%%%
%\usepackage{kbordermatrix}

%%%%%%%%%%%%%%%%%%%%%%% ADDITIONAL FONTS %%%%%%%%%%%%%%%%%%%%%%%%%%%%%%
%%% Package to make \mathbbm, in particular to have 1 as for indicator function
\usepackage{bbm}
%%% Package to make special curl fonts, by using \mathscr{F}
\usepackage{mathrsfs}
%%% dsfont sometimes used instead of mathbb or mathbbm
%\usepackage{dsfont}
%%%%%%%%%%%%%%%%%%%%%%%%%%%%%%%%%%%%%%%%%%%%%%%%%%%%%%%%%%%%%%%%%%%%%%%


%%%%%%%%%%%%%%%%%%%%%%%%%%%%%%%%%%%%%%%%%%%%%%%%%%%%%%%%%%%%%%%%%%%%%%
%%% The ulem package provides various types of underlining that can
%%% stretch between words and be broken across lines. Convenient for editing. \sout{xxxx}
%%% http://ctan.unixbrain.com/macros/latex/contrib/ulem/ulem.pdf
%%%% Remark: if used with natbib, then the bibliography comes underlined. Comment this package at last compilation
%\usepackage{ulem}
%%%%%%%%%%%%%%%%%%%%%%%%%%%%%%%%%%%%%%%%%%%%%%%%%%%%%%%%%%%%%%%%%%%%%%

%%%%%%%%%%%%%%%%%%%%%%%%%%%%%%%%%%%%%%%%%%%%%%%%%%%%%%%%%%%%%%%%%%%%%%
%% Cancel is used to cross out in math mode. \xcancel, \cancel Convenient for editting.
%% http://ctan.math.utah.edu/ctan/tex-archive/macros/latex/contrib/cancel/cancel.pdf
%\usepackage{cancel}
%%%%%%%%%%%%%%%%%%%%%%%%%%%%%%%%%%%%%%%%%%%%%%%%%%%%%%%%%%%%%%%%%%%%%%


%%%%%%%%%%%%%%%%%%%%%%%%%%%%%%%%%%%%%%%%%%%%%%%%%%%%%%%%%%%%%%%%%%%%%%%
%This packages adds support of handling eps images to package graphics
%or graphicx with option pdftex. If an eps image is detected, epstopdf is
%automatically called to convert it to pdf format.
\usepackage{epstopdf}
%%%%%%%%%%%%%%%%%%%%%%%%%%%%%%%%%%%%%%%%%%%%%%%%%%%%%%%%%%%%%%%%%%%%%%%


%%%%%%%%%%%%%%%%%%%%%%%%%%%%%%%%%%%%%%%%%%%%%%%%%%%%%%%%%%%%%%%%%%%%%%%
% various features for using graphics, including subfigure, captions, subcaptions etc.
\usepackage{graphicx}
%\usepackage{caption}
\usepackage[font=sl,labelfont=bf]{caption}
\usepackage{subcaption}
%%%%%%%%%%%%%%%%%%%%%%%%%%%%%%%%%%%%%%%%%%%%%%%%%%%%%%%%%%%%%%%%%%%%%%%


%%%%%%%%%%%%%%%%%%%%%%%%%%%%%%%%%%%%%%%%%%%%%%%%%%%%%%%%%%%%%%%%%%%%%%%
% This package improves the interface for defining floating objects such
% as figures and tables in LaTeX.
% http://www.ctan.org/pkg/float
\usepackage{float}
\restylefloat{table}
%%%%%%%%%%%%%%%%%%%%%%%%%%%%%%%%%%%%%%%%%%%%%%%%%%%%%%%%%%%%%%%%%%%%%%%


%%% use for diagonals in the table's cells
%\usepackage{slashbox}  %Removed by Ares

%%%%%%%%%%%%%%%%%%%%%%%%%%%%%%%%%%%%%%%%%%%%%%%%%%%%%%%%%%%%%%%%%%%%%%%


%%%%%%%%%%%%%%%%%%%%%%%%%%%%%%%%%%%%%%%%%%%%%%%%%%%%%%%%%%%%%%%%%%%%%%%
% This package gives the enumerate environment an optional argument
% which determines the style in which the counter is printed.
% http://www.ctex.org/documents/packages/table/enumerate.pdf
\usepackage{enumerate}
%%%%%%%%%%%%%%%%%%%%%%%%%%%%%%%%%%%%%%%%%%%%%%%%%%%%%%%%%%%%%%%%%%%%%%%


%%%%%%%%%%%%%%%%%%%%%%%%%%%%%%%%%%%%%%%%%%%%%%%%%%%%%%%%%%%%%%%%%%%%%%%
%selectively in/exclude pieces of text: the user can determine new comment versions,
%and each is controlled separately. Special comments can be determined where the
%user specifies the action that is to be taken with each comment line.
% http://get-software.net/macros/latex/contrib/comment/comment.pdf
%\usepackage{comment}
%%%%%%%%%%%%%%%%%%%%%%%%%%%%%%%%%%%%%%%%%%%%%%%%%%%%%%%%%%%%%%%%%%%%%%%


%%%%%%%%%%%%%%%%%%%%%%%%%%%%%%%%%%%%%%%%%%%%%%%%%%%%%%%%%%%%%%%%%%%%%%
%% To display the labels used in a tex file in the dvi file (for example, if a theorem is labelled with the \label command) use the package
%% http://www.ctan.org/pkg/showkeys

%%%%%% Use Option 1
%%%%%% This will display all labels (for example, in the case of a labelled theorem, the label of the theorem will occur in the margin of the dvi file.
%%%%%% The command above by itself not only displays labels when they are first named but also when they are cited (or referenced).
%\usepackage{showkeys}

%%%%%% Use Option 2
%%%%%% It will NOT display citations and references. Only Labels
%\usepackage[notref, notcite]{showkeys}

%%%%%% Use Options 3
%%%%%% formats to smaller fonts the labels etc.
%\usepackage[usenames,dvipsnames]{color}
%\providecommand*\showkeyslabelformat[1]{{\normalfont \tiny#1}}
%\usepackage[notref,notcite,color]{showkeys}
%\definecolor{labelkey}{rgb}{0,0,1}
%%%%%%%%%%%%%%%%%%%%%%%%%%%%%%%%%%%%%%%%%%%%%%%%%%%%%%%%%%%%%%%%%%%%%%


%%%%%%%%%%%%%%%%%%%%%%%%%%%%%%%%%%%%%%%%%%%%%%%%%%%%%%%%%%%%%%%%%%%%%%%
% It extends the functionality of all the LATEX cross-referencing commands (including the table of contents, bibliographies etc) to produce \special commands which a driver can turn into hypertext links; it also provides new commands to allow the user to write ad hoc hypertext links, including those to external documents and URLs.
% http://www.tug.org/applications/hyperref/manual.html
\usepackage[colorlinks=true, pdfstartview=FitV, linkcolor=blue,
            citecolor=blue, urlcolor=blue]{hyperref}
\usepackage[usenames]{color}
%%%%%%%%%%%%%%%%%%%%%%%%%%%%%%%%%%%%%%%%%%%%%%%%%%%%%%%%%%%%%%%%%%%%%%%%
% custom textbox
\usepackage{tcolorbox}
\tcbuselibrary{breakable}
\tcbuselibrary{skins}
\tcbset{my left line/.style={
          enhanced, frame hidden, borderline west = {0.5pt}{0pt}{black}, % specify border
          opacityframe=0, opacityback=0,opacityfill=0, % color
          arc = 0mm, left skip=1em, % border
          left = 1mm,top=0mm,bottom=0mm,boxsep=1mm,middle=1mm, % text and border
}}

\newtcbox{\mybox}[1][]{my left line, #1}
\newtcolorbox{myleftlinebox}[1][breakable]{my left line, #1}
% usage
% use \mybox[on line]{your TEXT} for inline box otherwise forcing line breaks   
% for whole break, use \begin{myleftlinebox
%%%%%%%%%%%%%%%%%%%%%%%%%%%%%%%%%%%%%%%%%%%%%%%%%%%%%%%%%%%%%%%%%%%%%%%
\usepackage[utf8]{inputenc}


%%%%%%%%%%%%%%%%%%%%%%%%%%%%%%%%%%%%%%%%%%%%%%%%%%%%%%%%%%%%%%%%%%%%%%%
% a good looking way to format urls
% http://mirror.its.uidaho.edu/pub/tex-archive/help/Catalogue/entries/url.html
\usepackage{url}
% Define a new 'leo' style for the package that will use a smaller font.
\makeatletter\def\url@leostyle{%
 \@ifundefined{selectfont}{\def\UrlFont{\sf}}{\def\UrlFont{\scriptsize\ttfamily}}} \makeatother\urlstyle{leo}
%%%%%%%%%%%%%%%%%%%%%%%%%%%%%%%%%%%%%%%%%%%%%%%%%%%%%%%%%%%%%%%%%%%%%%%


%%%%%%%%%%%%%%%%%%%%%%%%%%%%%%%%%%%%%%%%%%%%%%%%%%%%%%%%%%%%%%%%%%%%%%%%%%%%
%%%%%% The present package defines the environment mdframed which automatically deals with page breaks in framed text.
%%%%%% http://mirrors.ibiblio.org/CTAN/macros/latex/contrib/mdframed/mdframed.pdf
\usepackage[framemethod=default]{mdframed}
%%%%%%%%%%%%%%%%%%%%%%%%%%%%%%%%%%%%%%%%%%%%%%%%%%%%%%%%%%%%%%%%%%%%%%%%%%%%

%%%%%%%%%%%%%%%%%%%%%%%%%%%%%%%%%%%%%%%%%%%%%%%%%%%%%%%%%%%%%%%%%%%%%%%%%%%%%%%%
%%%%%% An extended implementation of the array and tabular environments
%%%%%% which extends the options for column formats, and provides "programmable" format specifications.
%\usepackage{array}
%%%%%%%%%%%%%%%%%%%%%%%%%%%%%%%%%%%%%%%%%%%%%%%%%%%%%%%%%%%%%%%%%%%%%%%%%%%

%%%%%%%%%%%%%%%%%%%%%%%%%%%%%%%%%%%%%%%%%%%%%%%%%%%%%%%%%%%%%%%%%%%%%%%%%%%%
%%% The package enhances the quality of tables in LATEX, providing extra
%%% commands as well as behind-the-scenes optimization. Guidelines
%%% are given as to what constitutes a good table in this context.
%%% From version 1.61, the package offers long table compatibility.
%\usepackage{booktabs}
%%%%%%%%%%%%%%%%%%%%%%%%%%%%%%%%%%%%%%%%%%%%%%%%%%%%%%%%%%%%%%%%%%%%%%%%%%%%%%


\usepackage{tikz}




%% or use GEOMETRY package
\usepackage[margin=1.0in, letterpaper]{geometry}
%\def\baselinestretch{1.1}


%%%%%%%%%%%%% OR USE EXACT DIMENSIONS %
%\setlength{\textwidth}{6.5in}     %%
%\setlength{\oddsidemargin}{0in}   %%
%\setlength{\evensidemargin}{0in}  %%
%\setlength{\textheight}{8.5in}    %%
%\setlength{\topmargin}{0in}       %%
%\setlength{\headheight}{0in}      %%
%\setlength{\headsep}{.3in}         %%
%\setlength{\footskip}{.5in}       %%
%%%%%%%%%%%%%%%%%%%%%%%%%%%%%%%%%%%%%%%%%%%%%%%%%%%%%%%%%%%%%%%%%%%%%%%


%%%%%%%%%%%%%%%%%%%%%%%%%%%%%%%%%%%%%%%%%%%%%%%%%%%%%%%%%%%%%%%%%%%%%%%
%%%%%%%%%%%%%%%%%%%%%%%% NUMBERING %%%%%%%%%%%%%
\newtheorem{theorem}{Theorem}
\newtheorem{conjecture}{Conjecture}
\newtheorem{conclusion}{Conclusion}
\newtheorem{proposition}[theorem]{Proposition}
\newtheorem{lemma}[theorem]{Lemma}
\newtheorem{corollary}[theorem]{Corollary}
\newtheorem{assumption}{Assumption}
\newtheorem{condition}{C}
\theoremstyle{definition}
\newtheorem{definition}[theorem]{Definition}
\newtheorem{example}[theorem]{Example}
\theoremstyle{remark}
\newtheorem{remark}[theorem]{Remark}
\newtheorem{question}[theorem]{Question}
\newtheorem{problem}[theorem]{Problem}
\newtheorem{NB}[theorem]{Nota Bene}

\numberwithin{equation}{section}
\numberwithin{theorem}{section}
%\renewcommand{\labelitemi}{ {\small $\rhd$}}
%%%%%%%%%%%%%%%%%%%%%%%%%%%%%%%%%%%%%%%%%%%%%%%%%%%%%%%%%%%%%%%%%%%%%%%

%%%%%%%%%%%%%%%%%%%%%%%%%%%%%%%%%%%%%%%%%%%%%%%%%%%%%%%%%%%%%%%%%%%%%%%
\definecolor{Red}{rgb}{0.9,0,0.0}
\definecolor{Blue}{rgb}{0,0.0,1.0}
%%%%%%%%%%%%%%%%%%%%%%%%%%%%%%%%%%%%%
%%% used for editing and making comments in color
% Example \ig{Remarks and Commets}
\newcommand{\ig}[1]{\textcolor[rgb]{0.00, 0.0, 1.0}{{\tiny \textsuperscript{[\textrm{IC:Rem}]}} \ #1}}
\newcommand{\igAdd}[1]{\textcolor[rgb]{0.7, 0.0, 0.0}{{\tiny \textsuperscript{[\textrm{IC:Add}]}} \ #1}}
\newcommand{\igEdit}[1]{\textcolor[rgb]{0.7, 0.0, 0.0}{{\tiny \textsuperscript{[\textrm{IC:Edit}]}} \  #1}}

\newcommand{\trb}[1]{\begin{color}[rgb]{0.98, 0.0, 0.98}{TRB: #1} \end{color}}
\newcommand{\ti}[1]{\begin{color}[rgb]{1.00,0.00,0.00}{TI: #1} \end{color}}
\newcommand{\Red}[1]{\textcolor{Red}{#1}}
%%%%%%%%%%%%%%%%%%%%%%%%%%%%%%%%%%%%%

%%%%%%%%%%%%%%%%%%%%%%%%%%%%%%%%%%%%%
%%%     Igor's macros
%% \mathcal Letters
\def\cA{\mathcal{A}}
\def\cB{\mathcal{B}}
\def\cC{\mathcal{C}}
\def\cD{\mathcal{D}}
\def\cE{\mathcal{E}}
\def\cF{\mathcal{F}}
\def\cG{\mathcal{G}}
\def\cH{\mathcal{H}}
\def\cI{\mathcal{I}}
\def\cJ{\mathcal{J}}
\def\cK{\mathcal{K}}
\def\cL{\mathcal{L}}
\def\cM{\mathcal{M}}
\def\cN{\mathcal{N}}
\def\cO{\mathcal{O}}
\def\cP{\mathcal{P}}
\def\cQ{\mathcal{Q}}
\def\cR{\mathcal{R}}
\def\cS{\mathcal{S}}
\def\cT{\mathcal{T}}
\def\cU{\mathcal{U}}
\def\cV{\mathcal{V}}
\def\cW{\mathcal{W}}
\def\cX{\mathcal{X}}
\def\cY{\mathcal{Y}}
\def\cZ{\mathcal{Z}}

%% \mathbb Letters
\def\bA{\mathbb{A}}
\def\bB{\mathbb{B}}
\def\bC{\mathbb{C}}
\def\bD{\mathbb{D}}
\def\bE{\mathbb{E}}
\def\bF{\mathbb{F}}
\def\bG{\mathbb{G}}
\def\bH{\mathbb{H}}
\def\bI{\mathbb{I}}
\def\bJ{\mathbb{J}}
\def\bK{\mathbb{K}}
\def\bL{\mathbb{L}}
\def\bM{\mathbb{M}}
\def\bN{\mathbb{N}}
\def\bO{\mathbb{O}}
\def\bP{\mathbb{P}}
\def\bQ{\mathbb{Q}}
\def\bR{\mathbb{R}}
\def\bS{\mathbb{S}}
\def\bT{\mathbb{T}}
\def\bU{\mathbb{U}}
\def\bV{\mathbb{V}}
\def\bW{\mathbb{W}}
\def\bX{\mathbb{X}}
\def\bY{\mathbb{Y}}
\def\bZ{\mathbb{Z}}

%% \mathscr Letters, for filtration, sigma algebras etc
\def\sA{\mathscr{A}}
\def\sB{\mathscr{B}}
\def\sC{\mathscr{C}}
\def\sD{\mathscr{D}}
\def\sE{\mathscr{E}}
\def\sF{\mathscr{F}}
\def\sG{\mathscr{G}}
\def\sH{\mathscr{H}}
\def\sI{\mathscr{I}}
\def\sJ{\mathscr{J}}
\def\sK{\mathscr{K}}
\def\sL{\mathscr{L}}
\def\sM{\mathscr{M}}
\def\sN{\mathscr{N}}
\def\sO{\mathscr{O}}
\def\sP{\mathscr{P}}
\def\sQ{\mathscr{Q}}
\def\sR{\mathscr{R}}
\def\sS{\mathscr{S}}
\def\sT{\mathscr{T}}
\def\sU{\mathscr{U}}
\def\sV{\mathscr{V}}
\def\sW{\mathscr{W}}
\def\sX{\mathscr{X}}
\def\sY{\mathscr{Y}}
\def\sZ{\mathscr{Z}}


%%%% \mathsf for Matrices
\def\mA{\mathsf{A}}
\def\mB{\mathsf{B}}
\def\mC{\mathsf{C}}
\def\mD{\mathsf{D}}
\def\mE{\mathsf{E}}
\def\mF{\mathsf{F}}
\def\mG{\mathsf{G}}
\def\mH{\mathsf{H}}
\def\mI{\mathsf{I}}
\def\mJ{\mathsf{J}}
\def\mK{\mathsf{K}}
\def\mL{\mathsf{L}}
\def\mM{\mathsf{M}}
\def\mN{\mathsf{N}}
\def\mO{\mathsf{O}}
\def\mP{\mathsf{P}}
\def\mQ{\mathsf{Q}}
\def\mR{\mathsf{R}}
\def\mS{\mathsf{S}}
\def\mT{\mathsf{T}}
\def\mU{\mathsf{U}}
\def\mV{\mathsf{V}}
\def\mW{\mathsf{W}}
\def\mX{\mathsf{X}}
\def\mY{\mathsf{Y}}
\def\mZ{\mathsf{Z}}


%%%% \mathbf for Matrices or vectors
\def\bfB{\boldsymbol{B}}
\def\bfC{\boldsymbol{C}}
\def\bfD{\boldsymbol{D}}
\def\bfA{\boldsymbol{A}}
\def\bfE{\boldsymbol{E}}
\def\bfF{\boldsymbol{F}}
\def\bfG{\boldsymbol{G}}
\def\bfH{\boldsymbol{H}}
\def\bfI{\boldsymbol{I}}
\def\bfJ{\boldsymbol{J}}
\def\bfK{\boldsymbol{K}}
\def\bfL{\boldsymbol{L}}
\def\bfM{\boldsymbol{M}}
\def\bfN{\boldsymbol{N}}
\def\bfO{\boldsymbol{O}}
\def\bfP{\boldsymbol{P}}
\def\bfQ{\boldsymbol{Q}}
\def\bfR{\boldsymbol{R}}
\def\bfS{\boldsymbol{S}}
\def\bfT{\boldsymbol{T}}
\def\bfU{\boldsymbol{U}}
\def\bfV{\boldsymbol{V}}
\def\bfW{\boldsymbol{W}}
\def\bfX{\boldsymbol{X}}
\def\bfY{\boldsymbol{Y}}
\def\bfZ{\boldsymbol{Z}}

%%%% \mathbf for Matrices
\def\bfa{\boldsymbol{a}}
\def\bfb{\boldsymbol{b}}
\def\bfc{\boldsymbol{c}}
\def\bfd{\boldsymbol{d}}
\def\bfe{\boldsymbol{e}}
\def\bff{\boldsymbol{f}}
\def\bfg{\boldsymbol{g}}
\def\bfh{\boldsymbol{h}}
\def\bfi{\boldsymbol{i}}
\def\bfj{\boldsymbol{j}}
\def\bfk{\boldsymbol{k}}
\def\bfl{\boldsymbol{l}}
\def\bfm{\boldsymbol{m}}
\def\bfn{\boldsymbol{n}}
\def\bfo{\boldsymbol{o}}
\def\bfp{\boldsymbol{p}}
\def\bfq{\boldsymbol{q}}
\def\bfr{\boldsymbol{r}}
\def\bfs{\boldsymbol{s}}
\def\bft{\boldsymbol{t}}
\def\bfu{\boldsymbol{u}}
\def\bfv{\boldsymbol{v}}
\def\bfw{\boldsymbol{w}}
\def\bfx{\boldsymbol{x}}
\def\bfy{\boldsymbol{y}}
\def\bfz{\boldsymbol{z}}

%%%% \boldsymbol for lower case greek letter in math mode
\def\bfalpha{\boldsymbol{\alpha}}
\def\bfbeta{\boldsymbol{\beta}}
\def\bfgamma{\boldsymbol{\gamma}}
\def\bfdelta{\boldsymbol{\delta}}
\def\bfepsilon{\boldsymbol{\epsilon}}
\def\bfzeta{\boldsymbol{\zeta}}
\def\bfeta{\boldsymbol{\eta}}
\def\bftheta{\boldsymbol{\theta}}
\def\bfiota{\boldsymbol{\iota}}
\def\bfkappa{\boldsymbol{\kappa}}
\def\bflambda{\boldsymbol{\lambda}}
\def\bfmu{\boldsymbol{\mu}}
\def\bfnu{\boldsymbol{\nu}}
\def\bfomicron{\boldsymbol{\omicron}}
\def\bfpi{\boldsymbol{\pi}}
\def\bfrho{\boldsymbol{\rho}}
\def\bfsigma{\boldsymbol{\sigma}}
\def\bftau{\boldsymbol{\tau}}
\def\bfupsilon{\boldsymbol{\upsilon}}
\def\bfphi{\boldsymbol{\phi}}
\def\bfchi{\boldsymbol{\chi}}
\def\bfpsi{\boldsymbol{\psi}}
\def\bfomega{\boldsymbol{\omega}}

%%%% \boldsymbol for upper case greek letter in math mode
\def\bfPhi{\boldsymbol{\Phi}}
\def\bfTheta{\boldsymbol{\Theta}}
\def\bfPsi{\boldsymbol{\Psi}}
\def\bfOmega{\boldsymbol{\Omega}}
\def\bfSigma{\boldsymbol{\Sigma}}


%%%%%%%%%%%%%%%%%% Shortcuts %%%%%%%%%%%%%%%%%%%%%%%%%%%%%%%%%%%%%%%%%%%
\newcommand{\wt}{\widetilde}
\newcommand{\wh}{\widehat}

%%%%%%%%%%%%%%%%%%%%%%%%%%%%%%%%%%%%%%%%%%%%%%%%%%%%%%%%%%%%%%%%%%%%%%%%%%%%%%%%%
%%%%%%%%%%%%%%%%%%   Nonstandard notations  %%%%%%%%%%%%%%%%%%%%%%%%%%%%%%%%%%%%%
\newcommand{\pd}[1]{\partial_{#1}}      % partial derivative
\newcommand{\1}{\mathbbm{1}}            % preferable way of writing indicator function
\newcommand{\set}[1]{\{#1\}}            % set: {xyz} to be used for inline formulas
\newcommand{\Set}[1]{\left\{#1\right\}} % set: {xyz} to be used for separate (not inline) formulas
\renewcommand{\mid}{\,|\,}              % mid bar with small spaces before and after: x | y
\newcommand{\Mid}{\,\Big | \,}          % big bar with small spaces before and after:
\newcommand{\norm}[1]{ \| #1 \| }       % mid bar with small spaces before and after: x | y
\newcommand{\abs}[1]{\left\vert#1\right\vert}   % absolute value
\newcommand{\CB}[1]{\left\{ #1 \right\}}
\newcommand{\SB}[1]{\left[ #1 \right]}
\newcommand{\Pare}[1]{\left( #1 \right)}
\newcommand{\AB}[1]{\left \langle #1 \right \rangle}
\newcommand{\given}[1]{\left.#1\right|}
\newcommand{\givenAlt}[1]{\left.#1\right]}
\newcommand{\Tran}[1]{{#1}^\top}
\newcommand{\bsde}{BS$\Delta$E}         % BS\DeltaE
\newcommand{\bsdes}{BS$\Delta$Es}       % BS\DeltaE
\newcommand{\ow}{\text{otherwise}}
\newcommand{\imblies}{\Longleftarrow}
\newcommand{\Exp}[1]{\mathrm{E}\left[ #1 \right]}
\newcommand{\Var}[1]{\mathrm{Var}\left[ #1 \right]}
\newcommand{\Cov}[1]{\mathrm{Cov}\left[ #1 \right]}
\newcommand{\Bspace}{\;\;\;\;}
\newcommand{\dif}{\,\mathrm{d}}        % used for differential, same as in commath.sty

\newcommand{\LHS}{\text{LHS}} % left hand side
\newcommand{\RHS}{\text{RHS}} % right hand side
\newcommand{\Adj}{\text{Adj}} %adjoint matrix

\newcommand{\RNum}[1]{\uppercase\expandafter{\romannumeral #1\relax}} % romanian numerals

\DeclareMathOperator{\logit}{logit}
\DeclareMathOperator*{\esssup}{ess\,sup} % ess sup
\DeclareMathOperator*{\essinf}{ess\,inf} % ess inf
\DeclareMathOperator*{\esslimsup}{ess\,\limsup}
\DeclareMathOperator*{\essliminf}{ess\,\liminf}
\DeclareMathOperator*{\argmin}{arg\,min} % argmin
\DeclareMathOperator*{\diag}{diag} % diag
\DeclareMathOperator*{\argmax}{arg\,max} % argmax
\DeclareMathOperator*{\Arg}{Arg} % arguments
\DeclareMathOperator*{\rank}{rank\,} % argmax
\DeclareMathOperator*{\KL}{KL} % KL divergence
\DeclareMathOperator*{\Proj}{Proj} % Projection
\DeclareMathOperator{\Std}{\mathrm{Std}} % \std for Standard deviation
\DeclareMathOperator{\sgn}{\mathrm{sgn}} % sign of a variable
\DeclareMathOperator{\tr}{\mathrm{trace}} % matrix trace
% trigonometric and hyperbolic functions
\DeclareMathOperator{\sech}{sech}
\DeclareMathOperator{\csch}{csch}
\DeclareMathOperator{\arcsec}{arcsec}
\DeclareMathOperator{\arccot}{arcCot}
\DeclareMathOperator{\arccsc}{arcCsc}
\DeclareMathOperator{\arccosh}{arcCosh}
\DeclareMathOperator{\arcsinh}{arcsinh}
\DeclareMathOperator{\arctanh}{arctanh}
\DeclareMathOperator{\arcsech}{arcsech}
\DeclareMathOperator{\arccsch}{arcCsch}
\DeclareMathOperator{\arccoth}{arcCoth} 

%%%%%%%%%%%%%%%%%%%%%%%%%%%%%%% FINANCE %%%%%%%%%%%%%%%%%%%%%%%%%%%%%%%%%%%%%%%%%%%%%%%%%%
\DeclareMathOperator{\var}{\mathrm{V}@\mathrm{R}}           % \V@R Value-at-risk
\DeclareMathOperator{\tce}{\mathrm{TCE}}                    % Tail Conditional Expectation
\DeclareMathOperator{\tvar}{\mathrm{TV}@\mathrm{R}}         % \TV@R tail Value-at-risk
\DeclareMathOperator{\avar}{\mathrm{AV}@\mathrm{R}}         % \AV@R average Value-at-risk
\DeclareMathOperator{\ent}{\mathrm{Ent}}                    % \ent = Entropic Risk Measure
\DeclareMathOperator{\glr}{\mathrm{GLR}}                    % \glr = gain to loss ratio

\DeclareMathOperator{\ES}{\mathrm{ES}}

\newcommand{\ask}{\text{ask}}           % ask price
\newcommand{\bid}{\text{bid}}           % bid price

%%%%%%%%%%%%%%%%%%%%%%%%%%%%%%%%%%%%%%%%%%%%%%%%%%%%%%%
% \bibliographystyle{amsplain}
% \bibliographystyle{alpha} % standard LaTeX bibliography format. Preferable to be used
% \bibliography{MathFinanceMaster-12-28-2014}
%\bibliography{D:/_research/latex/lib_igor/igor_bib_mathfinance}
% help on how to use several bit files \bibliography{videogames,comics,interface,theory}
%%%%%%%%%%%%%%%%%%%%%%%%%%%%%%%%%%%%%%%%%%%%%%%%%%%%%%%




\title{HW2}
\author{Yuanxing Cheng, A20453410, MMAE-501-f22}

\begin{document}

\maketitle

\section*{1}


\begin{myleftlinebox}
    Solve the system 
    \begin{align*}
        \frac{\dif x_1}{\dif t} = x_1 + 2 x_2\\
        \frac{\dif x_2}{\dif t} = 2 x_1 + x_2
    \end{align*}
    subject to initial condition \(x_1(0) = 1\), \(x_2(0) = 3\).
    \tcblower
    Write the equation as \(\frac{\dif X}{\dif t} = AX\) where \(X = [x_1,x_2]^\top\) and \(A = \begin{bmatrix}
        1 & 2 \\
        2 & 1
    \end{bmatrix}\)
    For matrix \(A\) we find its eigenvalue and eigenvector pairs. We first solve \(\det (\lambda I-A) = (\lambda-1)^2-(-2)^2 = (\lambda-3)(\lambda+1)=0\) so \(\lambda_1 = 3\) and \(\lambda_2 = -1\). Their corresponding eigenvectors are obtained by solving \(\lambda_i I-A=0\) which gives \(x_1 = [1, 1]^\top\) and \(x_2 = [1,-1]^\top\). So we write the solution in the matrix form as

    \begin{equation*}
        X(t) = [x_1,x_2][c_1 e^{\lambda_1 t}, c_2 e^{\lambda_2 t}]^\top = \begin{bmatrix}
            1 & 1\\
            1 & -1
        \end{bmatrix}\begin{bmatrix}
            c_1 e^{3 t}\\
            c_2 e^{-t}
        \end{bmatrix}
    \end{equation*}
    plug in intial conditions we have \(X(0) = [1,3]^\top = \begin{bmatrix}
        1 & 1\\
        1 & -1
    \end{bmatrix}[c_1,c_2]^\top\), then solve this we have \([c_1,c_2]=[2,-1]\). The final solution is then 
    \begin{align*}
        x_1(t) &= 2 e^{3t} - e^{-t}\\
        x_2(t) &= 2 e^{3t} + e^{-t}
    \end{align*}
\end{myleftlinebox}

\section*{2}
\begin{myleftlinebox}
    Verify by direct calculation that the matrix \(A = \begin{bmatrix}
        -2 & -3 & -1\\
        1 & 2 & 1\\
        3 & 3 & 2
    \end{bmatrix}\) satisfies the Cayley–Hamilton theorem.
    \tcblower
    \begin{equation*}
        \begin{split}
            p_A(\lambda) &= (\lambda+2)(\lambda-2)(\lambda-2)+9+3+3(\lambda-2)+3(\lambda-2)-3(\lambda+2)\\
            &= \lambda^3-2\lambda^2-4\lambda+8+12+3\lambda-18\\
            &= \lambda^3-2\lambda^2-\lambda+2\\
        \end{split}
    \end{equation*}
    Next we plug in \(\lambda=A\), we obtain the following:
    \begin{equation*}
        \begin{split}
            p_A(A) &= \begin{bmatrix}
                -8 & -9 & -7 \\
                7 & 8 & 7\\
                9 & 9 & 8
            \end{bmatrix}-2\begin{bmatrix}
                -2 & -3 & -3\\
                3 & 4 & 3\\
                3 & 3 & 4
            \end{bmatrix}-\begin{bmatrix}
                -2 & -3 & -1\\
                1 & 2 & 1\\
                3 & 3 & 2
            \end{bmatrix}+\begin{bmatrix}
                2 & 0 & 0\\
                0 & 2 & 0\\
                0 & 0 & 2
            \end{bmatrix}\\
            &=\begin{bmatrix}
                -8+4+2+2 & -9+6+3+0 & -7+6+1+0\\
                7-6-1+0 & 8-8-2+2 & 7-6-1+0\\
                9-6-3-0 & 9-6-3-0 & 8-8-2+2    
            \end{bmatrix} = 0
        \end{split}
    \end{equation*}
\end{myleftlinebox}

\section*{3}
\begin{myleftlinebox}
    Determine the general solution of the following system of equations by diagonalization
    \begin{align*}
        \dot x_1 &= -10 x_1-18 x_2+t\\
        \dot x_2 &= 6 x_1 +11 x_2+3
    \end{align*}
    \tcblower
    We first rewrite the original system as \(\dot X(t) = AX(t)+f(t)\) where \(A = \begin{bmatrix}
        -10 & -18\\
        6 & 11
    \end{bmatrix}\) and \(f(t) = [t,3]^\top\). Use the same way, we first get the eigenvalues of matrix \(A\), \(\lambda_1 = 2\) and \(\lambda_2 = -1\), and corresponding eigenvectors are \([3,-2]^\top\) and \([2,-1]^\top\) so we have the diagonalization \(A = B\Lambda B^{-1}\) where \(\Lambda = \begin{bmatrix}
        2 & 0 \\
        0 & -1
    \end{bmatrix}\) and \(B = \begin{bmatrix}
        3 & 2\\
        -2 & -1
    \end{bmatrix}\). We can then let \(X = BY\) and left multiply \(B^{-1}\) on both side and obtain the following equation.
    \begin{equation*}
        \dot Y(t) = \Lambda Y(t)+B^{-1}f(t)
    \end{equation*}
    We first solve the homogenous equation \(y_1(t) = c_1 e^{2t}\) and \(y_2(t)=c_2 e^{-t}\); then guess the particular solution. 
    \begin{equation*}
        B^{-1}f(t) = [-t-6,2t+9]^\top\implies y_1(t) = \frac{1}{2}t+\frac{13}{4}, y_2(t) = 2t+7
    \end{equation*}
    We can then combine these two results and obtain the final solution for \(Y\) as
    \begin{equation*}
        y_1(t) = c_1 e^{2t} +\frac{1}{2}t+\frac{13}{4},y_2(t) = c_2 e^{-t}+ 2t+7
    \end{equation*}
    So that
    \begin{equation*}
        x_1(t) = 3y_1(t)+2y_2(t)= 3c_1 e^{2t}+2c_2 e^{-t} +\frac{11}{2}t+\frac{95}{4}, x_2(t) = -2y_1(t)-y_2(t) = -2c_1 e^{2t}-c_2 e^{-t} -3t-\frac{27}{2}
    \end{equation*}
\end{myleftlinebox}

\section*{4}
\begin{myleftlinebox}
    Determine the general solution of the following system of equations by diagonalization
    \begin{equation*}
        \begin{split}
            \dot x_1 &= -2x_1+2x_2+2x_3+\sin t\\
            \dot x_2 &= -x_2 + 3\\
            \dot x_3 &= -2x_1+4 x_2 + 3x_3
        \end{split}
    \end{equation*}
    \tcblower
    Similarly, we write \(\dot X = AX+f\) where \(A = \begin{bmatrix}
        -2 & 2 & 2\\
        0 & -1 & 0\\
        -2 & 4 & 3
    \end{bmatrix}\) and \(f=[\sin t,3,0]^\top\). We first find \(A\)'s eigenvalues \(\lambda_1 = 2\) and \(\lambda_2 = -1\), and corresponding eigenvectors \([1,0,2]^\top\), \([2,0,1]^\top\) and \([2,1,0]^\top\), so we have the diagonalization \(A = B\Lambda B^{-1}\) where \(\Lambda=\begin{bmatrix}
        2 & 0 & 0\\
        0 & -1 & 0\\
        0 & 0 & -1
    \end{bmatrix}\) and \(B = \begin{bmatrix}
        1 & 2 & 2\\
        0 & 1 & 0\\
        2 & 0 & 1
    \end{bmatrix}\). We then let \(X = BY\) and left multiply \(B^{-1}=\frac{1}{3}\begin{bmatrix}
        -1 & 2 & 2\\
        0 & 3 & 0\\
        2 & -4 & -1
    \end{bmatrix}\) and this gives \(\dot Y(t) = \Lambda Y(t) + B^{-1} f(t)\). So that \(y_1(t) = c_1 e^{2t}, y_{2,3}(t) = c_{2,3} e^{-t}\). Then we guess the particular solution using undetermined coefficients.
    \begin{equation*}
        \begin{split}
            C \cos t-D\sin t+E &= 2  C\sin t+2D\cos t+2E t+2F+\frac{1}{3}\Pare{-\sin t+6}\\
            \implies y_1(t) &= \frac{2}{15}\sin t+\frac{1}{15}\cos t-1\\
            C \cos t-D\sin t+E &= -C\sin t-D\cos t-Et-F+\frac{1}{3}(3*3)\\
            \implies y_2(t) &= 3\\
            C \cos t-D\sin t+E &=-C\sin t-D\cos t-Et-F+\frac{1}{3}\Pare{2\sin t-12}\\
            \implies y_3(t) &= \frac{1}{3}\sin t-\frac{1}{3}\cos t-4
        \end{split}
    \end{equation*}
    Then the general solution for \(Y\) is
    \begin{align*}
        y_1(t) &= c_1 e^{2t}+\frac{2}{15}\sin t+\frac{1}{15}\cos t-1\\
        y_2(t) &= c_2 e^{-t}+3\\
        y_3(t) &= c_3 e^{-t}+\frac{1}{3}\sin t-\frac{1}{3}\cos t-4
    \end{align*}
    And finally the general solution for \(X\) is 
    \begin{align*}
        x_1(t) &= y_1(t)+2y_2(t)+2y_3(t)\\
        &= c_1 e^{2t}+2(c_2+c_3)e^{-t}+\frac{4}{5}\sin t-\frac{3}{5}\cos t-3\\
        x_2(t) &= y_2(t)\\
        &= c_2 e^{-t}+3\\
        x_3(t) &= 2y_1(t)+y_3(t)\\
        &= 2c_1 e^{2t}+c_3 e^{-t}+\frac{3}{5}\sin t-\frac{1}{5}\cos t-6
    \end{align*}
\end{myleftlinebox}

\section*{5}
\begin{myleftlinebox}
    Determine the singular value decomposition of the matrix \(\begin{bmatrix}
        0 & 1 & 1\\
        \sqrt{2} & 2 & 0\\
        0 & 1 & 1
    \end{bmatrix}\). Compare the singular values and singular vectors of \(A\) to its eigenvalues and eigenvectors.
    \tcblower
    To find the singular values, we find the eigenvalues of \(AA^\top=\begin{bmatrix}
        2 & 2 & 2\\
        2 & 6 & 2\\
        2 & 2 & 2
    \end{bmatrix}\) which are \(8,2,0\) and corresponding eigenvectors are \([1,2,1]^\top,[1,-1,1]^\top,[-1,0,1]^\top\). So that \(U=\begin{bmatrix}
        1/\sqrt 6 & 1/\sqrt 3 & -1/\sqrt 2\\
        2/\sqrt 6 & -1/\sqrt 3 & 0\\
        1/\sqrt 6 & 1/\sqrt 3 & 1/\sqrt 2
    \end{bmatrix}\). Then \(v_1 = \frac{1}{\sqrt 8} A^\top u_1 = [\sqrt 6/6,\sqrt 3/2,\sqrt 3/6]^\top\), similarly, \(v_2 = [-\sqrt 3/3,0,\sqrt 6/3]^\top\), and for \(v_3\), just make \(v\)s an orthonormal basis. From \(v_2\),  we choose \(v_3 = [\sqrt 2,\eta, 1]\), then since \(v_1v_3=0\), we see \(\eta=-1\), after normalization, we obtain \(V = \begin{bmatrix}
        \sqrt6/6 & -\sqrt 3/3 & \sqrt 2/2\\
        \sqrt3/2 & 0 & -1/2\\
        \sqrt3/6 & \sqrt6/3 & 1/2
    \end{bmatrix}\). So that \(A = U\Sigma V^\top\) where \(\Sigma=\begin{bmatrix}
        2\sqrt 2 & 0 & 0\\\
        0 & \sqrt 2 & 0\\
        0 &0 & 0
    \end{bmatrix}\).
    
    Next we compare above with its eigenvalues and eigenvectors. We first obtain its characteristic polynomial as \(\lambda^3-3\lambda^2+(2-\sqrt 2)\lambda\) so \(\lambda_1=0\), \(\lambda_{2,3} = \frac{3\pm \sqrt{1+4\sqrt 2}}{2}\) and corresponding eigenvectors are \([\sqrt 2,-1,1]^\top\), \([1,\frac{1\pm \sqrt{1+4\sqrt 2}}{2},1]^\top\). They aren't very related.
\end{myleftlinebox}

\section*{6}

\begin{myleftlinebox}
    Consider the matrix \(\begin{bmatrix}
        1 & 0 & 1 & 0 & 1 & 0 & 1\\
        0 & 1 & 0 & 1 & 0 & 1 & 0
    \end{bmatrix}\) compute the left singular vectors and singular values of \(A\); then using the result find the first two right singular vectors of \(A\).
    \tcblower
    For the left singular values, we first compute \(AA^\top=\begin{bmatrix}
        4 & 0\\
        0 & 3
    \end{bmatrix}\) so matrix \(A\) has singular value \(\sqrt 4=2\), \( \sqrt 3\). And the singular vectors are the eigenvectors of \(AA^\top\) which are \(u_1=[1,0]^\top\) and \(u_2=[0,1]^\top\). 

    The first 2 right singular values can be found using the formula. \(v_1 = \frac{1}{2}A^\top u_1 = \frac{1}{2}[1,0,1,0,1,0,1]^\top\) and \(v_2 = \frac{1}{\sqrt 3}A^\top u_2 = \frac{1}{\sqrt 3}[0,1,0,1,0,1,0]^\top\).
\end{myleftlinebox}

%\newpage

%\bibliographystyle{acm}
%\bibliography{D:/Papers.bib}

%\newpage
%\appendix
%\section{Appendix}

\end{document}